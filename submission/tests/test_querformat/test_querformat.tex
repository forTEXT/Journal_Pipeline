% Options for packages loaded elsewhere
\PassOptionsToPackage{unicode}{hyperref}
\PassOptionsToPackage{hyphens}{url}
\PassOptionsToPackage{dvipsnames,svgnames,x11names}{xcolor}
%
\documentclass[
          a4paper,
        ]{article}
\raggedbottom
\usepackage{amsmath,amssymb}
\usepackage{setspace}
\usepackage{iftex}
\RequireLuaTeX

\usepackage{lscape} % für Querformat-Seiten
\usepackage{hyperref}

\usepackage{unicode-math} % this also loads fontspec
\defaultfontfeatures{Scale=MatchLowercase}
\defaultfontfeatures[\rmfamily]{Ligatures=TeX,Scale=1}

  \usepackage[]{plex-otf}
  
  
  
  
  
  
  
\directlua{luaotfload.add_fallback
   ("fallbacks",
    {
      "STIX Two Math:;",
      "NotoColorEmoji:mode=harf;",
    }
   )}
\defaultfontfeatures{RawFeature={fallback=fallbacks}}

\usepackage{academicons}
\usepackage{fontawesome5}
\usepackage{ccicons}

  
\IfFileExists{upquote.sty}{\usepackage{upquote}}{}
\IfFileExists{microtype.sty}{% use microtype if available
  \usepackage[]{microtype}
  \UseMicrotypeSet[protrusion]{basicmath} % disable protrusion for tt fonts
}{}
\usepackage{lua-widow-control}
  \makeatletter
\@ifundefined{KOMAClassName}{% if non-KOMA class
  \IfFileExists{parskip.sty}{%
    \usepackage{parskip}
  }{% else
    \setlength{\parindent}{0pt}
    \setlength{\parskip}{6pt plus 2pt minus 1pt}}
}{% if KOMA class
  \KOMAoptions{parskip=half}}
\makeatother
  
  
\usepackage[dvipsnames,svgnames,x11names]{xcolor}

\definecolor{oa-orange}     {RGB} {246, 130, 18}
\definecolor{ft-red}        {RGB} {168, 0, 0}
\definecolor{orcid-green}   {RGB} {166, 206, 57}
\definecolor{fortext-green}   {RGB} {26, 73, 76}


\colorlet{oa-orange}        {black}

\colorlet{highlightcolor1}  {DarkOrange}
\colorlet{highlightcolor2}  {DarkBlue}
\colorlet{highlightcolor3}  {DarkGreen}
\colorlet{highlightcolor4}  {DarkMagenta}

\colorlet{filecolor}        {highlightcolor1}
\colorlet{linkcolor}        {fortext-green}
\colorlet{citecolor}        {fortext-green}
\colorlet{urlcolor}         {fortext-green}


\NewDocumentCommand \oalogo { }
{%
  \textcolor{oa-orange}{\sffamily%\bfseries%
  open \aiOpenAccess\ access}
}


\NewDocumentCommand \orcidlink { m }
{%
	\texorpdfstring
	{\href{https://orcid.org/#1}{\textcolor{orcid-green}{\raisebox{-.2ex}{\aiOrcid}}}}
	{https://orcid.org/#1}%
}

\NewDocumentCommand \emaillink { m }
{%
	\texorpdfstring
	{\href{mailto:#1}{\textcolor{black}{\raisebox{-.2ex}{\faEnvelopeOpen[regular]}}}}
	{mailto:#1}%
}

  \usepackage{geometry}
\geometry{
	paper=a4paper,
	top=25mm,
	bottom=20mm,
	right=30mm,
	left=30mm,
	footskip=10mm,
	% showframe,
}
  
  \usepackage{listingsutf8}
\lstset{
  language=python,
  basicstyle=\ttfamily\footnotesize,
  columns=fullflexible,
  xleftmargin=2em,
  xrightmargin=2em,
  % frame=single,
  breaklines=true,
  postbreak=\mbox{\textcolor{highlightcolor1}{\(\hookrightarrow\)}\space},
}
\newcommand{\passthrough}[1]{#1}
\lstset{defaultdialect=[5.3]Lua}
\lstset{defaultdialect=[x86masm]Assembler}
  
  
  
\usepackage{longtable,booktabs,array}
\usepackage{tabularray}
\usepackage{ifthen}

  
\usepackage{calc} % for calculating minipage widths
% Correct order of tables after \paragraph or \subparagraph
\usepackage{etoolbox}
\makeatletter
\patchcmd\longtable{\par}{\if@noskipsec\mbox{}\fi\par}{}{}
\makeatother
% Allow footnotes in longtable head/foot
\IfFileExists{footnotehyper.sty}{\usepackage{footnotehyper}}{\usepackage{footnote}}
\makesavenoteenv{longtable}
\usepackage{graphicx}
\usepackage[export]{adjustbox}
\makeatletter
\def\maxwidth{\ifdim\Gin@nat@width>\linewidth\linewidth\else\Gin@nat@width\fi}
\def\maxheight{\ifdim\Gin@nat@height>\textheight\textheight\else\Gin@nat@height\fi}
\makeatother
\setkeys{Gin}{width=.8\maxwidth,height=\maxheight,keepaspectratio}
\makeatletter
\def\fps@figure{hbp}
\makeatother

  
\usepackage{luacolor}
\usepackage[soul]{lua-ul}

\setlength{\emergencystretch}{3em} % prevent overfull lines
\providecommand{\tightlist}{%
  \setlength{\itemsep}{0pt}\setlength{\parskip}{0pt}}

  \setcounter{secnumdepth}{-\maxdimen} % remove section numbering
  
  
  
\usepackage{fancyhdr}

\fancypagestyle{plain}{% used for the first page
	\fancyhf{}% clear all header and footer fields
	\fancyhead[L]{}
	\fancyfoot[L]{}
	\fancyfoot[R]{\small\thepage}
	\renewcommand{\headrulewidth}{0pt}%
	\renewcommand{\footrulewidth}{0pt}%
}

\fancypagestyle{page}{%
	\fancyhf{}% clear all header and footer fields
	\fancyhead[L]{}
	\fancyhead[R]{\small Guhr, \textit{Lehreinheit: Textannotation (mit
CATMA) als Blended Learning}}
	\fancyfoot[L]{\small\textit{forTEXT} 2(1): Textannotation in der
Hochschullehre}
	\fancyfoot[R]{\thepage}
	\renewcommand{\headrulewidth}{0pt}%
	\renewcommand{\footrulewidth}{0pt}%
}

\pagestyle{page}


\usepackage[
	small,
	sf,bf,
	raggedright,
	clearempty,
]{titlesec}

\titleformat{\section}{\Large\sffamily\bfseries}{\thesection.}{.5em}{}
\titleformat{\subsection}{\large\sffamily\bfseries}{\thesubsection}{.5em}{}
\titleformat{\subsubsection}{\normalsize\sffamily\bfseries}{\thesubsubsection}{.5em}{}
\titleformat{\paragraph}{\normalsize\sffamily\bfseries}{\thesubsubsection}{.5em}{}
\titleformat{\subparagraph}{\normalsize\sffamily\bfseries}{\thesubsubsection}{.5em}{}

  \NewDocumentCommand\citeproctext{}{}
\NewDocumentCommand\citeproc{mm}{%
  \begingroup\def\citeproctext{#2}\cite{#1}\endgroup}
\makeatletter
 \let\@cite@ofmt\@firstofone
 \def\@biblabel#1{}
 \def\@cite#1#2{{#1\if@tempswa , #2\fi}}
\makeatother
\newlength{\cslhangindent}
\setlength{\cslhangindent}{1.5em}
\newlength{\csllabelwidth}
\setlength{\csllabelwidth}{3em}
\newenvironment{CSLReferences}[2] % #1 hanging-indent, #2 entry-spacing
 {\begin{list}{}{%
  \setlength{\itemindent}{0pt}
  \setlength{\leftmargin}{0pt}
  \setlength{\parsep}{0pt}
  % turn on hanging indent if param 1 is 1
  \ifodd #1
   \setlength{\leftmargin}{\cslhangindent}
   \setlength{\itemindent}{-1\cslhangindent}
  \fi
  % set entry spacing
  \setlength{\itemsep}{#2\baselineskip}}}
 {\end{list}}
\usepackage{calc}
\newcommand{\CSLBlock}[1]{\hfill\break\parbox[t]{\linewidth}{\strut\ignorespaces#1\strut}}
\newcommand{\CSLLeftMargin}[1]{\parbox[t]{\csllabelwidth}{\strut#1\strut}}
\newcommand{\CSLRightInline}[1]{\parbox[t]{\linewidth - \csllabelwidth}{\strut#1\strut}}
\newcommand{\CSLIndent}[1]{\hspace{\cslhangindent}#1}
  
  \usepackage[bidi=basic]{babel}
    \babelprovide[main,import]{ngerman}
                  \let\LanguageShortHands\languageshorthands
\def\languageshorthands#1{}
  
\usepackage[
	ragged,
	bottom,
	norule,
	multiple,
]{footmisc}

\makeatletter
\RenewDocumentCommand \footnotemargin { } {0em}
\RenewDocumentCommand \thefootnote { } {\arabic{footnote}}
\RenewDocumentCommand \@makefntext { m } {\noindent{\@thefnmark}. #1}
\interfootnotelinepenalty=10000
\makeatother

\usepackage{changepage}
% \newlength{\overhang}
% \setlength{\overhang}{\marginparwidth}
% \addtolength{\overhang}{\marginparsep}

  \usepackage{caption}
\captionsetup{labelformat=empty,font={small,it}}
  
\usepackage{selnolig}

  
  
  
  
\usepackage{csquotes}
\usepackage{bookmark}
\IfFileExists{xurl.sty}{\usepackage{xurl}}{} % add URL line breaks if available
\urlstyle{same}

  
  \usepackage[inkscapelatex=false]{svg}

\makeatletter
\def\@maketitle{%
	%
	% title
	%
	\newpage
	\null
	\vspace*{-\topskip}
	\begin{tblr}{
    vline{1,5} = {1pt},
    colspec={lllX},
    width=\textwidth,
    columns={font=\small\sffamily,},
    column{1,3}={
      rightsep=.3em,
      font=\footnotesize\sffamily,
    },
}
\hline[1pt]
\SetCell[c=4]{t,l}{\normalsize\textbf{Lehreinheit: Textannotation (mit
CATMA) als Blended Learning}} & & & \\
& & & \SetCell[r=3]{c,r}{\includegraphics[height=2\baselineskip]{}} \\
\SetCell[c=3]{t,l}{\small Svenja Guhr \orcidlink{0000-0002-7686-3609}
\textsuperscript{\scriptsize 1}
} & & &\\
\SetCell[c=2]{t,l}{% 
\footnotesize%
1. Technische Universität Darmstadt \\
} & & & \\
		\hline[1pt]
        Ausgabe:              & Textannotation in der
Hochschullehre & & \\
    
        Jahrgang:             & 2 & & \\
    
        Ausgabe:              & 1 & & \\
    
    
        Beitragstyp:          & Lehrkonzept & & \\
    
        DOI:                  & \href{https://doi.org/}{10.48694/fortext.4067} & & \\
    
        Begutachtende:        & \href{mailto:marie.flueh@uni-hamburg.de}{Marie
Flüh}, \href{mailto:jan-christoph.meister@uni-hamburg.de}{Jan Christoph
Meister} & & \\
    
		Lizenz:               & \faCreativeCommons\ \faCreativeCommonsBy\ \faCreativeCommonsSa\ & & \oalogo  \\
		\hline[1pt]
	\end{tblr}
}
\makeatother

\hypersetup{
      pdftitle={Lehreinheit: Textannotation (mit CATMA) als Blended
Learning},
      pdfauthor={Svenja Guhr},
      pdflang={de},
      pdfsubject={forTEXT 2(1): Textannotation in der Hochschullehre},
      pdfkeywords={Selbststudieneinheit, Blended
Learning, CATMA, Manuelle Annotation, Literaturwissenschaft},
      colorlinks=true,
  linkcolor={linkcolor},
  filecolor={filecolor},
  citecolor={citecolor},
  urlcolor={urlcolor},
  pdfcreator={LuaLaTeX via pandoc}
}


  
  \usepackage{ragged2e}
\usepackage[section]{placeins}
% Manage float placement
\usepackage{float}
\floatplacement{figure}{H}

\usepackage{marginnote}
\RenewDocumentCommand \marginfont { }
{ \sffamily\footnotesize }
\RenewDocumentCommand \raggedleftmarginnote { } { }
% \author{true}

% \date{}

\usepackage{everypage}
\usepackage[printwatermark]{xwatermark}
\usepackage{lipsum}


\begin{document}



\pagestyle{plain}


\maketitle



% \marginnote{\RaggedRight}[30\baselineskip]%









\setstretch{1.2}




\pagestyle{page}

\renewcommand{\arraystretch}{3}  % Zeilenabstand in Tabellen erhöhen

\section{Abstract}\label{abstract}

Das Lehrkonzept stellt eine vierwöchige Lehreinheit vor, die Studierende
in die Textannotation am Beispiel des Annotations- und Analysetools
CATMA einführt. Vor dem Hintergrund der Erzähltheorie wird in einer
Kombination aus asynchronen Selbstlernphasen und synchronen
Präsenzsitzungen die Methode der Textannotation eingeführt und ihre
praktische Anwendung angeleitet. Zudem lernen Studierende, wie sie
literaturwissenschaftliche Fragestellungen formulieren und
Einsatzmöglichkeiten digitaler Textannotation kritisch reflektieren
können.

\section{Inhalt}\label{inhalt}

\begin{enumerate}
\def\labelenumi{\arabic{enumi}.}
\tightlist
\item
  \hyperref[einfuxfchrung]{Einführung}
\item
  \hyperref[gesamtablauf]{Gesamtablauf}
\item
  \hyperref[tabellarische-sitzungsuxfcbersicht]{Tabellarische
  Sitzungsübersicht}
\item
  \hyperref[elemente-der-lehreinheit]{Elemente der Lehreinheit}
\item
  \hyperref[reflexion]{Reflexion}
\end{enumerate}

\section{1. Einführung}\label{einfuxfchrung}

Das nachfolgend vorgestellte Lehrkonzept führt Studierende in die
Textannotation am Beispiel des digitalen Annotationstools CATMA ein.
Dabei werden asynchrone und synchrone Elemente im Sinne eines
Blended-Learning-Ansatzes kombiniert.

\subsection{1.1. Rahmenbedingungen}\label{rahmenbedingungen}

Die Lehreinheit zum Thema „Textannotation (mit CATMA)`` greift eine
Kombination aus asynchroner und synchroner Lehre auf (Blended
Learning)\footnote{Für mehr Informationen zu Blended Learning-Methoden,
  siehe (\citeproc{ref-graham_framework_2013}{Graham, Woodfield und
  Harrison 2013}).} und deckt einen Zeitraum von vier Semesterwochen ab.
Die Lehreinheit wurde im Rahmen der Einführungsveranstaltung
\emph{Grundkurs Literaturwissenschaft 2} im Sommersemester an der
Technischen Universität Darmstadt entwickelt, wo sie bereits dreimal
durchgeführt wurde (Sommersemester 2022, 2023 und 2024). Der
\emph{Grundkurs 2} deckt den zweiten Teil eines zweisemestrigen Moduls
zur Einführung in die Literaturwissenschaft ab, das im
Germanistikstudium in den ersten zwei Studiensemestern die Grundlagen
der deutschsprachigen Literaturwissenschaft vermitteln soll, um die
Studierenden auf das literaturwissenschaftliche Arbeiten in den darauf
aufbauenden thematischen Pro- und Hauptseminaren der höheren Semester
vorzubereiten. Die erfolgreiche Teilnahme an der Lehrveranstaltung wird
durch eine 90-minütige schriftliche Klausur am Ende der Vorlesungszeit
festgestellt. Nach Bestehen der Abschlussklausur erhalten die
Teilnehmenden fünf ECTS-Punkte
(\citeproc{ref-modulhandbuch_2019}{Institut für Sprach- und
Literaturwissenschaft 2019}).

Die Lehreinheit repräsentiert eine Schnittstelle zwischen
Geisteswissenschaften und Digital Humanities. Sie integriert digitale
Methoden in die literaturwissenschaftliche Lehre, indem sie den
Studierenden die Nutzung von CATMA\footnote{CATMA steht für Computer
  Assisted Text Markup and Analysis, siehe Gius u.~a.
  (\citeproc{ref-gius_catma_2016}{2016}); Gius u.~a.
  (\citeproc{ref-giusCATMA2024}{2024}).} als Annotations- und
Analysetool nahebringt. Diese Interdisziplinarität ermöglicht es,
traditionelle geisteswissenschaftliche Fragestellungen durch den Einsatz
von Technologien zu vertiefen und zu erweitern, und fördert gleichzeitig
die digitalen Kompetenzen der Studierenden.

Blended Learning wurde gewählt, um vorbereitende Schritte wie die
Seminarlektüre, den Überblick über zentrale Funktionen sowie die erste
Erprobung der Annotationssoftware unter Rückgriff auf vorhandenes
Lernmaterial in eine Selbstlernphase auszulagern. Im synchronen Teil
kann so der Fokus auf die Vermittlung komplexer
literaturwissenschaftlicher Sachverhalte, auf Diskussionen zu den
gesammelten Annotationserfahrungen und die Beantwortung
literaturwissenschaftlicher Fragestellungen gelegt werden. Die
Lehreinheit findet zeitlich in der Mitte des Semesters statt, wobei die
Selbstlernphase die Hälfte der Lehreinheit abdeckt. Ihre Vorteile liegen
insbesondere in der Auslagerung des zeitaufwändigen Einrichtens und
Ausprobierens des CATMA-Accounts sowie des Durcharbeitens der
einführenden Tutorials, wobei die Studierenden die Software in der von
ihnen gewählten Geschwindigkeit kennenlernen, ausprobieren und bei
Interesse noch weiter vertiefen können.

Der Kurs wird als synchrone Lehrveranstaltung in Lehrform eines
Grundkurses mit zwei Semesterwochenstunden für eine Gruppe von ca. 20-30
Studierenden (Germanistik Bachelor / Deutsch Lehramt an Gymnasien)
angeboten, die i.d.R. bereits den ersten Teil des Grundkurses besucht
haben und somit erste grundlegende Kompetenzen in der
Literaturwissenschaft mitbringen.

Die vorgesehenen Lerninhalte des \emph{Grundkurses 2} als
Rahmenlehrveranstaltung sind laut Modulbeschreibung die „Einführung in
erweiterte Gebiete der Literaturwissenschaft. Studierende sollen am Ende
des Kurses mit Themen der Erzähltheorie, der Literaturgeschichte und der
Editionswissenschaft sowie mit den entsprechenden Theorien und Konzepten
vertraut sein und diese unter Anleitung kritisch einordnen und
diskutieren können'' (\citeproc{ref-modulhandbuch_2019}{Institut für
Sprach- und Literaturwissenschaft 2019}). Die unter den
Literaturwissenschaftsdozent*innen abgestimmten großen Themen der
Lehrveranstaltung umfassen zwei Sitzungen zu Literaturtheorien, zwei
Sitzungen zur Literaturgeschichte (19. und 20. Jahrhundert), vier
Sitzungen zur Großgattung Prosa, drei Sitzungen zur Großgattung Lyrik
sowie drei Sitzungen zu Organisatorischem, Klausurvorbereitung und
Klausurdurchführung. Eine erfolgreiche Teilnahme am Kurs befähigt die
Studierenden zum Umgang mit Begriffen und Konzepten erweiterter Gebiete
der Literaturwissenschaft. Sie können Analysen mittels wichtiger
Methoden des jeweiligen Teilgebiets durchführen. Darüber hinaus haben
sie ein grundlegendes Verständnis der Literaturwissenschaft und ihrer
Unterdisziplinen erlangt und sind mit den Grundlagen der
literaturwissenschaftlichen Analyse, dem analytischen Lesen und dem
wissenschaftlichen Arbeiten vertraut.

\subsection{1.2. Voraussetzungen der
Teilnehmenden}\label{voraussetzungen-der-teilnehmenden}

Zur Durchführung der Lehreinheit müssen die Studierenden Zugang zu einem
internetverbundenen Laptop haben, einen der gängigen Browser verwenden
können (z.B. Firefox, Chrome oder Safari) sowie grundlegende
Sprachkenntnisse im Englischen mitbringen, um das User Interface der
Textannotationssoftware CATMA verstehen und nutzen zu können.\\
Die benötigten technischen Vorkenntnisse erarbeiten sich die
Studierenden in der Selbstlernphase. Weitere technische Kenntnisse sind
nicht erforderlich.\\
Als fachliche Vorkenntnis wird ein generelles Verständnis im Umgang mit
literarischen Texten und literarischer Erzähltextanalyse vorausgesetzt,
das dem Niveau des Oberstufendeutschunterrichts bzw. darauf aufbauend
ggf. dem Besuch des \emph{Grundkurses Literaturwissenschaft 1} an der
Technischen Universität Darmstadt entspricht.

\subsection{1.3 Ausführung der
Lehreinheit}\label{ausfuxfchrung-der-lehreinheit}

Die Lehreinheit besteht aus vier Elementen:

\begin{enumerate}
\def\labelenumi{\arabic{enumi}.}
\tightlist
\item
  eine vorbereitende synchrone Sitzung zur Einführung in die
  Erzähltextanalyse,
\item
  eine asynchrone Selbstlernphase von zwei Wochen, in der die
  Studierenden die Video-Tutorials und die vorbereitende Lektüre
  (Primär- und Sekundärliteratur) durcharbeiten,
\item
  eine synchrone Sitzung zur Einführung in die
  literaturwissenschaftliche Annotationspraxis mit Zeit für Fragen und
  einer Anwendungsaufgabe zur Verwendung von CATMA für die Annotation
  und anschließende Figurenanalyse des Primärtextes,
\item
  eine Aufgabe zur manuellen Textannotation in der Abschlussklausur zur
  Überprüfung des Erreichens der Lernziele der Lehreinheit.
\end{enumerate}

Für die synchronen Veranstaltungssitzungen benötigt die*der Lehrende
einen internetverbundenen Laptop sowie einen Beamer. Die asynchronen
Anteile wie Lektüre und Links zu den Video-Tutorials der Selbstlernphase
werden von dem*der Lehrenden über die begleitende Medienplattform (hier
Moodle) zur Verfügung gestellt.

Zur Vermittlung von Kompetenzen werden verschiedene Medien eingesetzt:\\
In der asynchronen Selbstlernphase haben die Studierenden Zugang zu
Video-Tutorials. Diese Tutorials bieten eine Einführung in die manuelle
Annotation mit dem digitalen Tool CATMA
(\citeproc{ref-fortext_tutorial_2019}{forTEXT 2019a};
\citeproc{ref-fortext_tutorial_2020}{forTEXT 2020a};
\citeproc{ref-fortext_tutorial_2019-1}{forTEXT 2019b};
\citeproc{ref-fortext_tutorial_2019-2}{forTEXT 2019c};
\citeproc{ref-fortext_tutorial_2020-1}{forTEXT 2020b}). Zudem werden
vorbereitend drei Einführungstexte zum manuellen und kollaborativen
Annotieren (analog und digital mit CATMA) zur Verfügung gestellt
(\citeproc{ref-schumacherToolbeitragCATMA2019}{Schumacher 2024};
\citeproc{ref-jackeMethodenbeitragManuelleAnnotation2018}{Jacke 2024b};
\citeproc{ref-jackeMethodenbeitragKollaborativesLiteraturwissenschaftliches2018}{Jacke
2024c}). Während der synchronen Lehrveranstaltungen wird als kurzer
Primärtext die Erzählung \emph{Krambambuli} von Ebner-Eschenbach
(\citeproc{ref-von_ebner-eschenbach_marie_1896}{1896}) diskutiert und
die Aufgabe zur Verwendung von CATMA mit Fokus auf literarische
Erzähltext- und Figurenanalyse
(\citeproc{ref-Hansen_figuren_2016}{Hansen 2016}) literaturtheoretisch
und methodisch eingebettet.

Zur Unterstützung der Studierenden wurde in der dritten Iteration dieser
Lehreinheit ein*e Tutor*in eingesetzt. Diese*r stand während der
asynchronen Selbstlernphase in einem Moodleforum für Fragen und
Hilfestellungen zur Verfügung. Während der synchronen Sitzungen half
er*sie bei technischen Problemen mit dem Annotationstool CATMA und
unterstützte die Studierenden bei der Durchführung der
Annotationsaufgabe.

In der Abschlussklausur ist eine Aufgabe zur manuellen Textannotation
vorgesehen, um das Erreichen der Lernziele der Lehreinheit zu
überprüfen.

\section{2. Gesamtablauf}\label{gesamtablauf}

Die Lehreinheit wird im Semesterplan als Teil des Themas „Einführung in
die Gattungstheorie: Prosa`` verortet und baut auf den Inhalten der
Sitzungen zu Literaturgeschichte des 19. Jahrhunderts und
Literaturtheorie auf. Sie besteht aus vier Teilen: Der erste Teil
beinhaltet eine theoretische Einführung in die Analyse von Figuren und
in das Formulieren von literaturwissenschaftlichen Fragestellungen in
einer synchronen Lehrveranstaltungssitzung.\\
Der zweite Teil besteht aus einer zweiwöchigen asynchronen
Selbstlernphase. In dieser Phase arbeiten die Studierenden mit
zusammengestellten Materialien zur Einführung in die
literaturwissenschaftliche Textannotation. Diese Materialien umfassen
Artikel zur manuellen und kollaborativen Annotation (analog und digital
mit CATMA) sowie kurze Videotutorials zur Einführung in die manuelle
Annotation mit dem digitalen Annotationstool CATMA. Zusätzlich lesen die
Studierenden den Primärtext \emph{Krambambuli} von Ebner-Eschenbach
(\citeproc{ref-von_ebner-eschenbach_marie_1896}{1896}).\\
Der dritte Teil besteht aus einer synchronen Sitzung zur Einführung in
die literaturwissenschaftliche Annotationspraxis mit Zeit für Fragen und
für die Anwendungsaufgabe zur Verwendung von CATMA für die Annotation
und anschließende Figurenanalyse des Primärtexts. Dieser dritte Teil
kann auf 90 oder 120 Minuten angesetzt werden, je nachdem, wie viele
Teilnehmende erwartet werden, wie fortgeschritten die Studierenden im
Umgang mit digitalen Methoden sind und wie engagiert sie erfahrungsgemäß
an Diskussionen teilnehmen und Fragen stellen.\\
Im vierten und letzten Teil wird das Gelernte durch eine Prüfungsaufgabe
in der Klausur überprüft, die auf die literaturwissenschaftliche
Textannotation fokussiert.

Im Fokus der Lehreinheit steht die Vermittlung der Erzähltheorie und
ihrer praktischen Anwendung durch die literaturwissenschaftliche Methode
der Annotation. Die Studierenden sollen nach dieser Lehreinheit in der
Lage sein, Methoden der Erzähltheorie auf bekannte literarische
Primärtexte des 19. Jahrhunderts anzuwenden und drei
literaturwissenschaftliche Annotationsmodi umzusetzen: manuell-analoges
Annotieren literarischer Texte, manuell-digitales Annotieren
literarischer Texte und kollaborativ manuell-digitales Annotieren
literarischer Texte. Des Weiteren sollen die Studierenden den Nutzen und
die Anwendbarkeit literaturwissenschaftlicher Textannotation
reflektieren und lernen, situationsspezifisch zu entscheiden, wie
verschiedene Annotationsverfahren am sinnvollsten anzuwenden sind, indem
z.B. die Erstellung und der Einsatz von Annotationstagsets im Kontext
einer ausgewählten Forschungsfrage reflektiert wird. Außerdem sollen sie
literaturwissenschaftliche Forschungsfragen formulieren und sich der
Beantwortung dieser Fragen mit Hilfe literaturwissenschaftlicher
Textannotation annähern können.

\begin{landscape}

\section{3. Tabellarische
Sitzungsübersicht}\label{tabellarische-sitzungsuxfcbersicht}

\footnotesize

\begin{longtable}[]{@{}
  >{\raggedright\arraybackslash}p{(\columnwidth - 14\tabcolsep) * \real{0.0800}}
  >{\raggedright\arraybackslash}p{(\columnwidth - 14\tabcolsep) * \real{0.0500}}
  >{\raggedright\arraybackslash}p{(\columnwidth - 14\tabcolsep) * \real{0.1500}}
  >{\raggedright\arraybackslash}p{(\columnwidth - 14\tabcolsep) * \real{0.1500}}
  >{\raggedright\arraybackslash}p{(\columnwidth - 14\tabcolsep) * \real{0.1500}}
  >{\raggedright\arraybackslash}p{(\columnwidth - 14\tabcolsep) * \real{0.1500}}
  >{\raggedright\arraybackslash}p{(\columnwidth - 14\tabcolsep) * \real{0.1500}}
  >{\raggedright\arraybackslash}p{(\columnwidth - 14\tabcolsep) * \real{0.1500}}@{}}
\toprule\noalign{}
\begin{minipage}[b]{\linewidth}\raggedright
\textbf{Element}
\end{minipage} & \begin{minipage}[b]{\linewidth}\raggedright
\textbf{Modus}
\end{minipage} & \begin{minipage}[b]{\linewidth}\raggedright
\textbf{Thema}
\end{minipage} & \begin{minipage}[b]{\linewidth}\raggedright
\textbf{Inhalt}
\end{minipage} & \begin{minipage}[b]{\linewidth}\raggedright
\textbf{Lernziel}
\end{minipage} & \begin{minipage}[b]{\linewidth}\raggedright
\textbf{Vorbereitung der Studierenden}
\end{minipage} & \begin{minipage}[b]{\linewidth}\raggedright
\textbf{Lehrvorbereitung}
\end{minipage} & \begin{minipage}[b]{\linewidth}\raggedright
\textbf{Aufgabe}
\end{minipage} \\
\midrule\noalign{}
\endhead
\bottomrule\noalign{}
\endlastfoot
1 & synchron; Präsenz & theoretische Einführung in die Analyse von
Figuren & Einführung in die Figurenanalyse und das Formulieren von
Forschungsfragen & Grundlagen der Figurenanalyse verstehen und anwenden;
Forschungsfragen formulieren & Lektüre
(\citeproc{ref-Hansen_figuren_2016}{Hansen 2016}) & Laptop; Beamer;
vorbereitende Texte; Foliensatz „Figuren`` (siehe Anhang 1:
FS\_Figurenanalyse) & Formulieren einer Forschungsfrage zur
Figurenanalyse \\
2 & asynchron; online & Selbststudium: Einführung in die
literaturwissenschaftliche Textannotation & Erarbeitung von
Einführungstexten und Video-Tutorials zur manuellen und digitalen
Annotation & Grundlagen des manuellen und digitalen Annotierens
verstehen; Anwendung des Tools CATMA & Lektüre
(\citeproc{ref-schumacherToolbeitragCATMA2019}{Schumacher 2024};
\citeproc{ref-jackeMethodenbeitragManuelleAnnotation2018}{Jacke 2024b};
\citeproc{ref-jackeMethodenbeitragKollaborativesLiteraturwissenschaftliches2018}{Jacke
2024c}); Anschauen der Tutorials
(\citeproc{ref-fortext_tutorial_2019}{forTEXT 2019a};
\citeproc{ref-fortext_tutorial_2020}{forTEXT 2020a};
\citeproc{ref-fortext_tutorial_2019-1}{forTEXT 2019b};
\citeproc{ref-fortext_tutorial_2019-2}{forTEXT 2019c};
\citeproc{ref-fortext_tutorial_2020-1}{forTEXT 2020b}) & sicherstellen,
dass Links funktionieren und Materialien über Moodle verfügbar sind &
Durcharbeiten der Materialien; Vorbereitung auf die synchrone Sitzung
(inklusive der Einrichtung eines persönlichen CATMA-Accounts) \\
3 & synchron; Präsenz & Einführung in die literaturwissenschaftliche
Annotationspraxis & zusammenfassende Einführung in die
literaturwissenschaftliche Annotationspraxis und praktische Anwendung
der Annotationsmethoden mit CATMA & Anwendung der Annotationsmethoden;
Reflexion der Nützlichkeit der Annotation & & Vorbereitung des
CATMA-Projekts mit Tagset zur Figurenanalyse (orientiert an
\citeproc{ref-jackeRessourcenbeitragTagsetNarratologie2020}{\textbf{jackeRessourcenbeitragTagsetNarratologie2020?}})
\& Annotationsbeispielen in vorbereiteter Annotationscollection;
Formulieren von Übungsaufgaben zur Klausurvorbereitung; Laptop; Beamer;
Internetzugang; Zugriff auf CATMA & Übung von Klausuraufgaben zu
literaturwissenschaftlicher Textannotation \\
4 & synchron; Präsenz & Überprüfung des Gelernten durch Klausuraufgabe &
Durchführung der Anwendungsaufgabe in der Klausur & Sicherstellung der
Lernzielerreichung & Diskussion und Übung von Prüfungsfragen zur
Textannotation als Teil der vorhergehenden Sitzung zur
Prüfungsvorbereitung & Bereitstellung der Prüfungsaufgabe und eines
Beispielprimärtextausschnitts & \\
\end{longtable}

\end{landscape}
\normalsize

\section{4. Elemente der Lehreinheit}\label{elemente-der-lehreinheit}

\subsection{Element 1: Theoretische Einführung in die Analyse von
Figuren}\label{element-1-theoretische-einfuxfchrung-in-die-analyse-von-figuren}

\emph{(Synchrone Sitzung, 90min)}\\
In der ersten Sitzung erhalten die Studierenden eine theoretische
Einführung in die Analyse von Figuren (aufbauend auf
\citeproc{ref-Hansen_figuren_2016}{Hansen 2016}) sowie in das
Formulieren von literaturwissenschaftlichen Fragestellungen. Diese
synchrone Sitzung beginnt mit einem Vortrag der Lehrperson (siehe Anhang
1: FS\_Figurenanalyse''), in dem die grundlegenden Konzepte und Methoden
der Figurenanalyse behandelt werden. Zunächst wird die Bedeutung und
Funktion von Figuren in literarischen Texten erläutert. Danach werden
die drei funktionalen Dimensionen nach Phelan
(\citeproc{ref-Phelan_2005}{2005}), die Unterscheidung von flachen und
runden Charakteren nach Forster (\citeproc{ref-Forster_1949}{1949}) und
das Figurenmodell nach Hansen (\citeproc{ref-Hansen_2000}{2000}) mit
seiner Unterscheidung von „showing`` und „telling`` präsentiert. Der
Inputvortrag wird abgeschlossen mit einer Diskussion, in der das Gehörte
durch die Studierenden angewendet werden soll, indem sie diskutieren,
welche Figuren im vorbereiteten Primärtext \emph{Krambambuli} von
Ebner-Eschenbach (\citeproc{ref-von_ebner-eschenbach_marie_1896}{1896})
vorkommen und wie diese hinsichtlich der vorgestellten Figurenmodelle
eingeordnet werden können.

Im Anschluss wird das Erarbeiten und Formulieren von
literaturwissenschaftlichen Forschungsfragen thematisiert. Die
Studierenden lernen, wie sie präzise und relevante Fragen zur
Figurenanalyse entwickeln können. Die Sitzung umfasst Diskussionen und
Gruppenarbeiten, in denen die Studierenden auf der Grundlage eines
kurzen Textauszugs aus der Primärlektüre Forschungsfragen formulieren,
die mit Hilfe der Figurenanalyse bearbeitet werden können.

Ziel dieser Sitzung ist es, ein Verständnis der grundlegenden Konzepte
der Figurenanalyse zu vermitteln, die Fähigkeit zur Unterscheidung und
Anwendung verschiedener Ansätze der Figurenanalyse zu entwickeln und die
Kompetenz zu fördern, literaturwissenschaftliche Forschungsfragen zu
formulieren. Im Rahmen des \emph{Grundkurses 2} an der TU Darmstadt
wurde dieses Element als Teil der umfassenderen
Lehrveranstaltungssitzung zur „Einführung in die Gattungstheorie - Fokus
Prosa`` durchgeführt. Dieser Sitzungsteil deckt ca. 30min einer 90
minütigen Sitzung ab.

\subsection{Element 2: Einführung in die literaturwissenschaftliche
Textannotation
(Selbstlernphase)}\label{element-2-einfuxfchrung-in-die-literaturwissenschaftliche-textannotation-selbstlernphase}

\emph{(Asynchrone Selbstlernphase über zwei Wochen, ca. 5
Arbeitsstunden)}\\
Der zweite Teil der Lehreinheit besteht aus einer asynchronen
Selbstlernphase, welche sich über zwei Wochen erstreckt. In dieser Zeit
haben die Studierenden die Gelegenheit, sich eigenständig in die
Grundlagen der literaturwissenschaftlichen Textannotation und die
Annotations- und Analysesoftware CATMA einzuarbeiten. Sie setzen sich
mit drei Einführungstexten auseinander, die das manuelle und
kollaborative Annotieren sowohl in analoger als auch digitaler Form
behandeln (\citeproc{ref-schumacherToolbeitragCATMA2019}{Schumacher
2024}; \citeproc{ref-jackeMethodenbeitragManuelleAnnotation2018}{Jacke
2024b};
\citeproc{ref-jackeMethodenbeitragKollaborativesLiteraturwissenschaftliches2018}{Jacke
2024c}). Ergänzend dazu stehen Video-Tutorials zur Einführung in die
manuelle Annotation mit dem digitalen Annotationstool CATMA zur
Verfügung (\citeproc{ref-fortext_tutorial_2019}{forTEXT 2019a};
\citeproc{ref-fortext_tutorial_2020}{forTEXT 2020a};
\citeproc{ref-fortext_tutorial_2019-1}{forTEXT 2019b};
\citeproc{ref-fortext_tutorial_2019-2}{forTEXT 2019c};
\citeproc{ref-fortext_tutorial_2020-1}{forTEXT 2020b}). Diese
Materialien bieten den Studierenden eine umfassende Einführung in die
Annotationsverfahren und ermöglichen ihnen, das Gelernte praktisch
auszuprobieren.

Ziel dieser Selbstlernphase ist es, die Grundlagen des manuellen und
digitalen Annotierens zu vermitteln, die Fähigkeit zur Nutzung des Tools
CATMA zu entwickeln und ein Verständnis für die Vor- und Nachteile
verschiedener Annotationsverfahren zu schaffen. Die Studierenden lesen
die Einführungstexte, schauen sich die Video-Tutorials an und führen
erste praktische Übungen zur Annotation mit CATMA durch. Ab der
CATMA-Version 7.2 Classroom Edition ist es auch möglich, den
Teilnehmenden Zugang zu einem gemeinsamen CATMA-Gruppenprojekt zu
gewähren und als Ergebnissicherung den erfolgreichen Abschluss der
Selbstlernphase durch obligatorische Annotationen im Gruppenprojekt zu
überprüfen.

\subsection{Element 3: Einführung in die literaturwissenschaftliche
Annotationspraxis}\label{element-3-einfuxfchrung-in-die-literaturwissenschaftliche-annotationspraxis}

\emph{(Synchrone Sitzung, 90min)}\\
In der dritten Sitzung steht die praktische Anwendung der
Annotationsverfahren im Fokus. Die Studierenden haben vorbereitend einen
CATMA-Account eingerichtet, erste Anwendungsbeispiele geübt und setzen
die während der asynchronen Selbstlernphase erworbenen Kenntnisse in
einer synchronen Sitzung in die Praxis um. Begleitet wird diese für
90min konzipierte Sitzung vom Foliensatz zur Einführung in die
Textannotation (siehe Anhang 2: FS\_Einführung\_Textannotation''). Um
den Studierenden mehr Zeit für die Diskussionen und Anwendungsaufgaben
zu geben, empfiehlt sich eine Sitzung von ca. 120min. Ziel dieser
Sitzung ist es, die Anwendung der Annotationsverfahren in der Praxis zu
vertiefen und die Fähigkeit zur kritischen Reflexion von Textannotation
zu fördern.

Die Sitzung beginnt mit einem Brainstorming (ca. 5min) in Partnerarbeit
zur Aktivierung des Gelernten aus der zweiwöchigen Selbstlernphase,
wobei das thematische Vokabular der Sitzung im geschützten Raum der
Partnerarbeit vorbereitet wird. Anschließend gibt es einen kurzen
Austausch im Plenum im Sinn einer reduzierten Form der
„Think-Pair-Share`` Lernmethode von Lyman
(\citeproc{ref-lyman_responsive_1998}{1998}).

Es folgt ein Inputvortrag (ca. 10min) der Lehrperson zur generellen
Einführung in das Thema und die Methodik des Annotierens
(Bildannotationen, Markup Languages, Glossen) sowie spezifisch zur
Annotation als textwissenschaftliche Praxis in der
Literaturwissenschaft. Zunächst werden drei verschiedene
Annotationsarten in der Sprach- und Literaturwissenschaft vorgestellt:
Freitextkommentare, taxonomiebasierte Annotation und technische
Annotation im Rahmen des Textauszeichnungsverfahrens (nach
\citeproc{ref-jackeMethodenbeitragManuelleAnnotation2018}{Jacke 2024b}).
Anschließend werden sechs Modi bzw. Verfahren der Textannotation anhand
von Beispielen gegenübergestellt: manuell vs.~maschinell, analog
vs.~digital, digital halbautomatisiert vs.~digital automatisiert (nach
\citeproc{ref-rapp_manuelle_2017}{Rapp 2017}).

Am Ende des Inputvortrags werden die Studierenden gebeten, in
Einzelarbeit manuell einen Textausschnitt zu annotieren (ca. 5min). Dazu
wurde vor Sitzungsbeginn der Anfang des ausgedruckten Primärtexts
ausgeteilt. Das Ziel dieser Aufgabe ist es, verschiedene
Annotationsideen auszuprobieren, die während des Impulsvortrags
aufgekommen sind. Der Erfahrung nach variieren die von den Studierenden
vorgenommen Annotationen stark (z.B. Unterstreichung, Umkreisung und
Umkastung von Wortarten und Wortfamilien). Nach ca. 5min haben die
Studierenden kurz die Möglichkeit, dem Plenum vorzustellen, was sie wie
annotiert haben. Damit wird zur Diskussionsfrage übergeleitet: „Braucht
man Vorgaben zum Annotieren?``. Der Erfahrung nach sprechen die
Studierenden in der Diskussion ihre anfängliche Unsicherheit mit der
Annotationsaufgabe an, weil sie nicht wussten, was genau sie annotieren
sollten. Der nächste Punkt der Diskussion wird eingeleitet durch die
Reflexionsfragen: „Welche Vorgaben hätten Sie sich gewünscht?`` und
„Welche Vorgaben haben Sie sich selbst gegeben?``. Durch diese offenen
Fragen werden sich die Studierenden der Notwendigkeit von
Annotationsrichtlinien bewusst. Sie diskutieren ihre selbstdefinierten
Vorgaben, z.B. wann unterstrichen und wann umkreist wurde und welche
Farbe welche Annotationseinheit bedeutete. Die Lehrperson leitet sodann
über zum zweiten Impulsvortrag.

Im zweiten Inputvortrag (ca. 10min) stellt die Lehrperson den Nutzen und
Aufbau von Annotationsrichtlinien vor. Unterstützt durch die Abbildung
von Gius und Jacke (\citeproc{ref-gius_jacke_2016}{2016, 7}) wird die
Komplexität des Annotationsprozesses verdeutlicht. Es folgt eine kurze
Wiederholung zu Annotationskategorien und Tagsets, die Veranschaulichung
der Textannotation als iterativen und zyklischen Prozess (nach
\citeproc{ref-rapp_manuelle_2017}{Rapp 2017};
\citeproc{ref-zinsmeister_korpuslinguistik_2015}{Zinsmeister und
Lemnitzer 2015}) sowie ein sehr kurzer Ausblick auf die Evaluation des
Annotationsprozesses und Goldstandardannotationen (nach
\citeproc{ref-jackeMethodenbeitragKollaborativesLiteraturwissenschaftliches2018}{Jacke
2024c}). Nach einer kurzen Unterbrechung für Fragen der Studierenden
folgt der Anwendungsteil „Annotation mit CATMA``.

Für den Anwendungsteil schlägt die Lehrperson eine Forschungsfrage, eine
Hypothese und eine Annotationsmodus vor. Ein mögliches
Frage-Hypothesen-Paar zur Figurenanalyse ist: „Welche Hinweise auf der
Textoberfläche kennzeichnen eine Figur als Protagonist*in?`` - „Anhand
der Häufigkeit der Figurenreferenz ist erkennbar, welche Figur
Protagonist*in des Textes ist.`` Außerdem bringt sie ein
CATMA-Gruppenprojekt mit vorbereitetem Tagset mit (z.B. das Tagset
Narratologie (histoire) von
\citeproc{ref-jacke_ressourcenbeitrag_2024}{Jacke 2024a}), das die
Studierenden anwenden können. Nach einer kurzen Vorstellung der
Materialien treten die Teilnehmenden dem CATMA-Gruppenprojekt bei und
erhalten den folgenden Arbeitsauftrag (ca. 20min):

\begin{quote}
Bitte annotieren Sie \textbf{manuell digital} den Anfang aus Marie von
Ebener-Eschenbachs \emph{Krambambuli}.\\
a) Erstellen Sie eine \textbf{Annotation Collection} nach dem Schema:
„Name\_Annotation\_Collection``.\\
b) Öffnen Sie den Text im \textbf{Annotationsmodus} („Annotate``).\\
c) Wählen Sie Ihre Annotation Collection aus und \textbf{annotieren} Sie
den Anfang von \emph{Krambambuli} unter Anwendung des Tagsets zur
Figurenannotation.\\
d) \textbf{Synchronisieren} Sie Ihre Annotationen auf der
Projektstartseite („SYNC`` → synchronize with the team).
\end{quote}

Während der Arbeitsphase steht die Lehrperson für Fragen und technische
Unterstützung zur Verfügung. Die Lehrperson wird dabei ggf.durch eine*n
Tutor*in vor Ort unterstützt. Die Arbeitsphase wird mit einer Reflexion
abgeschlossen, in der die Studierenden berichten, wie sie mit den
Annotationskategorien und den technischen Prozessen zurechtkamen. Zudem
sollen sie aufführen, welche Tags ihnen gefehlt haben oder welche sie
nicht verwendet haben. Schließlich sollen sie auf der Grundlage ihrer
Annotationserfahrung argumentieren, wie die Forschungsfrage beantwortet
werden könnte.

Anschließend werden exemplarisch Abfragen, Visualisierungen und
halbautomatisierte Annotationen mit CATMA durchgeführt, um eine mögliche
Annäherung an die Forschungsfrage mittels quantitativer
Annotationsauswertung (der vorbereiteten Annotationen) vorzustellen
(z.B. Keyword in Context Abfragen (KWIC) und Visualisierung der
Häufigkeit und Verteilung von Annotationen pro Figurenreferenz über den
Verlauf des annotierten Textes).

Den Abschluss der Sitzung bildet eine Diskussion im Plenum, in der die
Studierenden aufgefordert werden, die Unterschiede, Gemeinsamkeiten,
Grenzen und Mehrwerte von digitaler und nicht-digitaler Annotation sowie
die Kombination beider Annotationsformen praxisnah herauszuarbeiten und
zu diskutieren. Die Diskussion endet mit der perspektivischen Frage,
welche weiteren Forschungsfragen zum Primärtext und darüber hinaus die
Studierenden interessieren würden, die sie im Rahmen eines
CATMA-Annotationsprojekts angehen könnten.

\subsection{Element 4: Überprüfung des Gelernten durch
Klausuraufgabe}\label{element-4-uxfcberpruxfcfung-des-gelernten-durch-klausuraufgabe}

\emph{(Schriftliche Aufgabe, ca. 15 bzw. ca. 25min)}\\
In der letzten Woche der Vorlesungszeit wird das in den vorangegangenen
Sitzungen Gelernte anhand einer Aufgabe in der 90-minütigen
Abschlussklausur angewendet. Die Klausur besteht in der Regel aus sechs
verschiedenen Aufgaben, von denen eine die Textannotation betrifft. Im
Rahmen der Klausur des \emph{Grundkurses 2} wurden im Sommersemester
2022, 2023 und 2024 Variationen der folgenden zwei Aufgaben gestellt:

\subsubsection{Aufgabe 1:}\label{aufgabe-1}

Eine Klausuraufgabe mit Umfang von 5,5P. (ca. 15min Aufwand der 90min
Gesamtklausur (30P.)):

\begin{enumerate}
\def\labelenumi{\alph{enumi})}
\tightlist
\item
  Erklären Sie kurz die Methode der manuellen Textannotation (1P.).
\item
  Entwickeln Sie ein Tagset bestehend aus mindestens drei verschiedenen
  Tags für die Annotation des Primärtextausschnitts. Nennen Sie dafür
  den Tagnamen und erläutern Sie (kurz) Ihr Annotationsvorgehen (3,5P.:
  jeweils 0,5P. pro Nennung des Namens (Tagset und 3 Tags) und jeweils
  0,5P. pro Kurzerläuterung des Annotationsvorgehens pro Tag, z.B.
  Unterstreichen, Tagfarbe, gewählte Kategorie).
\item
  Annotieren Sie den Textausschnitt systematisch mit Ihren
  neudefinierten Tags (1P.).
\end{enumerate}

Die erste Klausuraufgabe sollte im Klausuraufbau einen mittleren Platz
einnehmen, weil sie die ersten drei Taxonomiestufen kognitiver Lernziele
nach Anderson und Krathwohl
(\citeproc{ref-anderson_taxonomy_2001}{2001}) abdeckt. Da höhere
Taxonomiestufen tendenziell später in der Klausur angeordnet werden
sollten, eignet sich diese Aufgabe nicht als Einstiegsaufgabe:\\
Einerseits wird bereits das reine Nennen von Tagsetnamen und Tagnamen
mit Punkten vergütet (Taxonomiestufe 1), andererseits aber auch die
Erklärung der Methode gefordert sowie die Erläuterung des
Annotationsvorgehens (Taxonomiestufe 2) und die Anwendung der
Annotationsmethode auf den gegebenen Primärtextausschnitt
(Taxonomiestufe 3).

\subsubsection{Aufgabe 2:}\label{aufgabe-2}

Eine Klausuraufgabe mit Umfang von 7P. (ca. 20-25min Aufwand):

\begin{quote}
Lesen Sie den Primärtextausschnitt.\\
a) Formulieren Sie eine Forschungsfrage hinsichtlich der im Kurs
behandelten Inhalte (z.B. aus dem Bereich zu Prosatheorie, Erzähltheorie
-- etwa Diskurs, Geschichte, Figuren -- oder Literaturgeschichte).
Beachten Sie bei der Formulierung die im Kurs besprochenen inhaltlichen
und formalen Anforderungen an eine Forschungsfrage (1 Satz = 1P.).\\
b) Begründen Sie die (literaturwissenschaftliche) Relevanz Ihrer
Forschungsfrage kurz, indem Sie sie in den literaturwissenschaftlichen
Forschungskontext einordnen (1-2 Sätze, 2P.).\\
c) Entscheiden Sie, ob und wie literaturwissenschaftliche Textannotation
als Grundlage der Annäherung an die Beantwortung Ihrer Forschungsfrage
angewendet werden kann. Begründen Sie Ihre Entscheidung und beschreiben
Sie Ihre mögliche Herangehensweise (3-5 Sätze, 4P.).
\end{quote}

Die zweite Klausuraufgabe sollte im Klausuraufbau den Platz einer der
letzten oder der letzten Aufgabe einnehmen. Sie deckt die höchsten
Taxonomiestufen kognitiver Lernziele nach Anderson und Krathwohl
(\citeproc{ref-anderson_taxonomy_2001}{2001}) ab, indem eine
Forschungsfrage formuliert (Produktion, Taxonomiestufe 6), ihre Relevanz
begründet (Beurteilung, Taxonomiestufe 5) sowie für oder gegen die
Anwendung von Textannotation als Methode entschieden werden muss
(Entscheidung, Taxonomiestufe 5).

Mit der erfolgreichen Bearbeitung der Klausuraufgabe weisen die
Studierenden nach, dass sie das Lernziel erreichen, theoretische
Konzepte in konkreten Analysekontexten anzuwenden, indem sie
literaturwissenschaftlich relevante Forschungsfragen formulieren und
Textannotation als Methode zur Annäherung an die selbstformulierte
Forschungsfrage verwenden sowie die Anwendung reflektieren.

\section{5. Reflexion}\label{reflexion}

\subsection{5.1. Erfahrungen bei der Durchführung der
Lehrveranstaltung}\label{erfahrungen-bei-der-durchfuxfchrung-der-lehrveranstaltung}

Ich habe das vorgestellte Lehrkonzept in meiner Rolle als Dozentin
bisher drei Mal durchgeführt. Dabei konnte ich retrospektiv feststellen,
dass die Aufteilung der Lehreinheit zur „Literarische{[}n{]}
Textannotation (mit CATMA)`` von den Studierenden gut angenommen wurde.
Vor allem das Angebot einer asynchronen Selbststudienphase anstelle
einer synchronen Sitzung wurde positiv aufgenommen. Im Feedback der
Studierenden wurde deutlich, dass sie die Mischung aus synchronen und
asynchronen Einheiten sowie die praxisorientierte Anwendung der Theorie
schätzten. Besonders die Video-Tutorials fanden die Studierenden
hilfreich, weil sie sich dadurch die Inhalte in ihrem eigenen Tempo
erarbeiten und wahlweise Teile des Tutorials wiederholen konnten. Die
Teilnehmenden meldeten außerdem positiv zurück, dass sie das Gelernte
aus der Selbstlernphase und den Theorieinputvorträgen in der synchronen
Sitzung (3) selbst anwenden konnten und zum Weiterdenken angeregt
wurden.

Die synchrone Sitzung (3) zur Einführung in die
literaturwissenschaftliche Annotationspraxis, das Herzstück des
vorgestellten Lehrkonzepts, verlief in allen drei Iterationen ohne
Probleme. Die technischen Voraussetzungen waren gegeben: Der Beamer und
das WLAN funktionierten, die Studierenden brachten internetverbundene
Laptops mit sowie eingerichtete CATMA-Accounts, sodass sie die
Anwendungsaufgaben durchführen konnten.

Ich konnte den Lernerfolg aus der Selbstlernphase zum manuell-digitalen
Annotationsverfahren daran festmachen, dass die Studierenden, die an der
synchronen Sitzung teilnahmen, sie als Vorbereitung durchgeführt haben
mussten, um während der Sitzung CATMA anwenden und gut mitarbeiten zu
können. Während der Bearbeitungszeit der Anwendungsaufgaben konnte ich
im Dialog mit den Teilnehmenden feststellen, ob die grundsätzliche
Anwendung von CATMA verstanden wurde und aktives Feedback zu den
Arbeitsschritten geben sowie bei Schwierigkeiten helfen und individuelle
Fragen beantworten. Darüber hinaus konnten erste Annotationsversuche als
Teil der Selbstlernphase integriert werden. Diese Erweiterung und
Überprüfung der Aufgabenstellung wird in Zukunft durch die CATMA Version
7.2 Classroom Edition erleichtert, indem die Teilnehmenden auf ein
gemeinsames CATMA Gruppenprojekt zugreifen können. Im Sommersemester
2024 war dies noch nicht möglich.

Ab der dritten Iteration war es für Studierende eine wertvolle
Unterstützung beim Umgang mit CATMA und der individuellen Vorbereitung,
eine*n Tutor*in einzusetzen. Insbesondere bei technischen Fragen und
individuellen Problemen konnte der*die Tutor*in direkt vor Ort oder
bereits vorher online im Moodleforum Hilfestellungen geben. Sollte bei
der Nachnutzung dieses Lehrkonzepts in externen Veranstaltungskontexten
keine Unterstützung durch eine*n Tutor*in möglich sein, könnte ein von
den Studierenden selbstmoderiertes Onlineforum (z.B. bei Moodle)
eingesetzt werden, in dem Fragen während der asynchronen Selbstlernphase
individuell gestellt und beantwortet werden können (auch unterstützt
durch die Lehrperson). Außerhalb des Grundkurskontextes könnte zudem ein
kürzerer Primärtext zur Annotation zur Verfügung gestellt werden, damit
die Teilnehmenden einen vollständigen Text annotieren können und nicht
nur einen kurzen Ausschnitt. Dafür könnte zum Beispiel ein Märchen als
Textgrundlage zum Einsatz kommen. Ein kürzerer Text würde es den
Teilnehmenden ermöglichen, den gesamten Text zu erfassen und somit ein
besseres Verständnis für die Struktur und die narrativen Techniken zu
entwickeln.

Bei der Bewertung der Klausuraufgabe fiel auf, dass auch Studierende,
die nicht an der synchronen Sitzung (3) teilgenommen hatten, sich durch
die Nachbereitung der Materialien ausreichend für eine erfolgreiche
Bearbeitung der Klausuraufgabe vorbereiten konnten.

Insgesamt bin ich zufrieden mit dem Aufbau und allen drei Durchführungen
der Lehreinheit. Meine Bedenken, dass die geplanten Zeitabschnitte für
Arbeitsaufgaben und Plenumsdiskussionen zu eng getaktet sein könnten,
haben sich nicht bestätigt. Dafür war es jedoch wichtig, die Zeitplanung
durchgängig im Blick zu haben und die Diskussionen so zu moderieren,
dass der vorgesehene Zeitrahmen eingehalten wurde.

Ich kann mir vorstellen, dass eine Verlängerung der synchronen Sitzung
(3) um 30min sinnvoll sein könnte, um den Studierenden mehr Zeit für
Annotation, Austausch und Diskussion zu geben. Alternativ könnte man
eine Pause vor dem zweiten Inputvortrag einplanen, damit die
Studierenden für den zweiten Teil der synchronen Sitzung mehr Energie
und Aufmerksamkeit haben sowie die gelernten Inhalte besser verarbeiten
können. Je nach Gruppengröße sowie bei erwartbar mehr technischen Hürden
bei den Teilnehmenden wäre auch eine Zweiteilung der synchronen Sitzung
auf zwei Sitzungen denkbar.

Eine weiterhin offene Herausforderung besteht, wenn Studierende die
Selbstlernphase nicht absolvieren, sei es aufgrund einer Abneigung
gegenüber digitalen Methoden oder weil sie keinen CATMA-Account
einrichten möchten. Zwar ist das Verständnis von Annotationsverfahren
und -taxonomien nicht zwingend an die Nutzung von CATMA gebunden und hat
somit keine direkten Auswirkungen auf die Klausur, doch kann die
fehlende Vorbereitung das Arbeitsklima in der synchronen Sitzung
beeinträchtigen. Ein weiteres Hindernis ergab sich in einem Fall, in dem
ein*e Studierende*r keinen eigenen Laptop besaß und die Selbstlernphase
am Familiencomputer absolvieren musste. Um die Teilnahme an der
synchronen Sitzung zu ermöglichen, stellte ich vor Ort einen Laptop zur
Verfügung.

\subsection{5.2. Studierende}\label{studierende}

Mit Blick auf die tatsächlich teilnehmenden Studierenden habe ich in den
drei Iterationen der Lehreinheit (Sommersemester 2022, 2023 und 2024)
ganz unterschiedliche Erfahrungen gesammelt. Vor allem die Anzahl der
Teilnehmenden an der synchronen Sitzung (3) variierte: In den ersten
beiden Iterationen waren es jeweils 15 Teilnehmende, die sich aktiv in
den synchronen Sitzungen engagierten und in der Abschlussklausuraufgabe
zur Textannotation gute bis sehr gute Ergebnisse erzielten.\\
Bei der dritten Iteration waren nur vier Studierende in der zweiten
synchronen Sitzung anwesend, die sich aber sehr engagiert beteiligten.
Ich verpasste leider die Gelegenheit zu eruieren, ob es am geringen
Interesse für Textannotation oder der nicht ausreichend deutlichen
Ankündigung von Textannotation als Klausurthema lag, dass so wenige
Studierende in der synchronen Sitzung anwesend waren, oder andere Gründe
für das Fernbleiben vorlagen. Die Ergebnisse in der
Abschlussklausuraufgabe fielen jedoch überwiegend gut bis sehr gut aus,
was darauf hinweist, dass die zur Verfügung gestellten Materialien
ausreichten, um die Lernziele zu erreichen.\\
Alle Teilnehmenden waren wie geplant eingeschriebene Studierende im
Bachelor Germanistik oder Deutsch Lehramt an Gymnasien und brachten die
verlangten Vorkenntnisse und technischen Voraussetzungen mit, wobei bei
jeder Iteration immer mindestens ein*e Studierende*r dabei war, die*der
anstelle eines Laptops ein Tablet als internetfähiges Endgerät in die
Veranstaltung mitbrachte, auf dem zwar die Annotationen einsehbar waren
aber nicht selbst vorgenommen werden konnten. Diese Studierenden führten
die Anwendungsaufgabe in der synchronen Sitzung in Partnerarbeit durch.

\phantomsection\label{refs}
\begin{CSLReferences}{1}{0}
\bibitem[\citeproctext]{ref-anderson_taxonomy_2001}
Anderson, Lorin W. und David R. Krathwohl, Hrsg. 2001. \emph{A taxonomy
for learning, teaching, and assessing: a revision of {Bloom}'s taxonomy
of educational objectives}. Complete ed. New York: Longman.

\bibitem[\citeproctext]{ref-von_ebner-eschenbach_marie_1896}
Ebner-Eschenbach, Marie von. 1896. \emph{Marie von {Ebner}-{Eschenbach}:
{Krambambuli}}.
\url{https://www.projekt-gutenberg.org/ebnresch/krambamb/krambamb.html}
(zugegriffen: 14. Oktober 2024).

\bibitem[\citeproctext]{ref-Forster_1949}
Forster, Edward Morgan. 1949. \emph{Ansichten des Romans}. Suhrkamp.

\bibitem[\citeproctext]{ref-fortext_tutorial_2019}
forTEXT. 2019a. Tutorial: {CATMA} 6 zur manuellen {Annotation} nutzen.
Oktober. doi:
\href{https://doi.org/10.5281/zenodo.10353556}{10.5281/zenodo.10353556},
\url{https://zenodo.org/records/10353556}.

\bibitem[\citeproctext]{ref-fortext_tutorial_2019-1}
---------. 2019b. Tutorial: {Projektmanagement} in {CATMA} 6. November.
doi:
\href{https://doi.org/10.5281/zenodo.10353713}{10.5281/zenodo.10353713},
\url{https://zenodo.org/records/10353713}.

\bibitem[\citeproctext]{ref-fortext_tutorial_2019-2}
---------. 2019c. Tutorial: {Tagsets} in {CATMA} 6 anlegen. Dezember.
doi:
\href{https://doi.org/10.5281/zenodo.10377968}{10.5281/zenodo.10377968},
\url{https://zenodo.org/records/10377968}.

\bibitem[\citeproctext]{ref-fortext_tutorial_2020}
---------. 2020a. Tutorial: {In} {CATMA} 6 annotieren. Januar. doi:
\href{https://doi.org/10.5281/zenodo.10353910}{10.5281/zenodo.10353910},
\url{https://doi.org/10.5281/zenodo.10353910}.

\bibitem[\citeproctext]{ref-fortext_tutorial_2020-1}
---------. 2020b. Tutorial: {Analysieren} und visualisieren mit {CATMA}.
Februar. doi:
\href{https://doi.org/10.5281/zenodo.10276637}{10.5281/zenodo.10276637},
\url{https://doi.org/10.5281/zenodo.10276637}.

\bibitem[\citeproctext]{ref-gius_jacke_2016}
Gius, Evelyn und Janina Jacke. 2016. Zur {Annotation} narratologischer
{Kategorien} der {Zeit}. {Guidelines} zur {Nutzung} des
{CATMA}-{Tagsets}.
\url{http://heureclea.de/wp-content/uploads/2016/11/guidelinesV2.pdf}.

\bibitem[\citeproctext]{ref-gius_catma_2016}
Gius, Evelyn, Janina Jacke, Jan Christoph Meister und Marco Petris.
2016. {CATMA} ({Computer} {Aided} {Textual} {Markup} and {Analysis}) --
ein {Tool} für die hermeneutische {Textanalyse}. In:
\emph{{CLARIN}-{D}}. Hamburg.
\url{https://www.clarin-d.net/de/konferenz-abstracts/380-catma-computer-aided-textual-markup-and-analysis-ein-tool-fuer-die-hermeneutische-textanalyse}.

\bibitem[\citeproctext]{ref-giusCATMA2024}
Gius, Evelyn, Jan Christoph Meister, Malte Meister, Marco Petris,
Dominik Gerstorfer und Mari Akazawa. 2024. {CATMA}. Zenodo, Juni. doi:
\href{https://doi.org/10.5281/zenodo.12092195}{10.5281/zenodo.12092195},.

\bibitem[\citeproctext]{ref-graham_framework_2013}
Graham, Charles R., Wendy Woodfield und J. Buckley Harrison. 2013. A
framework for institutional adoption and implementation of blended
learning in higher education. \emph{The Internet and Higher Education}
18: 4--14. doi:
\href{https://doi.org/10.1016/j.iheduc.2012.09.003}{10.1016/j.iheduc.2012.09.003},
\url{https://linkinghub.elsevier.com/retrieve/pii/S1096751612000607}
(zugegriffen: 14. Oktober 2024).

\bibitem[\citeproctext]{ref-Hansen_2000}
Hansen, Per Krogh. 2000. \emph{Karakterens rolle. Aspekter af en
litterær karakterologi {[}Die Rolle des Characters. Aspekte einer
literarischen Charakterologie{]}}. Medusa.

\bibitem[\citeproctext]{ref-Hansen_figuren_2016}
---------. 2016. {IV}. 3.3 {Figuren}. In: \emph{Einführung in die
{Erzähltextanalyse}}, hg. von Silke Lahn und Jan Christoph Meister,
übers. von Marie Isabel Schlinzig, 234--249. 3. Aufl. Lehrbuch.
Stuttgart: J. B. Metzler Verlag.
\url{https://doi.org/10.1007/978-3-476-05415-9} (zugegriffen: 27. Januar
2021).

\bibitem[\citeproctext]{ref-modulhandbuch_2019}
Institut für Sprach- und Literaturwissenschaft. 2019. Modulhandbuch
{Germanistik} {J}.{B}.{A}. Technische Universität Darmstadt.
\url{https://www.tu-darmstadt.de/media/daa_responsives_design/02_studium_medien/01_studieninteressierte_medien/02_studienangebot_medien/joint_bachelor_of_arts_2/germanistik_1/modulhandbuch_7/MHB-JBA-Germanistik-2019.pdf}
(zugegriffen: 14. Oktober 2024).

\bibitem[\citeproctext]{ref-jacke_ressourcenbeitrag_2024}
Jacke, Janina. 2024a. Ressourcenbeitrag: {Tagset} {Narratologie}
(histoire) 1, Nr. 4. Manuelle Annotation. doi:
\href{https://doi.org/10.48694/fortext.3757}{10.48694/fortext.3757},
\url{https://www.fortext-hefte.de/article/id/3757/}.

\bibitem[\citeproctext]{ref-jackeMethodenbeitragKollaborativesLiteraturwissenschaftliches2018}
---------. 2024c. Methodenbeitrag: {Kollaboratives}
literaturwissenschaftliches {Annotieren}. \emph{forTEXT} 1, Nr. 4.
Manuelle Annotation (August). doi:
\href{https://doi.org/10.48694/fortext.3749}{10.48694/fortext.3749},
\url{https://www.fortext-hefte.de/article/id/3749/}.

\bibitem[\citeproctext]{ref-jackeMethodenbeitragManuelleAnnotation2018}
---------. 2024b. Methodenbeitrag: {Manuelle} {Annotation}.
\emph{forTEXT} 1, Nr. 4. Manuelle Annotation (August). doi:
\href{https://doi.org/10.48694/fortext.3748}{10.48694/fortext.3748},
\url{https://www.fortext-hefte.de/article/id/3748/}.

\bibitem[\citeproctext]{ref-lyman_responsive_1998}
Lyman, F. T. 1998. The responsive classroom discussion: {The} inclusion
of all students. In: \emph{Mainstreaming {Digest}}, hg. von A. S.
Anderson, 109--113. College Park: University of Maryland Press.

\bibitem[\citeproctext]{ref-Phelan_2005}
Phelan, James. 2005. \emph{Living to Tell About It. A Rhetoric and
Ethics of Character Narration}. Cornell University Press.

\bibitem[\citeproctext]{ref-rapp_manuelle_2017}
Rapp, Andrea. 2017. Manuelle und automatische {Annotation}. In:
\emph{Digital {Humanities}. {Eine} {Einführung}}, hg. von Fotis
Jannidis, Hubertus Kohle, und Malte Rehbein, 253--267.

\bibitem[\citeproctext]{ref-schumacherToolbeitragCATMA2019}
Schumacher, Mareike. 2024. Toolbeitrag: {CATMA}. \emph{forTEXT} 1, Nr.
4. Manuelle Annotation (August). doi:
\href{https://doi.org/10.48694/fortext.3761}{10.48694/fortext.3761},
\url{https://www.fortext-hefte.de/article/id/3761/}.

\bibitem[\citeproctext]{ref-zinsmeister_korpuslinguistik_2015}
Zinsmeister, Heike und Lothar Lemnitzer. 2015. \emph{Korpuslinguistik.
{Eine} {Einführung}}. Narr.

\end{CSLReferences}




\end{document}
