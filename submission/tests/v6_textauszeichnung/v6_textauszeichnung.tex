% Options for packages loaded elsewhere
\PassOptionsToPackage{unicode}{hyperref}
\PassOptionsToPackage{hyphens}{url}
\PassOptionsToPackage{dvipsnames,svgnames,x11names}{xcolor}
%
\documentclass[
          a4paper,
        ]{article}
\raggedbottom
\usepackage{amsmath,amssymb}
\usepackage{setspace}
\usepackage{iftex}
\RequireLuaTeX

\usepackage{lscape} % für Querformat-Seiten
\usepackage{hyperref}

\usepackage{unicode-math} % this also loads fontspec
\defaultfontfeatures{Scale=MatchLowercase}
\defaultfontfeatures[\rmfamily]{Ligatures=TeX,Scale=1}

  \usepackage[]{plex-otf}
  
  
  
  
  
  
  
\directlua{luaotfload.add_fallback
   ("fallbacks",
    {
      "STIX Two Math:;",
      "NotoColorEmoji:mode=harf;",
    }
   )}
\defaultfontfeatures{RawFeature={fallback=fallbacks}}

\usepackage{academicons}
\usepackage{fontawesome5}
\usepackage{ccicons}

  
\IfFileExists{upquote.sty}{\usepackage{upquote}}{}
\IfFileExists{microtype.sty}{% use microtype if available
  \usepackage[]{microtype}
  \UseMicrotypeSet[protrusion]{basicmath} % disable protrusion for tt fonts
}{}
\usepackage{lua-widow-control}
  \makeatletter
\@ifundefined{KOMAClassName}{% if non-KOMA class
  \IfFileExists{parskip.sty}{%
    \usepackage{parskip}
  }{% else
    \setlength{\parindent}{0pt}
    \setlength{\parskip}{6pt plus 2pt minus 1pt}}
}{% if KOMA class
  \KOMAoptions{parskip=half}}
\makeatother
  
  
\usepackage[dvipsnames,svgnames,x11names]{xcolor}

\definecolor{oa-orange}     {RGB} {246, 130, 18}
\definecolor{ft-red}        {RGB} {168, 0, 0}
\definecolor{orcid-green}   {RGB} {166, 206, 57}
\definecolor{fortext-green}   {RGB} {26, 73, 76}


\colorlet{oa-orange}        {black}

\colorlet{highlightcolor1}  {DarkOrange}
\colorlet{highlightcolor2}  {DarkBlue}
\colorlet{highlightcolor3}  {DarkGreen}
\colorlet{highlightcolor4}  {DarkMagenta}

\colorlet{filecolor}        {highlightcolor1}
\colorlet{linkcolor}        {fortext-green}
\colorlet{citecolor}        {fortext-green}
\colorlet{urlcolor}         {fortext-green}


\NewDocumentCommand \oalogo { }
{%
  \textcolor{oa-orange}{\sffamily%\bfseries%
  open \aiOpenAccess\ access}
}


\NewDocumentCommand \orcidlink { m }
{%
	\texorpdfstring
	{\href{https://orcid.org/#1}{\textcolor{orcid-green}{\raisebox{-.2ex}{\aiOrcid}}}}
	{https://orcid.org/#1}%
}

\NewDocumentCommand \emaillink { m }
{%
	\texorpdfstring
	{\href{mailto:#1}{\textcolor{black}{\raisebox{-.2ex}{\faEnvelopeOpen[regular]}}}}
	{mailto:#1}%
}

  \usepackage{geometry}
\geometry{
	paper=a4paper,
	top=25mm,
	bottom=20mm,
	right=30mm,
	left=30mm,
	footskip=10mm,
	% showframe,
}
  
  \usepackage{listingsutf8}
\lstset{
  language=python,
  basicstyle=\ttfamily\footnotesize,
  columns=fullflexible,
  xleftmargin=2em,
  xrightmargin=2em,
  % frame=single,
  breaklines=true,
  postbreak=\mbox{\textcolor{highlightcolor1}{\(\hookrightarrow\)}\space},
}
\newcommand{\passthrough}[1]{#1}
\lstset{defaultdialect=[5.3]Lua}
\lstset{defaultdialect=[x86masm]Assembler}
  
  
  
\usepackage{longtable,booktabs,array}
\usepackage{tabularray}
\usepackage{ifthen}

  
\usepackage{calc} % for calculating minipage widths
% Correct order of tables after \paragraph or \subparagraph
\usepackage{etoolbox}
\makeatletter
\patchcmd\longtable{\par}{\if@noskipsec\mbox{}\fi\par}{}{}
\makeatother
% Allow footnotes in longtable head/foot
\IfFileExists{footnotehyper.sty}{\usepackage{footnotehyper}}{\usepackage{footnote}}
\makesavenoteenv{longtable}
\usepackage{graphicx}
\usepackage[export]{adjustbox}
\makeatletter
\def\maxwidth{\ifdim\Gin@nat@width>\linewidth\linewidth\else\Gin@nat@width\fi}
\def\maxheight{\ifdim\Gin@nat@height>\textheight\textheight\else\Gin@nat@height\fi}
\makeatother
\setkeys{Gin}{width=.8\maxwidth,height=\maxheight,keepaspectratio}
\makeatletter
\def\fps@figure{hbp}
\makeatother

  
\usepackage{luacolor}
\usepackage[soul]{lua-ul}

\setlength{\emergencystretch}{3em} % prevent overfull lines
\providecommand{\tightlist}{%
  \setlength{\itemsep}{0pt}\setlength{\parskip}{0pt}}

  \setcounter{secnumdepth}{-\maxdimen} % remove section numbering
  
  
  
\usepackage{fancyhdr}

\fancypagestyle{plain}{% used for the first page
	\fancyhf{}% clear all header and footer fields
	\fancyhead[L]{}
	\fancyfoot[L]{}
	\fancyfoot[R]{\small\thepage}
	\renewcommand{\headrulewidth}{0pt}%
	\renewcommand{\footrulewidth}{0pt}%
}

\fancypagestyle{page}{%
	\fancyhf{}% clear all header and footer fields
	\fancyhead[L]{}
	\fancyhead[R]{\small Brookshire, Hegel, \textit{Textauszeichnung:
Digitale Edition und Kommentierung}}
	\fancyfoot[L]{\small\textit{forTEXT} 2(13): Textannotation in der
Hochschullehre}
	\fancyfoot[R]{\thepage}
	\renewcommand{\headrulewidth}{0pt}%
	\renewcommand{\footrulewidth}{0pt}%
}

\pagestyle{page}


\usepackage[
	small,
	sf,bf,
	raggedright,
	clearempty,
]{titlesec}

\titleformat{\section}{\Large\sffamily\bfseries}{\thesection.}{.5em}{}
\titleformat{\subsection}{\large\sffamily\bfseries}{\thesubsection}{.5em}{}
\titleformat{\subsubsection}{\normalsize\sffamily\bfseries}{\thesubsubsection}{.5em}{}
\titleformat{\paragraph}{\normalsize\sffamily\bfseries}{\thesubsubsection}{.5em}{}
\titleformat{\subparagraph}{\normalsize\sffamily\bfseries}{\thesubsubsection}{.5em}{}

  \NewDocumentCommand\citeproctext{}{}
\NewDocumentCommand\citeproc{mm}{%
  \begingroup\def\citeproctext{#2}\cite{#1}\endgroup}
\makeatletter
 \let\@cite@ofmt\@firstofone
 \def\@biblabel#1{}
 \def\@cite#1#2{{#1\if@tempswa , #2\fi}}
\makeatother
\newlength{\cslhangindent}
\setlength{\cslhangindent}{1.5em}
\newlength{\csllabelwidth}
\setlength{\csllabelwidth}{3em}
\newenvironment{CSLReferences}[2] % #1 hanging-indent, #2 entry-spacing
 {\begin{list}{}{%
  \setlength{\itemindent}{0pt}
  \setlength{\leftmargin}{0pt}
  \setlength{\parsep}{0pt}
  % turn on hanging indent if param 1 is 1
  \ifodd #1
   \setlength{\leftmargin}{\cslhangindent}
   \setlength{\itemindent}{-1\cslhangindent}
  \fi
  % set entry spacing
  \setlength{\itemsep}{#2\baselineskip}}}
 {\end{list}}
\usepackage{calc}
\newcommand{\CSLBlock}[1]{\hfill\break\parbox[t]{\linewidth}{\strut\ignorespaces#1\strut}}
\newcommand{\CSLLeftMargin}[1]{\parbox[t]{\csllabelwidth}{\strut#1\strut}}
\newcommand{\CSLRightInline}[1]{\parbox[t]{\linewidth - \csllabelwidth}{\strut#1\strut}}
\newcommand{\CSLIndent}[1]{\hspace{\cslhangindent}#1}
  
  \usepackage[bidi=basic]{babel}
    \babelprovide[main,import]{ngerman}
                  \let\LanguageShortHands\languageshorthands
\def\languageshorthands#1{}
  
\usepackage[
	ragged,
	bottom,
	norule,
	multiple,
]{footmisc}

\makeatletter
\RenewDocumentCommand \footnotemargin { } {0em}
\RenewDocumentCommand \thefootnote { } {\arabic{footnote}}
\RenewDocumentCommand \@makefntext { m } {\noindent{\@thefnmark}. #1}
\interfootnotelinepenalty=10000
\makeatother

\usepackage{changepage}
% \newlength{\overhang}
% \setlength{\overhang}{\marginparwidth}
% \addtolength{\overhang}{\marginparsep}

  \usepackage{caption}
\captionsetup{labelformat=empty,font={small,it}}
\input{old}
  
\usepackage{selnolig}

  
  
  
  
\usepackage{csquotes}
\usepackage{bookmark}
\IfFileExists{xurl.sty}{\usepackage{xurl}}{} % add URL line breaks if available
\urlstyle{same}

  
  \usepackage[inkscapelatex=false]{svg}

\makeatletter
\def\@maketitle{%
	%
	% title
	%
	\newpage
	\null
	\vspace*{-\topskip}
	\begin{tblr}{
    vline{1,5} = {1pt},
    colspec={lllX},
    width=\textwidth,
    columns={font=\small\sffamily,},
    column{1,3}={
      rightsep=.3em,
      font=\footnotesize\sffamily,
    },
}
\hline[1pt]
\SetCell[c=4]{t,l}{\normalsize\textbf{Textauszeichnung: Digitale Edition
und Kommentierung}} & & & \\
& & & \SetCell[r=3]{c,r}{\includegraphics[height=2\baselineskip]{}} \\
\SetCell[c=3]{t,l}{\small Patrick
D. Brookshire \orcidlink{https://orcid.org/0000-0002-5843-7577}
\textsuperscript{\scriptsize 1}
} & & &\\
\SetCell[c=3]{t,l}{\small Philipp Hegel \orcidlink{https://orcid.org/0000-0001-6867-1511}
\textsuperscript{\scriptsize 1}
} & & &\\
\SetCell[c=2]{t,l}{% 
\footnotesize%
1. Akademie der Wissenschaften und der Literatur Mainz \\
} & & & \\
		\hline[1pt]
     Titel der Ausgabe: & Textannotation in der Hochschullehre & & \\ 
     Jahrgang: & 2 & & \\ 
     Ausgabe: & 13 & & \\ 
    
    
    
    
    Lizenz: & \faCreativeCommons\ \faCreativeCommonsBy\ \faCreativeCommonsSa\ & & \oalogo \\

		\hline[1pt]
	\end{tblr}
}
\makeatother

\hypersetup{
      pdftitle={Textauszeichnung: Digitale Edition und Kommentierung},
      pdfauthor={Patrick D. Brookshire, Philipp Hegel},
      pdflang={de},
      pdfsubject={forTEXT 2(13): Textannotation in der Hochschullehre},
      pdfkeywords={Digitale Editorik, Text Encoding
Initiative, X-Technologien, Briefedition, Masterstudium},
      colorlinks=true,
  linkcolor={linkcolor},
  filecolor={filecolor},
  citecolor={citecolor},
  urlcolor={urlcolor},
  pdfcreator={LuaLaTeX via pandoc}
}


  
  \usepackage{ragged2e}
\usepackage[section]{placeins}
% Manage float placement
\usepackage{float}
\floatplacement{figure}{H}

\usepackage{marginnote}
\RenewDocumentCommand \marginfont { }
{ \sffamily\footnotesize }
\RenewDocumentCommand \raggedleftmarginnote { } { }
% \author{true \and true}

% \date{}

\usepackage{everypage}
\usepackage[printwatermark]{xwatermark}
\usepackage{lipsum}


\begin{document}



\pagestyle{plain}


\maketitle



% \marginnote{\RaggedRight}[30\baselineskip]%









\setstretch{1.2}




\pagestyle{page}

\renewcommand{\arraystretch}{3}  % Zeilenabstand in Tabellen erhöhen

\section{Abstract}\label{abstract}

Das vorgestellte Lehrkonzept vermittelt Kompetenzen der digitalen
Editorik im Rahmen der Hochschullehre. Der technische Schwerpunkt der
vorgestellten Lehrveranstaltung liegt dabei auf der Vermittlung von
Auszeichnungssprachen. Ziel ist es, den Teilnehmenden zu ermöglichen,
ihren Lernprozess in der praktischen Erprobung editorischer Aufgaben
mittels digitaler Annotationen teilweise selbst zu regulieren. Dafür
werden Schriftstücke aus dem Liebesbriefarchiv transkribiert und
kommentiert.

\section{Inhalt}\label{inhalt}

\begin{enumerate}
\def\labelenumi{\arabic{enumi}.}
\item
  \hyperref[einfuxfchrung]{Einführung}
\item
  \hyperref[gesamtablauf]{Gesamtablauf}
\item
  \hyperref[sitzungsuxfcbersicht]{Sitzungsübersicht}
\item
  \hyperref[reflexion]{Reflexion}
\end{enumerate}

\section{1. Einführung}\label{einfuxfchrung}

Ziel der folgenden Darstellung ist nicht, den Inhalt der Veranstaltung
vollständig darzubieten und eine unmittelbare Nachnutzung zu
ermöglichen, da dies bei 14 Sitzungen den gebotenen Umfang übersteigen
würde. Zudem liegen zu wesentlichen Inhalten der Veranstaltung, nämlich
zu XML, zum „Dialekt`` der Text Encoding Initiative (TEI) und auch zu
den im Bereich der digitalen Editorik besonders relevanten TEI-Modulen
bereits ausreichend Lehrmaterialien vor
(\citeproc{ref-Schoech-2014}{\textbf{Schoech-2014?}}). Die Darstellung
reflektiert vielmehr die Ansprüche an das Lehrkonzept und bereits
vorgenommene Anpassungen. Darüber hinaus zeigt ein Vergleich mit einer
ähnlichen Lehrveranstaltung weitere Anpassungsmöglichkeiten an eigene
Angebote auf.

Diese Lehrinhalte können dem Bereich der digitalen Textannotation
zugeordnet werden (siehe \citeproc{ref-Rapp-2017}{Rapp 2017, 255}). In
der digitalen Editorik haben sich seit den 1980er Jahren die Vorschläge
der TEI verbreitet. Sie werden auch in Richtlinien und Vorgaben von
Förderinstitutionen (siehe zum Beispiel \citeproc{ref-DFG-2015}{Deutsche
Forschungsgemeinschaft 2015}) nahegelegt oder gefordert. Die Vorschläge
der TEI beziehen sich auf die Auszeichnungssprache XML. Derartige
Sprachen können nicht nur Informationen zu Formatierungen vom
eigentlichen Text trennen, sondern auch Strukturen wie Abschnitte und
Absätze markieren (siehe \citeproc{ref-DeRose-1990}{DeRose u.~a. 1990};
\citeproc{ref-Goldfarb-1981}{Goldfarb 1981};
\citeproc{ref-Goldfarb-2004}{Goldfarb und Prescod 2004, 7--10}). Die in
XML gegebene Erweiterbarkeit erlaubt zudem Anpassungen, um mit
Annotationen regelgeleitet und maschinenlesbar unter anderem
linguistisch, literaturwissenschaftlich und editionsphilologisch
relevante Textmerkmale und Schreibphänomene auszuzeichnen. Die
Annotationsschemata sind zum einen ein Modell des interessierenden
Gegenstandes; zum anderen bilden sie die Grundlage für Suchen und
Datenanalysen, in welche die annotierten Dokumente einbezogen werden
können.

\subsection{1.1 Rahmenbedingungen}\label{rahmenbedingungen}

Dieser Beitrag stellt ein Konzept für die Vermittlung digitaler Editorik
im Rahmen der Hochschullehre vor, das von einem der Autoren, Philipp
Hegel, in der Übung \emph{Textauszeichnung} im Masterstudiengang
\emph{Literary and Linguistic Computing} an der Technischen Universität
Darmstadt angewendet wird und auf Erfahrungen aus Projektberatungen
basiert. Das Konzept wurde über mehr als zehn Jahre erprobt und
angepasst. Behandelt werden digitalisierte Privatbriefe der 1950er Jahre
aus dem \href{https://liebesbriefarchiv.de/}{Liebesbriefarchiv}, die in
einem Projekt des Dozenten untersucht werden. Dabei liegen Schwerpunkte
auf der Transkription und dem Kommentieren der Briefe. Ziel des
Lehrkonzepts ist die Vermittlung theoretischer und praktischer
Grundkenntnisse des digitalen Editierens und entsprechender
Auszeichnungssprachen.

An der Lehrveranstaltung nehmen in der Regel fünf bis fünfzehn
Studierende teil und der Arbeitsaufwand entspricht fünf
Leistungspunkten. Zur Prüfungsleistung zählen die aktive Teilnahme in
Form von Beteiligungen an praktischen Übungen, zwei Präsentationen, die
als Zwischenberichte dienen, sowie Projektarbeiten in Form von zehn
transkribierten und kommentierten Briefen im XML-Format. Alle Sitzungen
finden in der Regel synchron in Präsenz statt. Die Zahl der Sitzungen
kann variieren, wobei im Folgenden von 14 Semesterwochen ausgegangen
wird und jede Sitzung 90 Minuten umfasst.

\subsection{1.2 Voraussetzungen der
Teilnehmenden}\label{voraussetzungen-der-teilnehmenden}

Es werden digitale Kompetenzen vorausgesetzt, weshalb sich die
Lehrveranstaltung dezidiert an Masterstudierende richtet.
Kursteilnehmende verfügen über grundlegende Kenntnisse und Erfahrungen
im Programmieren, haben mit X-Technologien, welche im Lehrkonzept eine
zentrale Rolle spielen, jedoch zuvor meist wenig Kontakt gehabt. Zudem
wird vorausgesetzt, dass Teilnehmende eigenständig geeignete
Lernstrategien anwenden können und mit dem wissenschaftlichen Arbeiten
vertraut sind.

Technisch wird außerdem eine Klassenraumlizenz für den
\href{https://www.oxygenxml.com/}{Oxygen XML Editor} vorausgesetzt sowie
eine dazu passende Hardwareausstattung der Studierenden. Die in der
Lehrveranstaltung gemachten technischen Voraussetzungen haben sich
bisher nicht als problematisch erwiesen. Es wird versucht, Studierende
dazu zu motivieren, etwaige Hindernisse anzusprechen. Sollten
Schwierigkeiten benannt oder offenkundig werden, werden Lösungen seitens
des Dozenten in Kooperation mit dem Institut und der Hochschule gesucht.
Studierende werden im Rahmen des Kurses auf Unterstützungsangebote wie
die Studienberatung und Mentoringprogramme am Institut sowie auf
Beratungsstellen an der Universität hingewiesen.

\subsection{1.3 Ausführungen der
Lehrveranstaltung}\label{ausfuxfchrungen-der-lehrveranstaltung}

\subsubsection{1.3.1 Lehrmaterial}\label{lehrmaterial}

Zur Durchführung der Lehrveranstaltung werden verschiedene Materialien
bzw. Medien genutzt und benötigt: Präsentationsfolien, der Oxygen XML
Editor sowie eine WLAN-Verbindung. Für besondere Fragen werden
Bildschirmaufnahmen erstellt. Auf der Lernplattform werden diese ebenso
wie die zu lesenden Texte hinterlegt. Die Lehrmaterialien wurden vom
Dozenten selbst erstellt und angepasst.

\subsubsection{1.3.2 Aufgabentypen}\label{aufgabentypen}

Es gibt wöchentliche Aufgaben (s. Spalte \emph{Abgabe/Aufgabe} in der
tabellarischen Sitzungsübersicht). In den Sitzungen selbst werden
Ratespiele und Gruppenarbeiten durchgeführt. Zudem sind zwei
Zwischenberichte vorgesehen, die als eine Art „Hackathon`` veranstaltet
werden, bei denen Studierende Kodierungsvorschläge diskutieren,
Schwierigkeiten bei der Transkription lösen und kommentierungsbedürftige
Stellen klären (vgl. zum Hackathon als „temporäres Labor``
\citeproc{ref-Trilcke-2018}{Trilcke und Fischer 2018}; und zur Umsetzung
in der akademischen Lehre \citeproc{ref-Mischke-2022}{Mischke, Trilcke
und Sluyter-Gäthje 2022}).

\subsubsection{1.3.3 Lehrstrategischer
Ansatz}\label{lehrstrategischer-ansatz}

Die Struktur der Übung hängt eng mit dem gewählten lehrstrategischen
Ansatz zusammen, der auf Selbstregulation setzt (siehe hierzu
zusammenfassend \citeproc{ref-Landmann-2015}{Landmann u.~a. 2015, 53}).
Der Planung, Beobachtung, Reflexion und Anpassung von Motivationen,
Verfahren und Zielsetzungen im Lernprozess durch die Studierenden selbst
sind jedoch institutionelle und organisatorische Grenzen gesetzt. Zum
einen sind curriculare Lernziele vorgegeben, zum anderen müssen am Ende
Noten für vergleichbare Leistungen vergeben werden (vgl. zu diesen
Schwierigkeiten \citeproc{ref-Boekaerts-2000}{Boekaerts und Niemivirta
2000, 418--419, 432}; \citeproc{ref-Goetz-2017}{Götz und Nett 2017,
148}). In selbstreflexiven Phasen können jedoch diese beiden Aspekte als
Kriterien der sogenannten „Meisterschaft`` und des normativen Vergleichs
auch der studentischen Selbstbewertung dienen (vgl.
\citeproc{ref-Zimmermann-2000}{Zimmermann 2000, 21--22}).

Der Ansatz kann als verteiltes Lernen anhand konkreter Beispiele und
Vertiefungsfragen mit Ansätzen zu Verschachtelungen und Lerntests
beschrieben werden. Verteilt ist das Lernen, weil diskontinuierlich
Themen wiederholt werden (vgl. \citeproc{ref-Moerth-2021}{Moerth,
Paridon und Sonntag 2021, 43}). Insbesondere dienen die Zwischenberichte
dazu, Gelerntes auf das Quellenmaterial anzuwenden und die Ergebnisse
sowie Problemfelder gemeinsam zu diskutieren. Zur Motivation und
Metakognition (vgl. \citeproc{ref-Goetz-2017}{Götz und Nett 2017, 152})
werden Präsentationen als Selbsttest eingesetzt. Der Motivation und
Elaboration dient die Illustration durch verschiedene konkrete
Beispiele, darunter vor allem die den Studierenden zugeordneten Briefe.
Fragen nach Methode, Zweck und Begründung der eigenen Untersuchungen
unterstützen die Elaboration (vgl. \citeproc{ref-Moerth-2021}{Moerth,
Paridon und Sonntag 2021, 45--56}). Verschachtelt ist das Lernen
insoweit, als innerhalb von Unterrichtseinheiten Themenwechsel vollzogen
werden (vgl. \citeproc{ref-Moerth-2021}{Moerth, Paridon und Sonntag
2021, 44}).

Präaktional legt der Dozent Lehrinhalte dar, Bewertungskriterien offen
und erkundet mit Umfragen die Vorkenntnisse bzw. macht diese bewusst.
Aktional erfolgt die Betreuung mit Einladungen zur Sprechstunde und
möglichst vielen Fragemöglichkeiten in den Sitzungen. Die Rückmeldungen
dienen auch postaktional der Betreuung (vgl. zu den Phasen
\citeproc{ref-Zimmermann-2000}{Zimmermann 2000, 16}; sowie
zusammenfassend \citeproc{ref-Otto-2015}{Otto, Perels und Schmitz 2015,
45--46}).

\section{2. Gesamtablauf}\label{gesamtablauf}

Inhaltlich lässt sich die Lehrveranstaltung in sechs Blöcke gliedern. Im
ersten Block (Sitzungen 1 bis 3) wird ein gemeinsamer Wissensstand zur
Editorik, zum Brief, zu Briefeditionen und zu Auszeichnungssprachen
erarbeitet. Auf diese Weise werden etwaige Kompetenzunterschiede eruiert
und möglichst ausgeglichen, da diese Themen grundlegend für den weiteren
Verlauf der Lehrveranstaltung sind.

Daran schließt sich ein Block an, in dem die Studierenden mit
verschiedenen Modulen der TEI-Richtlinien anhand von Beispielen vertraut
gemacht werden (Sitzungen 4 bis 6). Nach dem Abgabetermin für die ersten
Transkriptionen folgen mehrere Sitzungen, in denen die Studierenden ihre
bisherigen Ergebnisse und Probleme vorstellen. Je nach Kursgröße kann
die Zahl dieser Sitzungen variieren. Im exemplarischen Plan wurden drei
Sitzungen vorgesehen (Sitzungen 7 bis 9).

Es folgt eine Sitzung zur Kommentierung (Sitzung 10). Nach dem
Abgabetermin für die ersten kommentierten Transkriptionen folgen wieder
entsprechend viele Sitzungen für Zwischenberichte (im Plan Sitzungen 11
bis 13). Abgeschlossen wird die Veranstaltung mit einem Ausblick auf die
Fertigstellung der digitalen Edition und weiterführende X-Technologien
(Sitzung 14). In längeren Semestern kann dieser Ausblick vertieft
werden.

Editorische und gattungstheoretische Konzepte (Sitzungen 1 bis 3) werden
in den Sitzungen zur digitalen Annotation (Sitzungen 4 bis 6 sowie 10)
aufgegriffen und in der praktischen Anwendung auf konkrete Beispiele von
den Lernenden selbständig geübt (Sitzungen 7 bis 9 sowie 11 bis 13). Die
Anwendung wird mehrfach in der Gruppe diskutiert und Lernende müssen
ihre Entscheidungen und Vorgehen wissenschaftlich begründen. Die
Anwendung des Gelernten und die Präsentation der Ergebnisse haben die
Funktion von Lerntests. Dabei soll verknüpftes Wissen abgerufen werden,
während die Lernfortschritte beobachtet und Rückmeldungen gegeben werden
können.

\begin{landscape}

\section{4. Tabellarische
Sitzungsübersicht}\label{tabellarische-sitzungsuxfcbersicht}

\footnotesize

\begin{longtable}[]{@{}
  >{\raggedright\arraybackslash}p{(\columnwidth - 14\tabcolsep) * \real{0.0800}}
  >{\raggedright\arraybackslash}p{(\columnwidth - 14\tabcolsep) * \real{0.0500}}
  >{\raggedright\arraybackslash}p{(\columnwidth - 14\tabcolsep) * \real{0.1500}}
  >{\raggedright\arraybackslash}p{(\columnwidth - 14\tabcolsep) * \real{0.1500}}
  >{\raggedright\arraybackslash}p{(\columnwidth - 14\tabcolsep) * \real{0.1500}}
  >{\raggedright\arraybackslash}p{(\columnwidth - 14\tabcolsep) * \real{0.1500}}
  >{\raggedright\arraybackslash}p{(\columnwidth - 14\tabcolsep) * \real{0.1500}}
  >{\raggedright\arraybackslash}p{(\columnwidth - 14\tabcolsep) * \real{0.1500}}@{}}
\toprule\noalign{}
\begin{minipage}[b]{\linewidth}\raggedright
Sitzung
\end{minipage} & \begin{minipage}[b]{\linewidth}\raggedright
Modus
\end{minipage} & \begin{minipage}[b]{\linewidth}\raggedright
Thema
\end{minipage} & \begin{minipage}[b]{\linewidth}\raggedright
Inhalt
\end{minipage} & \begin{minipage}[b]{\linewidth}\raggedright
Lernziele
\end{minipage} & \begin{minipage}[b]{\linewidth}\raggedright
Vorbereitung
\end{minipage} & \begin{minipage}[b]{\linewidth}\raggedright
Für Lehrende
\end{minipage} & \begin{minipage}[b]{\linewidth}\raggedright
Abgabe/Aufgabe
\end{minipage} \\
\midrule\noalign{}
\endhead
\bottomrule\noalign{}
\endlastfoot
1 & synchron, Präsenz & Organisation der Veranstaltung und Grundlagen
des Edierens & Vorstellungsrunde. Semesterplan mit
Leistungsanforderungen, Quiz zu editionswissenschaftlichen
Grundbegriffen & Selbsteinschätzung bezüglich Gegenstand und
Leitungsanforderungen. Grundlegendes Verständnis editorischer Aufgaben &
- & Auswahl der Quellenmaterialien. Prüfung der Leistungsanforderungen &
Vorstellung der eigenen Person und der eigenen Lerninteressen.
Beteiligung an Quiz und Diskussion. Vergleich von digitalen Editionen \\
2 & synchron, Präsenz & Grundlagen der Briefforschung &
Wissenschaftliche Definitionen und Explikationen des Begriffs
\emph{Brief} & Reflexion der medialen und generischen Besonderheiten des
Briefes und verwandter Korrespondenzformen (Postkarte, E-Mail) & Ggf.
Anfrage nach einem Lizenzschlüssel für Oxygen XML Editor &
Wissensvermittlung über foliengestützten Vortrag & Gruppendiskussion zu
Spezifika der Gattung \\
3 & synchron, Präsenz & Grundlagen der Briefedition & Zusammenführung
von Wissen über die Gattung und über editiorische Aufgaben anhand von
drei Forschungspositionen. & Kritische Rekonstruktion, Diskussion und
Vergleich verschiedener Forschungspositionen. Ableitung von Konsequenzen
für die digitale Edition von Briefen & Verpflichtung zur Einhaltung der
datenschutzrechtlichen Anforderungen; Zusendung von in Sitzung 2
erarbeiteten Definitionen zum Brief; Lektüre der gruppenweise
zugeordneten Texten zu Briefeditionen & Hilfestellungen.
Diskussionsleitung & Präsentation der Argumentation des gelesenen
Textes; Beteiligung an Gruppenarbeit \\
4 & synchron, Präsenz & Grundlagen der Textauszeichnung gemäß TEI & TEI
\passthrough{\lstinline!core!}, \passthrough{\lstinline!textstucture!}
(Auswahl) & Erstellung eines ersten eigenen Dokuments & Transkription
eines Briefes als Textdatei & Wissensvermittlung über foliengestützten
Vortrag; Bereitstellung von Materialien/Beispielen & Beteiligung an
Diskussion. Bearbeitung des eigenen Beispiels \\
5 & synchron, Präsenz & Quellenbeschreibungen & TEI
\passthrough{\lstinline!header!},
\passthrough{\lstinline!msdescription!} (Auswahl) & Beispielhafte
Ergänzung von Auszeichnungen im Rückgriff auf die Spezifika der Gattung
und der editorischen Aufgabe & - & Wissensvermittlung über
foliengestützten Vortrag & Beteiligung an Diskussion. Bearbeitung des
eigenen Beispiels \\
6 & synchron, Präsenz & Transkription & TEI
\passthrough{\lstinline!transcr!}~(Auswahl) & Beispielhafte Ergänzung
von Auszeichnungen im Rückgriff auf die Spezifika der Gattung und der
editorischen Aufgabe & Ergänzung der Quellenbeschreibung zum
transkribierten Brief & Wissensvermittlung über foliengestützten Vortrag
& Beteiligung an Diskussion und Übung zur Paläographie. \\
7--9 & synchron, Präsenz & Zwischenberichte zu Quellenbeschreibungen und
Transkriptionen & Präsentationen der Studierenden & Kooperative Lösung
bestehender Probleme und Beantwortung offener Fragen & Erstellung von
mindestens fünf Transkriptionen als XML-Dateien & Hilfestellungen.
Diskussionsleitung & Präsentation zum Zwischenstand. Beteiligung an
Diskussion \\
10 & synchron, Präsenz & Kommentierung & V.a. TEI
\passthrough{\lstinline!namedates!}~(Auswahl), Ergänzungen aus TEI
\passthrough{\lstinline!core!} & Beispielhafte Ergänzung von
Auszeichnungen. Kenntnis von Hilfsmitteln und Anforderungen an den
Kommentar & Überarbeitung der fünf transkribierten Briefe &
Wissensvermittlung über foliengestützten Vortrag & Beteiligung an
Diskussion und Übung zum Kommentar \\
11--13 & synchron, Präsenz & Zwischenberichte zur Kommentierung &
Präsentationen der Studierenden & Kooperative Lösung bestehender
Probleme und Beantwortung offener Fragen & Erstellung von mindestens
fünf transkribierten und kommentierten Briefen im XML-Format &
Hilfestellung. Diskussionsleitung & Präsentation der Zwischenstände.
Beteiligung an Diskussion \\
14 & synchron, Präsenz & Abschlussdiskussion & Beantwortung offener
Fragen und Problemstellungen. Evaluation. & Kooperative Lösung
bestehender Probleme und Beantwortung offener Fragen & Zusendung
ausgewählter Problemstellen an zugeloste Kommiliton:innen &
Hilfestellung. Diskussionsleitung. Erstellung und Präsentation einer
exemplarischen HTML-Ansicht & Arbeit an weiteren Briefen. Mitteilung von
Fragen und Problemen \\
\end{longtable}

\end{landscape}
\normalsize

\section{3. Sitzungsübersicht}\label{sitzungsuxfcbersicht}

\subsection{3.1 Grundlagen der Editorik (Sitzung
1)}\label{grundlagen-der-editorik-sitzung-1}

Editorische Konzepte und Operationen sind Studierenden vor der
Veranstaltung in der Regel nur oberflächlich oder gar nicht vertraut. In
der ersten Sitzung wird daher auf verschiedene Problemstellungen der
Editorik eingegangen:

\begin{enumerate}
\def\labelenumi{\arabic{enumi}.}
\item
  \emph{Heuristik} als die Aufgabe, alle relevanten Textzeugen zu
  identifizieren, ist für die Veranstaltung selbst von untergeordneter
  Bedeutung. Da Privatbriefe behandelt werden, die dem Liebesbriefarchiv
  gespendet wurden, liegen bereits Katalogeinträge vor. In einzelnen
  Fällen ist dennoch eine Prüfung und gegebenenfalls neue Zuordnung
  notwendig, wenn beispielsweise Briefe unvollständig oder Seiten falsch
  sortiert sind.
\item
  Im Zusammenhang mit der \emph{Prüfung der Authentizität} von
  Textzeugen wird auf das Beispiel von Valla
  (\citeproc{ref-Valla-1994}{1994}) und die Konstantinische Schenkung
  eingegangen sowie auf Möglichkeiten der Materialanalyse hingewiesen.
  Da in der Veranstaltung mit Digitalisaten gearbeitet wird, spielen
  diese Möglichkeiten in späteren Einheiten keine Rolle mehr.
\item
  Die \emph{Quellenbeschreibung} als ein Moment der Erschließung und als
  Voraussetzung der editorischen Arbeit ist aufgrund der Verwendung von
  digitalen Reproduktionen eingeschränkt. Metadaten, die sich speziell
  auf schriftliche Korrespondenzen beziehen, werden jedoch ausführlich
  thematisiert.
\item
  Von besonderer Relevanz für die Lehrveranstaltung sind die
  \emph{Transkription} und die Berücksichtigung der \emph{Textgenese},
  verstanden als die Entstehung des Textes auf dem überlieferten
  Dokument. Diese Schritte bilden einen der beiden Schwerpunkte.
\item
  \emph{Textkritik}, \emph{Kollation} und \emph{Stemmatologie} sowie die
  Methoden der \emph{recensio} und \emph{emendatio} werden erklärt. Weil
  aber Briefe untersucht werden, die unikal überliefert sind, finden
  diese im weiteren Kursverlauf keine praktische Anwendung.
\item
  Neben der Transkription stellt der \emph{Kommentar} den zweiten
  Schwerpunkt dar. Nach einem Vortrag des Lehrenden über Theorie und
  Praxis des Kommentars wird das Augenmerk auf die Methoden und die
  benutzten Quellen bei der Erstellung von Einzelstellenkommentaren
  gelenkt.
\item
  Abschließend wird die Frage der \emph{medialen Präsentation}
  angesprochen. Übliche Verfahren zur Transformation und die
  Möglichkeiten, die der \emph{Author Mode} im Oxygen XML Editor bietet,
  werden benannt und zum Teil angewandt.
\end{enumerate}

Dieses Wissen wird in Form einer Umfrage zunächst abgefragt und durch
begleitende Erläuterungen des Dozenten vermittelt. Dies ist notwendig,
weil es im weiteren Verlauf darum geht, diese Strukturen und Phänomene
digital zu modellieren und durch Annotationen zu erfassen.

Es kann demotivierend sein, dass die mediale Präsentation der Ergebnisse
kein wesentlicher Teil der Übung ist. In der ersten Sitzung betrachten
und analysieren die Studierenden jedoch gedruckte und digitale Editionen
und vergleichen deren Schichten und Funktionalitäten. In der
abschließenden Sitzung wird dann anhand eines Briefes eine exemplarische
Umsetzung in HTML mit dem vorgeführt, um den gesamten Produktionsprozess
der Edition nicht aus dem Blick zu verlieren. Für die Vertiefung werden
die Teilnehmenden auf Plachta (\citeproc{ref-Plachta-2013}{2013}),
Plachta (\citeproc{ref-Plachta-2020}{2020}) sowie Sahle
(\citeproc{ref-Sahle-2013}{2013}) hingewiesen.

\subsection{3.2 Gattungstheorie des Briefes (Sitzungen 2 und
3)}\label{gattungstheorie-des-briefes-sitzungen-2-und-3}

In der Lehrveranstaltung wird in der Folge mit einer Sitzung zur Gattung
des Briefes an lebensweltliche und wissenschaftliche Kenntnisse der
Teilnehmenden zu diesem Thema angeknüpft. Ziel ist, ein Bewusstsein für
die Spezifika des Briefes im Hinblick auf Textstrukturen und
Textphänomene zu schaffen, das für deren Behandlung in Editionen und
Editionsrichtlinien grundlegend ist. In Gruppen wird zunächst an
Definitionen und Explikationen gearbeitet. Die Ergebnisse werden mit
Lexikonartikeln verglichen. Für die Vertiefung zu diesem Themenfeld
werden Matthews-Schlinzig u.~a.
(\citeproc{ref-Matthews-Schlinzig-2020-a}{2020a}) und Matthews-Schlinzig
u.~a. (\citeproc{ref-Matthews-Schlinzig-2020-b}{2020b}) genannt.

Es werden anschließend Aufsätze zur Edition von Briefen behandelt, die
jeweils unterschiedliche Aspekte betonen. Konkret werden die
editionswissenschaftlichen Ansätze von Gregolin
(\citeproc{ref-Gregolin-1990}{1990}), Zeller
(\citeproc{ref-Zeller-2002}{2002}) und Strobel
(\citeproc{ref-Strobel-2019}{2019}) jeweils in Gruppen behandelt und
dann gemeinsam im Plenum diskutiert. Auf diese Weise wird das Wissen
über Editionen mit dem Wissen über Briefe in Verbindung gebracht.
Ausgewählt wurden diese Aufsätze wegen ihrer konträren Positionen. Der
Kurs wird in drei Gruppen geteilt. Jede Gruppe konzentriert sich auf
einen der Texte, fasst die Argumentation zusammen und erarbeitet
mögliche Gegenargumente sowie deren etwaige Entkräftung. Ein Ergebnis
kann sein, dass bei Briefen sowohl textuelle als auch materielle und
pragmatische Aspekte editorisch zu beachten sind. Es kann diskutiert
werden, wo Schwerpunkte zu legen und wie die Aspekte editorisch
umzusetzen sind.

\subsection{3.3 Grundlagen der Textauszeichnung (Sitzungen 2 und
4)}\label{grundlagen-der-textauszeichnung-sitzungen-2-und-4}

Im informatischen Bereich stehen das Erlernen und das Verständnis von
Auszeichnungssprachen im Mittelpunkt der Veranstaltung. Deren
Grundannahmen und -anliegen werden daher ausführlicher behandelt.
Auszeichnungssprachen (\emph{Markup Languages}) wie SGML, HTML und XML
werden eingeführt, ihre Entwicklungen und Funktionen erläutert und
reflektiert.

Da in der Lehrveranstaltung keine Kenntnisse im Umgang mit XML-Daten
vorausgesetzt werden (siehe Abschnitt 1.2), sind zunächst die Grundlagen
in diesem Bereich zu vermitteln (Sitzungen 2 und 4). Diese umfassen
unter anderem die Konzepte von deskriptivem Markup sowie von
Wohlgeformtheit und Validität. Auf dieser Basis können anschließend die
Grundlagen von Textauszeichnungen nach den Richtlinien der TEI
vorgestellt werden. Diese umfassen unter anderem die Unterteilung der
Dokumente in Kopf (\passthrough{\lstinline!teiHeader!}) und Textkörper
(\passthrough{\lstinline!body!}), aber auch die Erfassung elementarer
Textstrukturen wie Abschnitte (\passthrough{\lstinline!div!}) und
Absätze (\passthrough{\lstinline!p!}). Insgesamt werden vor allem aus
den folgenden TEI-Modulen Elemente eingeführt und verwendet:

\begin{itemize}
\item
  \passthrough{\lstinline!core!}
\item
  \passthrough{\lstinline!header!}
\item
  \passthrough{\lstinline!msdescription!}
\item
  \passthrough{\lstinline!namesdate!}
\item
  \passthrough{\lstinline!textstructure!}
\item
  \passthrough{\lstinline!transcr!}
\end{itemize}

Zentraler Referenzpunkt für Studierenden sind neben den zur Verfügung
gestellten Folien die der TEI die auf der entsprechenden Webseite
versammelten Lehrmaterialien. Für die praktische Umsetzungen werden den
Studierenden in der Übung eine Beispieldatei, ein RELAX-NG-Schema sowie
ein Glossar in der verwendeten Lernplattform zur Verfügung gestellt.
Obwohl nur die besagten Module im Kurs detailliert behandelt werden,
weicht die Schemadatei nur geringfügig von \emph{TEI-All} ab.
\emph{TEI-All} dient als Orientierungspunkt, weil in den Briefen
zahlreiche Phänomene wie Bilder, Noten, Verse oder Fragmente aus
Theaterstücken auftreten. Die Abweichung von \emph{TEI-All} betrifft die
Modellierung von Grußformeln (\passthrough{\lstinline!salute!}) in
Briefen. Innerhalb der Veranstaltung und des mit ihr verbundenen
Projekts werden diese zusätzlich der Klasse
\passthrough{\lstinline!segLike!} zugeordnet, sodass sie abweichend vom
Standardmodell auch innerhalb von Absätzen (\passthrough{\lstinline!p!})
vorkommen können.

Falls genug Zeit vorhanden ist und innerhalb des Kurses Interesse
besteht, wird gezeigt und erklärt, wie Schemadateien aufgebaut sind und
mit dem Werkzeug Änderungen am Schema vorgenommen werden können. Die
Schemadatei dient den Studierenden, um die Validität ihrer Dateien zu
prüfen. Die Validität ist zudem neben der Wohlgeformtheit ein
Bewertungskriterium für die eingereichten Dateien. Die Beispieldatei
zeigt in Verbindung mit dem Glossar und dem Schema auf, welche Tiefe der
Auszeichnung bei verschiedenen Textphänomenen erwartet wird.

\subsection{3.4 Quellenbeschreibung (Sitzung
5)}\label{quellenbeschreibung-sitzung-5}

Obwohl in der Übung mit Digitalisaten gearbeitet wird, sollen die
materiellen Aspekte der Quellenbeschreibung, die in der Gattungs- und
Editionstheorie des Briefes angesprochen werden, nicht vollständig
ausgeblendet werden. So wird die mediale Form bezeichnet, indem zum
Beispiel zwischen Briefen, Post- und Grußkarten, Telegrammen und
Feldpost unterschieden wird. Außerdem werden Schriftträger und
Beschreibstoffe grob klassifiziert (vor allem im Rückgriff auf den )
sowie die Erscheinung der Schrift, zum Beispiel die Schriftrichtung, mit
\passthrough{\lstinline!@rend!}, \passthrough{\lstinline!@rendition!}
und CSS erfasst. Zur Handschriftenbeschreibung
(\passthrough{\lstinline!msDesc!}) werden Angaben zu Handlungen, Orten
und Personen im Zusammenhang mit dem Korrespondenzgeschehen in
\passthrough{\lstinline!correspDesc!} ergänzt. Es werden Register für
Entitäten erstellt und in Verbindung mit Briefmarken, Vordrucken und
Stempeln (\passthrough{\lstinline!stamp!} und
\passthrough{\lstinline!fw!}) können diese Informationen Hinweise auf
die situativen Umstände und den historischen Kontext geben. Anhänge und
Beilagen (\passthrough{\lstinline!accMat!} und
\passthrough{\lstinline!back!}) werden verzeichnet, beschrieben und
gegebenenfalls transkribiert. Damit wird die pragmatische Dimension des
Briefwechsels erschlossen.

\subsection{3.5 Transkriptionen (Sitzungen 6 bis
9)}\label{transkriptionen-sitzungen-6-bis-9}

Bei der wissenschaftlichen Umschrift wird vor allem die textuelle
Dimension der Briefe adressiert. Es wird eine Einheit zur deutschen
Paläographie des 20. Jahrhunderts angeboten. In dieser Sitzung wird eine
Übersicht über die Entwicklung der Schrift anhand von Tafeln gegeben und
ein kurzer Brief aus dem Korpus gemeinsam entziffert. Außerdem werden
textgenetische Phänomene wie Ersetzungen
(\passthrough{\lstinline!subst!}) behandelt.

\subsection{3.6 Kommentierung (Sitzungen 10 bis
13)}\label{kommentierung-sitzungen-10-bis-13}

In der Übung ist die Kommentierung von Einzelstellen die zweite zentrale
Aufgabe. Die Kommentierung erfolgt mit den Elementen
\passthrough{\lstinline!note!}, \passthrough{\lstinline!term!} und
\passthrough{\lstinline!gloss!}. Das Lemma
(\passthrough{\lstinline!term!}) muss nicht identisch mit dem Ausdruck
im Text sein. Als Referenzpunkt im Text können bestehende
Auszeichnungen, etwa Absätze (\passthrough{\lstinline!p!}), oder
ergänzte Elemente wie \passthrough{\lstinline!seg!} genutzt werden. Das
Attribut \passthrough{\lstinline!@corresp!} dient der Verknüpfung von
Textstelle und Anmerkung. Die Referenz wird in beide Richtungen gesetzt.
Auf diese Weise können mehrere Stellen mit einer Anmerkung verknüpft
werden. Exemplarisch kann die Auszeichnung wie folgt aussehen:

\begin{lstlisting}
<seg corresp="#note-01" xml:id="seg-01">
  <rs ref="#Mariawald">Mariawald</rs>
</seg>
. . .
<note type="editorial" corresp="#seg-01" xml:id="note-01">
  <term>Mariawald</term>
  <gloss>Anspielung auf das Schweigegelübde in einem Kloster in der Eifel am gleichnamigen Ort. In seiner Autobiographie beschreibt ...</gloss>
</note>
\end{lstlisting}

Der Kommentierung bedürftige Ausdrücke können sich zum Beispiel auf
Objekte, Orte, Personen und Körperschaften, Zitate und Anspielungen,
historische Ereignisse und Zustände beziehen. Die Arbeit an diesen
umfasst Verweise auf Normdaten für Entitäten, soll sich aber nicht
hierauf beschränken. Vielmehr soll ergänzend kommentiert werden, wenn
Identifizierungen schwierig oder nicht eindeutig sind. Ist die
Identifikation unproblematisch, wird kein Kommentar verfasst und nur
Normdaten im Index verzeichnet.

Den Studierenden wird eingangs eine Liste mit Hilfsmitteln, vor allem
Hand- und Wörterbücher verschiedener Disziplinen, an die Hand gegeben.
In diesem Fall soll jedoch wie überhaupt versucht werden, die
verwendeten Informationen mindestens durch zwei möglichst voneinander
unabhängige Quellen zu belegen. Fachspezifische Handbücher und Lexika
sollen vor allem „Relais`` auf dem Weg zu wichtiger Fachliteratur und
historischen Belegen sein, die von den Teilnehmenden auszuwerten ist.
Die Kommentare sind somit Ergebnisse zum Teil aufwändiger, von den
Studierenden selbst zu leistender Forschung. Quellen für die Erläuterung
sollten dem historischen und kulturellen Umfeld immer möglichst nah
sein. Da die Briefe Privatdokumente sind, können biographische Quellen,
beispielsweise Querverweise zwischen den Briefen, aufschlussreich sein.
Um Wiederholungen zu vermeiden, kann auf Kommentare in anderen Briefen
verwiesen werden (\passthrough{\lstinline!ref[@target]!}).

Die Studierenden werden in den Briefen oft mit der Lebenswelt der
Korrespondierenden unmittelbar konfrontiert. Gerade das, was
Schreibenden selbstverständlich und vertraut ist, erschwert oft die
Kommentierung. Da die Gruppe manchmal sehr heterogen ist, ergeben sich
manchmal lange und komplizierte Recherchepfade, aber auch lehrreiche
Diskussionen. Während die Einzelstellenkommentare möglichst kurz sein
sollen, wurden in einigen Durchführungen der Übung diese Wege zur
Erkenntnis exemplarisch in einem Bericht dargestellt. Im hier
vorgestellten Konzept kommen diese Recherchen ausschließlich in den
Zwischenberichten vor. Neben Korrektheit, Vollständigkeit und Konsistenz
ist aber Selbstständigkeit nach wie vor ein Kriterium von hohem Gewicht
bei der Bewertung der Kommentare.

In der ersten Sitzung dieses Blocks wird der Brief, der bereits für die
paläographische Übung genutzt wurde, kommentiert. Die Teilnehmenden
bestimmen zunächst für sich, was kommentiert werden soll. Nach einer
Aussprache darüber, recherchieren sie zunächst für sich alleine, bevor
die Ergebnisse in der Gruppe vorgestellt und gemeinsam Kommentare
verfasst werden. In der letzten Sitzung der Einheit bringen Studierende
ihre Rechercheergebnisse zu ausgewählten Fragen zugeloster anderer
Teilnehmender ein.

\section{4. Reflexion}\label{reflexion}

\subsection{4.1 Rahmenbedingungen und Ausführung der
Veranstaltung}\label{rahmenbedingungen-und-ausfuxfchrung-der-veranstaltung}

\begin{enumerate}
\def\labelenumi{\arabic{enumi}.}
\item
  Aufgetreten sind in der Lehrveranstaltung vormals Schwierigkeiten vor
  allem beim selbstbestimmten Formulieren einer Forschungsfrage sowie
  bei der Begrenzung der Fragestellung in Hinblick auf die zur Verfügung
  stehende Zeit. Es wurden schon im Vorhinein Maßnahmen getroffen, um
  auf diese erwarteten Schwierigkeiten zu reagieren. Um Fehlplanungen
  vorzubeugen, wurden eingangs jeweils eine Projektplanung mit
  Zeitgerüst von den Studierenden entwickelt sowie mehrere
  Zwischenberichte mit Rückmeldungen eingeplant. Um Blockaden bei der
  Themenfindung vorzubeugen, wurde vom Dozenten ein Repertoire von
  Materialien und Aufgaben angelegt, auf das bei Bedarf und nach
  individuellem Interesse der Studierenden zurückgegriffen werden
  konnte.
\item
  Im Laufe der Jahre wurde mit einer stärkeren Normierung von Material
  und Aufgaben auf die genannten Probleme reagiert. Ursprünglich wurde
  in der Übung, die das selbstregulierte Lernen fördern soll und dieses
  auch voraussetzt, den Teilnehmenden selbst überlassen, eine für sie
  interessante Forschungsfrage zu formulieren. Der Gegenstand musste im
  Bereich der digitalen Annotationen liegen, aber nicht notwendigerweise
  in dem der digitalen Editorik. Dies führte zu einem hohen
  Betreuungsaufwand, da die Leistungen trotz heterogener Profile der
  Studierenden in Bezug auf editorische, informatische und sprachliche
  Kompetenzen vergleichbar sein müssen (siehe Abschnitt 1.3.3). Die
  beiden genannten Strategien, also der Entwurf eines Zeitplans und die
  Bereithaltung von Materialien, wurden auch beibehalten, nachdem das
  Lehrkonzept angepasst wurde. Die Quellen aus dem Liebesbriefarchiv
  waren zunächst nur eine solche Option aus dem Repertoire, wurden im
  Sinne der Vergleichbarkeit aber später zu dem gemeinsam bearbeiteten
  Material. Diese gemeinsame Datengrundlage hat dabei auch zu mehr
  Austausch zwischen den Teilnehmenden geführt sowie gegenseitigen
  Hilfestellungen ermöglicht. In der Lernplattform wird zur
  Unterstützung dieses Austausches ein Forum genutzt. Die Entwicklung
  einer eigenen Forschungsfrage wurde zwischenzeitlich auf einen
  schriftlichen Bericht zu einer für besonders interessant gehaltenen
  Frage der Kommentierung reduziert. Mittlerweile wird nur noch bei den
  zwei mündlichen Präsentationen berichtet. Entwarfen die Studierenden
  zuvor ihre eigenen Projekte, so arbeiten sie nun gemeinsam an einem
  Projekt mit. Der Anteil selbstregulierten Lernens ist damit
  eingeschränkt gegenüber dem früheren Vorgehen und auch gegenüber
  anderen Konzepten, wie sie Walsh (\citeproc{ref-Walsh-2023}{2023}) im
  Hinblick auf heterogene Lerngruppen beschreibt.
\item
  Das Fehlen einer praktischen Einheit zur Erstellung einer
  Webpräsentation, wie sie im holistischen Ansatz von Rehbein und Fritze
  (\citeproc{ref-Rehbein-2012}{2012}) und in vergleichbaren
  Lehrveranstaltungen (siehe Abschnitt 4.3) beinhaltet ist, kann ein
  Nachteil für die studentische Motivation sein, da nicht der gesamte
  Produktionsprozess einer digitalen Edition durchlaufen wird. Umgekehrt
  bewirkt diese Reduktion eine Konzentration der Lehrinhalte und -ziele
  auf andere Aspekte der editorischen Arbeit und schließt auch nicht
  aus, dass der gesamte Prozess bis zur fertigen Edition samt
  verschiedener dazu geeigneter Technologien angesprochen wird. Um den
  Prozess vollständig dazustellen und den erfolgreichen Abschluss der
  Arbeit vor Augen zu führen, spielt nunmehr der Dozent die Daten in
  eine eher minimalistisch gehaltene Oberfläche ein, um das Ergebnis in
  der Abschlusssitzung zu zeigen (siehe Abschnitt 3.1).
\item
  Es blieb trotz der Änderungen am Gesamtkonzept ein Ziel,
  selbstreguliertes Lernen zu fördern. Dabei sollte zum einen ein hoher
  Grad an Transparenz für die Benotung gegeben sein, zum anderen sollten
  die Studierenden durch den computer- und editionsphilologischen Umgang
  mit historischen Archivmaterialien in ein Arbeitsfeld eingeführt
  werden, mit dem viele zuvor nur am Rande oder gar nicht in Kontakt
  gekommen sind. Auf selbstreguliertes Lernen wird damit nicht
  vollkommen verzichtet, zumal die Studierenden die Zeit für die
  Transkription und Kommentierung weitgehend selbst einschätzen müssen.
  Sie arbeiten dabei zwischen den Sitzungen und nach Ende der
  Vorlesungszeit unter Selbstkontrolle und -regulation, das heißt
  weitgehend ohne Kontakt mit dem Lehrenden (siehe
  \citeproc{ref-Zimmermann-2000}{Zimmermann 2000, 30--31}). Ein eigener
  Gestaltungsraum zur Umsetzung des Vorhabens wird gewahrt, indem für
  die Zwischenberichte nur Mindestzahlen an Briefen festgelegt werden.
  Auch können die Studierenden zu diesen Zeitpunkten die Materialien
  unterschiedlich tief erschlossen haben. Eigene Nachforschungen sind
  zudem für die Kommentierung unumgänglich. Bei den Transkriptionen
  treten ferner oftmals neue Phänomene auf, die modelliert und
  ausgezeichnet werden müssen. Mit den Zwischenberichten haben die
  Teilnehmenden die Möglichkeit, Schwierigkeiten anzusprechen und
  gemeinsam mit dem Dozenten und der Gruppe Lösungen zu finden. Es
  wurden auch in jedem Kurs Einheiten vorgesehen, in denen sich je zwei
  Studierende gegenseitig austauschen und ihre Arbeiten beurteilen
  (siehe Abschnitt 3.6).
\end{enumerate}

\subsection{4.2 Studierende}\label{studierende}

Da die Lernenden insbesondere im Bereich der Editorik sowie im Umgang
mit XML heterogene Vorkenntnisse aufweisen, werden in der Veranstaltung
entsprechende Kenntnisse mittlerweile nicht mehr vorausgesetzt und
bestimmte Inhalte wie XSLT und XQuery zurückgestellt. Dafür sollte der
Umgang mit historischen Schriftdokumenten als Bestandteil der Ausbildung
beibehalten werden, weshalb die Einheit zur Paläographie hinzugefügt
wurde. Dabei wurde die frühzeitige gemeinsame Transkription eines
Briefes als eine notwendige Ergänzung gesehen. Die Heterogenität der
Kompetenzprofile wird darüber hinaus durch häufige Gruppenarbeiten in
den Sitzungen vor Ort kompensiert.

Die curricularen Rahmenbedingungen standen nicht im Zentrum dieses
Beitrags. Dennoch ist der Zusammenhang vermutlich nicht zweifelhaft.
Selbst bei Veranstaltungen in vergleichbaren Studiengängen stellen sich
rasch Unterschiede heraus. In Studiengängen der Digitalen
Geisteswissenschaften können oftmals Kenntnisse zu Datenmodellierung und
-verarbeitung angenommen werden. Der Bedarf, diese im Hinblick auf
digitale Editionen grundlegend zu vermitteln, besteht in der
Fachgemeinschaft indes weiterhin. Dies zeigt das Interesse an Formaten
wie einem Workshop (\citeproc{ref-Hegel-2023}{Hegel, Ries und Viehhauser
2023}), der eine Abwandlung des vorhandenen Kurskonzeptes darstellt und
mit Einheiten aus ähnlichen Lehrveranstaltungen kombiniert wurde.

Aufgrund der verschiedenen Anforderungen an derartige Übungen, vor allem
aufgrund der Vermittlung nicht nur editorischer, sondern auch
computerphilologischer Kompetenzen (siehe Abschnitt 1) werden
vergleichbare Veranstaltungen andernorts auf zwei Semester verteilt. Mit
Blick auf die digitalen Kompetenzen erscheint es eine Überlegung wert,
ob editorische Kenntnisse in philologischen und ähnlichen Studiengängen
gleich mit digitalen Methoden verbunden werden können.

Ähnlich wie die praktische Editorik theoretische Erkenntnisse über den
edierten Gegenstand voraussetzt und ermöglicht, kann der Unterricht
idealiter in der Textkodierung eine Reflexion der digitalen Praktiken
von Datenstrukturierung und -verarbeitung ermöglichen (vgl.
\citeproc{ref-McCarty-2005}{McCarty 2005};
\citeproc{ref-Burnard-2018}{Burnard 2018, 108--109, 112--114}; sowie
\citeproc{ref-Sahle-2015}{Sahle 2015}). Dies betrifft nicht nur die
Frage, wie der Gegenstand digital adäquat modelliert kann, sondern auch
die Frage, wie er digital präsentiert und bereitgestellt werden soll.
Ein solcher Unterricht steht vor der Aufgabe, sowohl den Gegenstand der
Edition zu überdenken als auch die medialen Spezifika der digitalen
Verarbeitung. Man kann dies positiv wenden: Der Unterricht ermöglicht
idealerweise eine tiefere Einsicht in das behandelte Objekt und in die
Methode seiner Behandlung, indem Strategien zur Datenmodellierung und
-verarbeitung gefunden werden, die dem Gegenstand und der Fragestellung
angemessen sind (vgl. \citeproc{ref-Behrmann-1991}{Behrmann 1991}, v.a.
8 und 11; sowie \citeproc{ref-Mahony-2012}{Mahony und Pierazzo 2012,
223--224}).

\subsubsection{4.3 Alternativen}\label{alternativen}

In vergleichbaren Lehrkontexten können die einzelnen Schritte anders
gewichtet sein. Wenn der Fokus auf dem Erproben von standardisierten
Workflows und Tools liegt, spielen die editorischen Grundlagen eine
indirektere Rolle. In Bezug auf die Punkte 1 bis 5 aus Abschnitt 3.1
werden dann bspw. Problemfelder nur angesprochen und auf vertiefende
Literatur sowie Recherchemittel wie das zum Selbststudium hingewiesen.
In anderen Veranstaltungen wird auf die \emph{Kommentierung} verzichtet,
während sich ein ganzer Block dem Themenfeld \emph{Mediale Präsentation}
widmet. Im Rahmen des Blocks \emph{Transkription} (s. Abschnitt 3.5)
wird in vergleichbaren Veranstaltungen üblicherweise auch das
einschlägige TEI-Kapitel zum kritischen Apparat
(\passthrough{\lstinline!textcrit!}) mit den verschiedenen Ansätzen der
lokalen Referenz, der doppelten Endpunktmarkierung und der parallelen
Segmentierung behandelt. Es wird aber im hier vorgestellten Lehrkonzept
bewusst ausgespart, da es aufgrund der Quellenlage eine weniger wichtige
Rolle spielt.

Je nach Studienmöglichkeiten können die fachlichen Hintergründe der
Studierenden variieren, sodass sich auch Inter- und Transdisziplinarität
unter Umständen als weitere Lernziele anbieten. Methodische Varianz kann
durch den zeitlichen Rahmen oder durch die Anbindung an
Forschungsvorhaben begründet sein, aber auch im Hinblick auf die
vorausgesetzten Kompetenzen erfolgen. Welche Möglichkeiten hier offen
stehen, sollen folgende Hinweise zu einem vergleichbaren Kurs andeuten,
der auf einer Übung zu X-Technologien aufbaut und auf eine Veranstaltung
zu Webtechnologien vorbereitet:

Statt Briefe manuell zu entziffern, erschließen in diesem Kurs zunächst
alle Teilnehmenden je einen handgeschriebenen Brief und ein Typoskript
mit , um praktische Erfahrung mit einem gängigen OCR/HTR-Werkzeug zu
erlangen. In einer anschließenden Gruppendiskussion werden dann typische
Problemfälle gemeinsam herausgearbeitet. Zusätzlich können Studierende
vorhandene XSLT-Kenntnisse vertiefen, indem sie ein eigenes Skript zur
Transformation des eScriptorium-Outputs in valides TEI entwickeln.

In einem zweiten Schritt entwickeln die Teilnehmenden unter
Hilfestellung ein XSLT-Skript, das Personen und Orte extrahiert. Das so
generierte Personenregister wird dann mit um Normdaten
(\href{https://gnd.network/}{GND}-IDs) angereichert. Umgekehrt erfolgt
die Anreicherung des Ortsregister durch manuelle Recherche in , sodass
die Teilnehmenden beide Wege kennenlernen. Des weiteren erfassen die
Studierenden Normdaten als CMIF (\citeproc{ref-Dumont-2020}{Dumont u.~a.
2020}) im Metadatenblock der TEI-Dokumente, was eine Validierung und
Vorschau über den im Briefkontext weit verbreiteten Webservice
(\citeproc{ref-Dumont-2016}{Dumont 2016}) ermöglicht.

Um die Methodenvielfalt in den Digitalen Geisteswissenschaften
aufzuzeigen, enthält der Vergleichskurs auch eine kleine Einheit mit
Einblicken in Nachbardisziplinen wie dem \emph{Natural Language
Processing}. Diese beginnt mit einer freien Exploration der Briefinhalte
mit den , gefolgt von einem gemeinsamen Erkenntnisaustausch. Darüber
hinaus hat die Kursleitung Batch- bzw. Shellskripte vorbereitet, die
XSLTs mit dem (Schmid (\citeproc{ref-Schmid-1995}{1995})) kombinieren,
um die TEI-Texte in eine lemmatisierte Version zu transformieren. Auf
diese Weise lernen die Teilnehmenden alternative Kodierungs- und
Analysemöglichkeiten kennen und können deren Vor- und Nachteile
explorieren und diskutieren.

Abschließend bereitet eine Gruppe von Studierenden das Personenregister
zu einer HTML-Ansicht auf, während eine andere eine Briefeinzelansicht
mit XSLT generiert, sodass sie praxisnahe Einblicke in alle Abschnitte
eines digitalen Editionsprozesses erfahren.

Alle diese real praktizierten Lehreinheiten sollen zeigen, dass das
vorgestellte Lehrkonzept in der Regel nicht \emph{in toto} übertragen
werden kann. Die Modularität sowohl der einzelnen Werkzeuge als auch der
TEI-Richtlinien erleichtern allerdings die Übernahme einzelner
Bestandteile aus diesem Konzept oder dem eingangs genannten frei
zugänglichen Lehrmaterial. Die Auswahl der Module hängt von
verschiedenen Faktoren ab: a) von den Kompetenzprofilen der
Teilnehmenden, b) der Einbindung in Curricula mit ihren jeweiligen
Voraussetzungen, Lernzielen und Zeitgerüsten, c) der Anbindung an
Forschungsprojekte sowie d) den individuellen wie institutionellen
technischen Voraussetzungen. Ein weiterer, aber entscheidender Faktor
wurde dabei noch nicht genannt: e) die Kompetenzprofile der Lehrenden.

\phantomsection\label{refs}
\begin{CSLReferences}{1}{0}
\bibitem[\citeproctext]{ref-Behrmann-1991}
Behrmann, Alfred. 1991. \emph{Philologische Praxis}. Bd. 2. Stuttgart:
Metzler.

\bibitem[\citeproctext]{ref-Boekaerts-2000}
Boekaerts, Monique und Markku Niemivirta. 2000. Self-Regulated Learning:
Finding A Balance Between Learning Goals and Ego-Protective Goals. In:
\emph{Handbook of Self-Regulation}, hg. von Monique Boekaerts, Paul R.
Pintrich, und Moche Zeidner, 417--450. San Diego: Academic Press.

\bibitem[\citeproctext]{ref-Burghart-2017}
Burghart, Marjorie und Elena Pierazzo. 2017. Digital Scholarly Editions:
Manuscripts, Texts and TEI Encoding. 17. Juli.
\url{https://teach.dariah.eu/course/view.php?id=32&section=0}
(zugegriffen: 21. Februar 2025).

\bibitem[\citeproctext]{ref-Burnard-2018}
Burnard, Lou. 2018. How Modeling Standards Evolve: The Case of the TEI.
In: \emph{The Shape of Data in Digital Humanities. Modeling Texts and
Text-Based Resources}, hg. von Julie Flanders und Fotis Jannidis,
99--116. London: Routledge.

\bibitem[\citeproctext]{ref-DeRose-1990}
DeRose, Steven J., David G. Durand, Elli Mylonas und Allen H. Renear.
1990. What Is Text, Really? \emph{Journal of Computing in Higher
Education} 1, Nr. 2: 3--26.

\bibitem[\citeproctext]{ref-DFG-2015}
Deutsche Forschungsgemeinschaft. 2015. Förderkriterien für
wissenschaftliche Editionen in der Literaturwissenschaft.
\emph{Informationen für Geistes- und Sozialwissenschaftler/innen}, Nr.
11.
\url{https://www.dfg.de/resource/blob/172080/895fcc3cb72ec6cd1d5dc84292fb2758/foerderkriterien-editionen-literaturwissenschaft-data.pdf}
(zugegriffen: 21. Februar 2025).

\bibitem[\citeproctext]{ref-Dumont-2016}
Dumont, Stefan. 2016. {correspSearch} -- {Connecting} {Scholarly}
{Editions} of {Letters}. \emph{Journal of the Text Encoding Initiative},
Nr. 10. doi:
\href{https://doi.org/10.4000/jtei.1742}{10.4000/jtei.1742},
\url{https://journals.openedition.org/jtei/1742} (zugegriffen: 21.
Februar 2025).

\bibitem[\citeproctext]{ref-Dumont-2020}
Dumont, Stefan, Ingo Börner, Jonas Müller-Laackman, Dominik Leipold und
Gerlinde Schneider. 2020. Correspondence Metadata Interchange Format
(CMIF). Berlin-Brandenburg Academy of Sciences; Humanities, 22. April.
\url{https://encoding-correspondence.bbaw.de/v1/CMIF.html} (zugegriffen:
21. Februar 2025).

\bibitem[\citeproctext]{ref-Goldfarb-1981}
Goldfarb, Charles F. 1981. A Generalized Approach to Document Markup.
In: \emph{Proceedings of the ACM SIGPLAN SIGOA Symposium on Text
Manipulation}, 68--73. New York: Association for Computing Machinery.

\bibitem[\citeproctext]{ref-Goldfarb-2004}
Goldfarb, Charles F. und Paul Prescod. 2004. \emph{XML Handbook}. 5.
Aufl. Upper Saddle River: Prentice Hall.

\bibitem[\citeproctext]{ref-Goetz-2017}
Götz, Thomas und Ulrike F. Nett. 2017. Selbstreguliertes Lernen. In:
\emph{Emotion, Motivation und selbstreguliertes Lernen}, hg. von Thomas
Götz, 144--185. 2. Aufl. Paderborn: Schöningh.

\bibitem[\citeproctext]{ref-Gregolin-1990}
Gregolin, Jürgen. 1990. Briefe als Texte: Die Briefedition.
\emph{Deutsche Vierteljahresschrift für Literaturwissenschaft und
Geistesgeschichte} 64, Nr. 4: 756--771.

\bibitem[\citeproctext]{ref-Hegel-2023}
Hegel, Philipp, Thorsten Ries und Gabriel Viehhauser. 2023. Workshop:
Digitale Edition -- Grundlagen und Perspektiven.
\url{https://events.gwdg.de/event/531/overview} (zugegriffen: 21.
Februar 2025).

\bibitem[\citeproctext]{ref-Landmann-2015}
Landmann, Meike, Franziska Perels, Barbara Otto, Kathleen
Schnick-Vollmer und Bernhard Schmitz. 2015. Selbstregulation und
selbstreguliertes Lernen. In: \emph{Pädagogische Psychologie}, hg. von
Elke Wild und Jens Möller, 45--65. 2. Aufl. Berlin: Springer.

\bibitem[\citeproctext]{ref-Mahony-2012}
Mahony, Simon und Elena Pierazzo. 2012. Teaching Skills or Teaching
Methodology? In: \emph{Digital Humanities Pedadogy. Practices,
Principles and Politics}, hg. von Brett D. Hirsch, 215--225.
{[}Cambridge{]}: Open Book.

\bibitem[\citeproctext]{ref-Matthews-Schlinzig-2020-a}
Matthews-Schlinzig, Marie Isabel, Jörg Schuster, Gesa Steinbrink und
Jochen Strobel, Hrsg. 2020a. \emph{Handbuch Brief. Von der Frühen
Neuzeit bis zur Gegenwart}. Bd. 1. Berlin: De Gruyter.

\bibitem[\citeproctext]{ref-Matthews-Schlinzig-2020-b}
---------, Hrsg. 2020b. \emph{Handbuch Brief. Von der Frühen Neuzeit bis
zur Gegenwart}. Bd. 2. Berlin: De Gruyter.

\bibitem[\citeproctext]{ref-McCarty-2005}
McCarty, Willard. 2005. \emph{Humanities Computing}. London: Palgrave.

\bibitem[\citeproctext]{ref-Mischke-2022}
Mischke, Dennis, Peer Trilcke und Henny Sluyter-Gäthje. 2022. Hackathons
als kollektiv-kreative Bildungsereignisse: Ein Konzept zur Gestaltung
offener Lehrveranstaltungen in den Digital Humanities. In:
\emph{Kulturen des digitalen Gedächtnisses}, hg. von Michaela Geierhos,
131--134. Potsdam: {[}Digital Humanities im deutschsprachigen Raum
e.V.{]}.

\bibitem[\citeproctext]{ref-Moerth-2021}
Moerth, Martina, Hiltraut Paridon und Ulrike Sonntag. 2021.
Kognitionswissenschaftliche Erkenntnisse und ihre Folgerungen für
evidenzbasierte Hochschullehre. \emph{Die Hochschullehre} 7, Nr. 5:
38--48.

\bibitem[\citeproctext]{ref-Otto-2015}
Otto, Barbara, Franziska Perels und Bernhard Schmitz. 2015.
Selbstreguliertes Lernen. In: \emph{Empirische Bildungsforschung.
Gegenstandsbereiche}, hg. von Heinz Reinders, Hartmut Ditton, Cornelia
Gräsel, und Burkhard Gniewosz, 41--53. 2. Aufl. Wiesbaden: Springer VS.

\bibitem[\citeproctext]{ref-Plachta-2013}
Plachta, Bodo. 2013. \emph{Editionswissenschaft: Eine Einführung in
Methode und Praxis der Edition neuerer Texte}. 2. Aufl. Stuttgart:
Reclam.

\bibitem[\citeproctext]{ref-Plachta-2020}
---------. 2020. \emph{Editionswissenschaft: Handbuch zu Geschichte,
Methode und Praxis der neugermanistischen Edition}. Stuttgart:
Hiersemann.

\bibitem[\citeproctext]{ref-Rapp-2017}
Rapp, Andrea. 2017. Manuelle und automatische Annotation. In:
\emph{Digital Humanities}, hg. von Fotis Jannidis, Hubertus Kohle, und
Malte Rehbein, 253--267. Stuttgart: Metzler.

\bibitem[\citeproctext]{ref-Rehbein-2012}
Rehbein, Malte und Christiane Fritze. 2012. Hands-On Teaching Digital
Humanities: A Didactic Analysis of a Summer School Course on Digital
Editing. In: \emph{Digital Humanities Pedadogy. Practices, Principles
and Politics}, hg. von Brett D. Hirsch, 47--57. {[}Cambridge{]}: Open
Book.

\bibitem[\citeproctext]{ref-Sahle-2013}
Sahle, Patrick. 2013. \emph{Digitale Editionsformen. Zum Umgang mit der
Überlieferung unter den Bedingungen des Medienwandels}. Bd. 2. Schriften
des Instituts für Dokumentologie und Editorik~8. Norderstedt: Book on
Demand. \url{http://nbn-resolving.de/urn:nbn:de:hbz:38-53523}.

\bibitem[\citeproctext]{ref-Sahle-2015}
---------. 2015. Digital Humanities? Gibt's doch gar nicht!
\emph{Zeitschrift für digitale Geisteswissenschaften} Sonderband~1.

\bibitem[\citeproctext]{ref-Schmid-1995}
Schmid, Helmut. 1995. Improvements in Part-of-Speech Tagging with an
Application to German. In: \emph{Proceedings of the ACL
SIGDAT-Workshop}, 1--9. Dublin: Association for Computational
Linguistics.
\url{https://cis.uni-muenchen.de/~schmid/tools/TreeTagger/data/tree-tagger2.pdf}.

\bibitem[\citeproctext]{ref-Strobel-2019}
Strobel, Jochen. 2019. Performanz in der Briefkommunikation und ihre
editorische Repräsentation. \emph{Editio} 33: 129--140.

\bibitem[\citeproctext]{ref-Trilcke-2018}
Trilcke, Peer und Frank Fischer. 2018. Literaturwissenschaft als
Hackathon: Zur Praxeologie der Digital Literary Studies und ihren
epistemischen Dingen. In: \emph{Wie Digitalität die
Geisteswissenschaften verändert. Neue Forschungsgegenstände und
Methoden}, hg. von Martin Huber und Sybille Krämer. Sonderbände der
Zeitschrift für digitale Geisteswissenschaften~3. Wolfenbüttel: Herzog
August Bibliothek.

\bibitem[\citeproctext]{ref-Valla-1994}
Valla, Laurentius. 1994. \emph{De falso credita et ementita constantini
donatione declamatio}. Hg. von Walther Schwahn. Stuttgart: Teubner.

\bibitem[\citeproctext]{ref-Walsh-2023}
Walsh, Brandon. 2023. The Three-Speed Problem in Digital Humanities
Pedagogy. In: \emph{What We Teach When We Teach DH. Digital Humanities
in the Classroom}, hg. von Brian Croxall und Diane K. Jakacki. Debates
in the Digital Humanities~10. Minneapolis: University of Minnesota
Press.

\bibitem[\citeproctext]{ref-Zeller-2002}
Zeller, Hans. 2002. Authentizität in der Briefedition: Integrale
Darstellung nichtsprachlicher Informationen des Originals. \emph{Editio}
16: 36--56.

\bibitem[\citeproctext]{ref-Zimmermann-2000}
Zimmermann, Barry J. 2000. Attaining Self-Regulation: A Social Cognitive
Perspective. In: \emph{Handbook of Self-Regulation}, hg. von Monique
Boekaerts, Paul R. Pintrich, und Moche Zeidner, 13--39. San Diego:
Academic Press.

\end{CSLReferences}




\end{document}
