% Options for packages loaded elsewhere
\PassOptionsToPackage{unicode}{hyperref}
\PassOptionsToPackage{hyphens}{url}
\PassOptionsToPackage{dvipsnames,svgnames,x11names}{xcolor}
%
\documentclass[
          a4paper,
        ]{article}
\raggedbottom
\usepackage{amsmath,amssymb}
\usepackage{setspace}
\usepackage{iftex}
\RequireLuaTeX

\usepackage{lscape} % für Querformat-Seiten
\usepackage{hyperref}

\usepackage{unicode-math} % this also loads fontspec
\defaultfontfeatures{Scale=MatchLowercase}
\defaultfontfeatures[\rmfamily]{Ligatures=TeX,Scale=1}

  \usepackage[]{plex-otf}
  
  
  
  
  
  
  
\directlua{luaotfload.add_fallback
   ("fallbacks",
    {
      "STIX Two Math:;",
      "NotoColorEmoji:mode=harf;",
    }
   )}
\defaultfontfeatures{RawFeature={fallback=fallbacks}}

\usepackage{academicons}
\usepackage{fontawesome5}
\usepackage{ccicons}

  
\IfFileExists{upquote.sty}{\usepackage{upquote}}{}
\IfFileExists{microtype.sty}{% use microtype if available
  \usepackage[]{microtype}
  \UseMicrotypeSet[protrusion]{basicmath} % disable protrusion for tt fonts
}{}
\usepackage{lua-widow-control}
  \makeatletter
\@ifundefined{KOMAClassName}{% if non-KOMA class
  \IfFileExists{parskip.sty}{%
    \usepackage{parskip}
  }{% else
    \setlength{\parindent}{0pt}
    \setlength{\parskip}{6pt plus 2pt minus 1pt}}
}{% if KOMA class
  \KOMAoptions{parskip=half}}
\makeatother
  
  
\usepackage[dvipsnames,svgnames,x11names]{xcolor}

\definecolor{oa-orange}     {RGB} {246, 130, 18}
\definecolor{ft-red}        {RGB} {168, 0, 0}
\definecolor{orcid-green}   {RGB} {166, 206, 57}
\definecolor{fortext-green}   {RGB} {26, 73, 76}


\colorlet{oa-orange}        {black}

\colorlet{highlightcolor1}  {DarkOrange}
\colorlet{highlightcolor2}  {DarkBlue}
\colorlet{highlightcolor3}  {DarkGreen}
\colorlet{highlightcolor4}  {DarkMagenta}

\colorlet{filecolor}        {highlightcolor1}
\colorlet{linkcolor}        {fortext-green}
\colorlet{citecolor}        {fortext-green}
\colorlet{urlcolor}         {fortext-green}


\NewDocumentCommand \oalogo { }
{%
  \textcolor{oa-orange}{\sffamily%\bfseries%
  open \aiOpenAccess\ access}
}


\NewDocumentCommand \orcidlink { m }
{%
	\texorpdfstring
	{\href{https://orcid.org/#1}{\textcolor{orcid-green}{\raisebox{-.2ex}{\aiOrcid}}}}
	{https://orcid.org/#1}%
}

\NewDocumentCommand \emaillink { m }
{%
	\texorpdfstring
	{\href{mailto:#1}{\textcolor{black}{\raisebox{-.2ex}{\faEnvelopeOpen[regular]}}}}
	{mailto:#1}%
}

  \usepackage{geometry}
\geometry{
	paper=a4paper,
	top=25mm,
	bottom=20mm,
	right=30mm,
	left=30mm,
	footskip=10mm,
	% showframe,
}
  
  \usepackage{listingsutf8}
\lstset{
  language=python,
  basicstyle=\ttfamily\footnotesize,
  columns=fullflexible,
  xleftmargin=2em,
  xrightmargin=2em,
  % frame=single,
  breaklines=true,
  postbreak=\mbox{\textcolor{highlightcolor1}{\(\hookrightarrow\)}\space},
}
\newcommand{\passthrough}[1]{#1}
\lstset{defaultdialect=[5.3]Lua}
\lstset{defaultdialect=[x86masm]Assembler}
  
  
  
\usepackage{longtable,booktabs,array}
\usepackage{tabularray}
\usepackage{ifthen}

  
\usepackage{calc} % for calculating minipage widths
% Correct order of tables after \paragraph or \subparagraph
\usepackage{etoolbox}
\makeatletter
\patchcmd\longtable{\par}{\if@noskipsec\mbox{}\fi\par}{}{}
\makeatother
% Allow footnotes in longtable head/foot
\IfFileExists{footnotehyper.sty}{\usepackage{footnotehyper}}{\usepackage{footnote}}
\makesavenoteenv{longtable}
\usepackage{graphicx}
\usepackage[export]{adjustbox}
\makeatletter
\def\maxwidth{\ifdim\Gin@nat@width>\linewidth\linewidth\else\Gin@nat@width\fi}
\def\maxheight{\ifdim\Gin@nat@height>\textheight\textheight\else\Gin@nat@height\fi}
\makeatother
\setkeys{Gin}{width=.8\maxwidth,height=\maxheight,keepaspectratio}
\makeatletter
\def\fps@figure{hbp}
\makeatother

  
\usepackage{luacolor}
\usepackage[soul]{lua-ul}

\setlength{\emergencystretch}{3em} % prevent overfull lines
\providecommand{\tightlist}{%
  \setlength{\itemsep}{0pt}\setlength{\parskip}{0pt}}

  \setcounter{secnumdepth}{-\maxdimen} % remove section numbering
  
  
  
\usepackage{fancyhdr}

\fancypagestyle{plain}{% used for the first page
	\fancyhf{}% clear all header and footer fields
	\fancyhead[L]{}
	\fancyfoot[L]{}
	\fancyfoot[R]{\small\thepage}
	\renewcommand{\headrulewidth}{0pt}%
	\renewcommand{\footrulewidth}{0pt}%
}

\fancypagestyle{page}{%
	\fancyhf{}% clear all header and footer fields
	\fancyhead[L]{}
	\fancyhead[R]{\small Guhr, \textit{Selbststudieneinheit „Textannotion
mit CATMA``}}
	\fancyfoot[L]{\small\textit{forTEXT} 2(13): Textannotation in der
Hochschullehre}
	\fancyfoot[R]{\thepage}
	\renewcommand{\headrulewidth}{0pt}%
	\renewcommand{\footrulewidth}{0pt}%
}

\pagestyle{page}


\usepackage[
	small,
	sf,bf,
	raggedright,
	clearempty,
]{titlesec}

\titleformat{\section}{\Large\sffamily\bfseries}{\thesection.}{.5em}{}
\titleformat{\subsection}{\large\sffamily\bfseries}{\thesubsection}{.5em}{}
\titleformat{\subsubsection}{\normalsize\sffamily\bfseries}{\thesubsubsection}{.5em}{}
\titleformat{\paragraph}{\normalsize\sffamily\bfseries}{\thesubsubsection}{.5em}{}
\titleformat{\subparagraph}{\normalsize\sffamily\bfseries}{\thesubsubsection}{.5em}{}

  \NewDocumentCommand\citeproctext{}{}
\NewDocumentCommand\citeproc{mm}{%
  \begingroup\def\citeproctext{#2}\cite{#1}\endgroup}
\makeatletter
 \let\@cite@ofmt\@firstofone
 \def\@biblabel#1{}
 \def\@cite#1#2{{#1\if@tempswa , #2\fi}}
\makeatother
\newlength{\cslhangindent}
\setlength{\cslhangindent}{1.5em}
\newlength{\csllabelwidth}
\setlength{\csllabelwidth}{3em}
\newenvironment{CSLReferences}[2] % #1 hanging-indent, #2 entry-spacing
 {\begin{list}{}{%
  \setlength{\itemindent}{0pt}
  \setlength{\leftmargin}{0pt}
  \setlength{\parsep}{0pt}
  % turn on hanging indent if param 1 is 1
  \ifodd #1
   \setlength{\leftmargin}{\cslhangindent}
   \setlength{\itemindent}{-1\cslhangindent}
  \fi
  % set entry spacing
  \setlength{\itemsep}{#2\baselineskip}}}
 {\end{list}}
\usepackage{calc}
\newcommand{\CSLBlock}[1]{\hfill\break\parbox[t]{\linewidth}{\strut\ignorespaces#1\strut}}
\newcommand{\CSLLeftMargin}[1]{\parbox[t]{\csllabelwidth}{\strut#1\strut}}
\newcommand{\CSLRightInline}[1]{\parbox[t]{\linewidth - \csllabelwidth}{\strut#1\strut}}
\newcommand{\CSLIndent}[1]{\hspace{\cslhangindent}#1}
  
  \usepackage[bidi=basic]{babel}
    \babelprovide[main,import]{ngerman}
                  \let\LanguageShortHands\languageshorthands
\def\languageshorthands#1{}
  
\usepackage[
	ragged,
	bottom,
	norule,
	multiple,
]{footmisc}

\makeatletter
\RenewDocumentCommand \footnotemargin { } {0em}
\RenewDocumentCommand \thefootnote { } {\arabic{footnote}}
\RenewDocumentCommand \@makefntext { m } {\noindent{\@thefnmark}. #1}
\interfootnotelinepenalty=10000
\makeatother

\usepackage{changepage}
% \newlength{\overhang}
% \setlength{\overhang}{\marginparwidth}
% \addtolength{\overhang}{\marginparsep}

  \usepackage{caption}
\captionsetup{labelformat=empty,font={small,it}}
  
\usepackage{selnolig}

  
  
  
  
\usepackage{csquotes}
\usepackage{bookmark}
\IfFileExists{xurl.sty}{\usepackage{xurl}}{} % add URL line breaks if available
\urlstyle{same}

  
  \usepackage[inkscapelatex=false]{svg}

\makeatletter
\def\@maketitle{%
	%
	% title
	%
	\newpage
	\null
	\vspace*{-\topskip}
	\begin{tblr}{
    vline{1,5} = {1pt},
    colspec={lllX},
    width=\textwidth,
    columns={font=\small\sffamily,},
    column{1,3}={
      rightsep=.3em,
      font=\footnotesize\sffamily,
    },
}
\hline[1pt]
\SetCell[c=4]{t,l}{\normalsize\textbf{Selbststudieneinheit „Textannotion
mit CATMA``}} & & & \\
& & & \SetCell[r=3]{c,r}{\includegraphics[height=2\baselineskip]{}} \\
\SetCell[c=3]{t,l}{\small Svenja Guhr \orcidlink{0000-0002-7686-3609}
\textsuperscript{\scriptsize 1}
} & & &\\
\SetCell[c=2]{t,l}{% 
\footnotesize%
1. Technische Universität Darmstadt \\
} & & & \\
		\hline[1pt]
     Titel der Ausgabe: & Textannotation in der Hochschullehre & & \\ 
     Jahrgang: & 2 & & \\ 
     Ausgabe: & 13 & & \\ 
    
    
    
    
    Lizenz: & \faCreativeCommons\ \faCreativeCommonsBy\ \faCreativeCommonsSa\ & & \oalogo \\

		\hline[1pt]
	\end{tblr}
}
\makeatother

\hypersetup{
      pdftitle={Selbststudieneinheit „Textannotion mit CATMA``},
      pdfauthor={Svenja Guhr},
      pdflang={de},
      pdfsubject={forTEXT 2(13): Textannotation in der Hochschullehre},
      pdfkeywords={Selbststudieneinheit, Blended
Learning, CATMA, Manuelle Annotation, Literaturwissenschaft},
      colorlinks=true,
  linkcolor={linkcolor},
  filecolor={filecolor},
  citecolor={citecolor},
  urlcolor={urlcolor},
  pdfcreator={LuaLaTeX via pandoc}
}


  
  \usepackage{ragged2e}
\usepackage[section]{placeins}
% Manage float placement
\usepackage{float}
\floatplacement{figure}{H}

\usepackage{marginnote}
\RenewDocumentCommand \marginfont { }
{ \sffamily\footnotesize }
\RenewDocumentCommand \raggedleftmarginnote { } { }
% \author{true}

% \date{}

\usepackage{everypage}
\usepackage[printwatermark]{xwatermark}
\usepackage{lipsum}


\begin{document}



\pagestyle{plain}


\maketitle



% \marginnote{\RaggedRight}[30\baselineskip]%









\setstretch{1.2}




\pagestyle{page}

\renewcommand{\arraystretch}{3}  % Zeilenabstand in Tabellen erhöhen

\section{Einführungstext}\label{einfuxfchrungstext}

\subsection{1.1. Rahmenbedingungen}\label{rahmenbedingungen}

Das vorliegende Lehrkonzept stellt eine vierteilige Selbststudieneinheit
zum Thema „Textannotation (mit CATMA)`` vor. Es handelt sich um eine
Kombination aus asynchroner und synchroner Lehre\footnote{Für mehr
  Informationen zu Blended Learning-Methoden, siehe
  (\citeproc{ref-graham_framework_2013}{Graham, Woodfield und Harrison
  2013}).}, die über einen Zeitraum von vier Semesterwochen im Rahmen
der Einführungsveranstaltung „Grundkurs Literaturwissenschaft 2`` im
Sommersemester an der Technischen Universität Darmstadt durchgeführt
wird. Der Grundkurs 2 ist der zweite Teil des zweiteiligen
Einführungsmoduls zur Einführung in die Literaturwissenschaft, das im
Germanistikstudium in den ersten zwei Studiensemestern die Grundlagen
der deutschsprachigen Literaturwissenschaft abdecken soll, um die
Studierenden auf das literaturwissenschaftliche Arbeiten in den darauf
aufbauenden Pro- und Hauptseminaren der höheren Semester vorzubereiten
(\citeproc{ref-modulhandbuch_2019}{Institut für Sprach- und
Literaturwissenschaft 2019}).

Die Selbststudieneinheit repräsentiert eine Schnittstelle zwischen
Geisteswissenschaften und Digital Humanities. Sie integriert digitale
Methoden in die literaturwissenschaftliche Forschung und Lehre, indem
sie den Studierenden die Nutzung von CATMA\footnote{CATMA steht für
  Computer Assisted Text Markup and Analysis, siehe
  {[}gius\_catma\_2016{]}.} als Annotations- und Analysetool nahebringt.
Diese Interdisziplinarität ermöglicht es, traditionelle
geisteswissenschaftliche Fragestellungen durch den Einsatz von
Technologien zu vertiefen und zu erweitern, und fördert gleichzeitig die
digitalen Kompetenzen der Studierenden.

Die Selbststudieneinheit deckt zeitlich 1/5 der Semesterwochen (4
Wochen) ab und ist vor allem im Sommersemester zur thematischen
Abdeckung der Wochen mit Feiertagsausfällen synchroner Sitzungen
praktisch im Semesterplan zu integrieren. Geprüft wird der erfolgreiche
Abschluss der Lehrveranstaltung durch eine 90-minütige schriftliche
Klausur am Ende der Vorlesungszeit. Nach erfolgreichem Bestehen der
Abschlussklausur erhalten die Teilnehmenden 5 ECTS-Punkte.

Der Grundkurs 2 und die Selbststudieneinheit wurden in dieser Form
bereits dreimal durchgeführt (im Sommersemester 2022, 2023 und 2024 an
der TU Darmstadt). Der Kurs wird als synchrone Lehrveranstaltung in
Lehrform eines Grundkurses mit zwei Semesterwochenstunden für eine
Gruppe von ca. 20-30 Studierenden (Germanistik Bachelor / Deutsch
Lehramt an Gymnasien) angeboten, die i.d.R. aber nicht zwingend bereits
den ersten Teil des Grundkurses besucht haben und somit erste
grundlegende Erfahrungen in Literaturwissenschaft mitbringen.

Die vorgesehenen Lerninhalte des Grundkurses als Rahmenlehrveranstaltung
sind laut Modulbeschreibung die „Einführung in erweiterte Gebiete der
Literaturwissenschaft. Studierende sollen am Ende des Kurses mit Themen
der Narrationstheorie, der Literaturgeschichte und der
Editionswissenschaft sowie mit den entsprechenden Theorien und Konzepten
vertraut sein und diese unter Anleitung kritisch einordnen und
diskutieren können'' (\citeproc{ref-modulhandbuch_2019}{Institut für
Sprach- und Literaturwissenschaft 2019}). Die unter den
Literaturwissenschaftsdozent:innen abgestimmten großen Themen der
Lehrveranstaltung Grundkurs Literaturwissenschaft 2 umfassen zwei
Sitzungen zu Literaturtheorien, zwei Sitzungen zur Literaturgeschichte
(19. und 20. Jahrhundert), vier Sitzungen zur Großgattung Prosa, drei
Sitzungen zur Großgattung Lyrik sowie drei Sitzungen zu
Organisatorischem, Klausurvorbereitung und Klausurdurchführung. Eine
erfolgreiche Teilnahme am Kurs befähigt die Studierenden zum Umgang mit
Begriffen und Konzepten erweiterter Gebiete der Literaturwissenschaft.
Sie können Analysen mittels wichtiger Methoden des jeweiligen
Teilgebiets durchführen. Darüber hinaus haben sie ein grundlegendes
Verständnis der Literaturwissenschaft und ihrer Unterdisziplinen erlangt
und sind mit den Grundlagen der literaturwissenschaftlichen Analyse, dem
analytischen Lesen und dem wissenschaftlichen Arbeiten vertraut.

\subsection{1.2. Voraussetzungen der
Teilnehmenden}\label{voraussetzungen-der-teilnehmenden}

Als Ausstattung zur Durchführung der Selbststudieneinheit müssen die
Studierenden Zugang zu einem internetverbundenen Laptop haben und einen
der gängigen Browser verwenden können (z.B. Firefox, Chrome oder Safari)
sowie grundlegende Sprachkenntnisse im Englischen mitbringen, um das
Optical User Interface der Textannotationssoftware CATMA verstehen und
nutzen zu können. Die benötigten technischen Vorkenntnisse erarbeiten
sich die Studierenden als Teil der Selbststudieneinheit, wodurch weitere
vorbereitende technische Kenntnisse nicht über die Verwendung eines
Internetbrowsers hinausgehen. Als fachliche Vorkenntnis wird ein
generelles Verständnis im Umgang mit literarischen Texten und
literarischer Erzähltextanalyse vorausgesetzt, das dem Niveau des
Oberstufendeutschunterrichts bzw. darauf aufbauend ggf. dem Besuch des
Grundkurses 1 Literaturwissenschaft an der Technischen Universität
Darmstadt entspricht.

\subsection{1.3 Ausführung der
Selbststudieneinheit}\label{ausfuxfchrung-der-selbststudieneinheit}

Die Selbststudieneinheit erstreckt sich über vier Semesterwochen:

\begin{enumerate}
\def\labelenumi{\arabic{enumi}.}
\tightlist
\item
  eine vorbereitende synchrone Sitzung zur Einführung in die
  Erzähltextanalyse,
\item
  eine asynchrone Lernphase von zwei Wochen, in der die Studierenden die
  Videotutorials und die vorbereitende Lektüre (Primär- und
  Sekundärliteratur) durcharbeiten,
\item
  eine synchrone Sitzung zur Einführung in die
  literaturwissenschaftliche Annotationspraxis mit Zeit für Fragen und
  für die Anwendungsaufgabe der Verwendung von CATMA für die Annotation
  und anschließende Figurenanalyse des Primärtexts,
\item
  eine Aufgabe zur manuellen Textannotation in der Abschlussklausur zur
  Überprüfung des Erreichens der Lernziele der Selbststudieneinheit.
\end{enumerate}

Für die synchronen Veranstaltungssitzungen benötigt die/der Lehrende
einen internetverbundenen Laptop sowie einen Beamer. Die asynchronen
Anteile wie Lektüre und Links zu den Videotutorials der
Selbststudieneinheit werden über den begleitenden Moodlekurs zur
Verfügung gestellt.

Zur Vermittlung von Kompetenzen wurden verschiedene Medien eingesetzt:\\
In der asynchronen Selbststudieneinheit haben die Studierenden Zugang zu
Videotutorials. Diese Tutorials bieten eine Einführung in die manuelle
Annotation mit dem digitalen Tool CATMA
(\citeproc{ref-fortext_tutorial_2019}{forTEXT 2019a};
\citeproc{ref-fortext_tutorial_2020}{forTEXT 2020a};
\citeproc{ref-fortext_tutorial_2019-1}{forTEXT 2019b};
\citeproc{ref-fortext_tutorial_2019-2}{forTEXT 2019c};
\citeproc{ref-fortext_tutorial_2020-1}{forTEXT 2020b}). Zudem wurden
vorbereitend drei Einführungstexte zum manuellen und kollaborativen
Annotieren (analog und digital mit CATMA) zur Verfügung gestellt
(\citeproc{ref-schumacherToolbeitragCATMA2019}{Schumacher 2024};
\citeproc{ref-jackeMethodenbeitragManuelleAnnotation2018}{Jacke 2024a};
\citeproc{ref-jackeMethodenbeitragKollaborativesLiteraturwissenschaftliches2018}{Jacke
2024b}). Während der synchronen Lehrveranstaltungen wurde als kurzer
Primärtext die Erzählung \emph{Krambambuli} von Ebner-Eschenbach
(\citeproc{ref-von_ebner-eschenbach_marie_1896}{1896}) diskutiert und
die Annotationsaufgabe für die Verwendung von CATMA literaturtheoretisch
und methodisch eingebettet mit Fokus auf literarische Erzähltext- und
Figurenanalyse (\citeproc{ref-krogh_figuren_2016}{Hansen 2016}).

Zur Unterstützung der Studierenden wurde in der dritten Iteration dieser
Selbststudieneinheit ein:e Tutor:in eingesetzt. Diese:r stand während
der asynchronen Selbststudieneinheit in einem Moodleforum für Fragen und
Hilfestellungen zur Verfügung. Während der synchronen Sitzungen half
er/sie bei technischen Problemen mit dem Annotationstool CATMA und
unterstützte die Studierenden bei der Durchführung der
Annotationsaufgabe.

In der Abschlussklausur ist eine Aufgabe zur manuellen Textannotation
vorgesehen, um das Erreichen der Lernziele der Selbststudieneinheit zu
überprüfen (siehe (\citeproc{ref-Abschnitt_4}{\textbf{Abschnitt\_4?}})).

\section{Beschreibung des
Gesamtablaufs}\label{beschreibung-des-gesamtablaufs}

Die Selbststudieneinheit verortet sich im Semesterplan als Teil des
Themas „Einführung in die Gattungstheorie: Prosa`` und baut auf den
Inhalten der Sitzungen zu Literaturgeschichte des 19. Jahrhunderts und
Literaturtheorie auf. Sie besteht aus vier Teilen: 1. Der erste Teil
beinhaltet eine theoretische Einführung in die Analyse von Figuren und
in das Formulieren von literaturwissenschaftlichen Fragestellungen in
einer synchronen Lehrveranstaltungssitzung. 2. Im zweiten Teil findet
eine asynchrone Lerneinheit über zwei Wochen statt, die gerade im
Sommersemester z.B. praktisch als Feiertagsüberbrückung verwendet werden
kann. In dieser Phase erarbeiten die Studierenden zusammengestellte
Materialien zur Einarbeitung in die literaturwissenschaftliche
Textannotation. Diese Materialien umfassen Einführungstexte ins manuelle
und kollaborative Annotieren (analog und digital mit CATMA) sowie kurze
Video-Tutorials zur Einführung in die manuelle Annotation mit dem
digitalen Annotationstool CATMA und den Primärtext \emph{Krambambuli}
von Ebner-Eschenbach
(\citeproc{ref-von_ebner-eschenbach_marie_1896}{1896}). 3. Der dritte
Teil besteht aus einer synchronen Lehrveranstaltung zur Einführung in
die literaturwissenschaftliche Annotationspraxis mit Zeit für Fragen und
für die Anwendungsaufgabe der Verwendung von CATMA für die Annotation
und anschließende Figurenanalyse des Primärtexts. 4. Im vierten und
letzten Teil wird das Gelernte durch eine Prüfungsaufgabe in der Klausur
überprüft, die sich auf die literaturwissenschaftliche Textannotation
fokussiert.

Die Selbststudieneinheit zielt darauf ab, die Lerninhalte Großgattung
Prosa und Narrationstheorie, die Erzählung \emph{Krambambuli} von Marie
von Ebner-Eschenbach sowie die Einführung und Anwendung der
literaturwissenschaftlichen Methode des Annotierens (manuell analog und
manuell digital mit dem Annotationstool CATMA) zu vermitteln. Zudem
beinhaltet die Einheit die Einzeltextanalyse der Primärlektüre mit einem
besonderen Fokus auf Figurenanalyse.

Die Qualifikationsziele und Lernergebnisse umfassen die Vertiefung der
Kenntnisse zur Narrationstheorie und ihrer Methodenanwendung. Die
Studierenden sollen nach Abschluss der Selbststudieneinheit in der Lage
sein, Methoden der Narrationstheorie auf bekannte literarische
Primärtexte des 19. Jahrhunderts anzuwenden und drei
literaturwissenschaftliche Annotationsmethoden durchzuführen:
manuell-analoges Annotieren literarischer Texte, manuell-digitales
Annotieren literarischer Texte und kollaborativ manuell-digitales
Annotieren literarischer Texte. Des Weiteren sollen die Studierenden die
Nützlichkeit literaturwissenschaftlicher Textannotation reflektieren und
lernen, situationsabhängig zu entscheiden, wie diese Methode am
sinnvollsten anzuwenden ist, z.B. indem sie entdecken, wie
Annotationstagsets entwickelt und sinnvoll eingesetzt werden, um sich
einer ausgewählten Forschungsfrage zu nähern. Schließlich sollen sie
literaturwissenschaftliche Forschungsfragen formulieren und sich der
Beantwortung dieser Fragen mit Hilfe literaturwissenschaftlicher
Textannotation annähern können.

\section{Detaillierte Darstellung der Sitzungen bzw. Einheiten zum Thema
Textannotation}\label{detaillierte-darstellung-der-sitzungen-bzw.-einheiten-zum-thema-textannotation}

\subsection{Sitzung 1: Theoretische Einführung in die Analyse von
Figuren}\label{sitzung-1-theoretische-einfuxfchrung-in-die-analyse-von-figuren}

In der ersten Sitzung erhalten die Studierenden eine theoretische
Einführung in die Analyse von Figuren (aufbauend auf
\citeproc{ref-krogh_figuren_2016}{Hansen 2016}) sowie in das Formulieren
von literaturwissenschaftlichen Fragestellungen. Diese synchrone
Lehrveranstaltung beginnt mit einem Vortrag der Lehrperson (siehe
(\citeproc{ref-Foliensatz_Figuren}{\textbf{Foliensatz\_Figuren?}})), in
dem die grundlegenden Konzepte und Methoden der Figurenanalyse behandelt
werden. Zunächst wird die Bedeutung und Funktion von Figuren in
literarischen Texten erläutert. Danach werden die drei funktionalen
Dimensionen nach (\citeproc{ref-Phelan_2005}{\textbf{Phelan\_2005?}}),
die Unterscheidung von flachen und runden Charakteren nach
(\citeproc{ref-Forster_1927}{\textbf{Forster\_1927?}}) und das
Figurenmodell nach (\citeproc{ref-Hansen_2000}{\textbf{Hansen\_2000?}})
mit seiner Unterscheidung von „showing`` und „telling`` präsentiert. Der
Inputvortrag wird abgeschlossen mit einer Diskussion, in der das Gehörte
durch die Studierenden angewendet werden soll, indem sie diskutieren,
welche Figuren im vorbereiteten Primärtext Krambambuli von
Ebner-Eschenbach (\citeproc{ref-von_ebner-eschenbach_marie_1896}{1896})
vorkommen und wie diese hinsichtlich der vorgestellten Figurenmodelle
eingeordnet werden können.

Im Anschluss an den theoretischen Input wird das Formulieren von
literaturwissenschaftlichen Forschungsfragen thematisiert. Die
Studierenden lernen, wie sie präzise und relevante Fragen zur
Figurenanalyse entwickeln können. Die Sitzung umfasst Diskussionen und
Gruppenarbeit, bei der die Studierenden anhand eines kurzen
Textausschnitts der Primärlektüre Forschungsfragen formulieren, denen
sich mit Figurenanalyse genähert werden kann.

Ziel dieser Sitzung ist es, ein Verständnis der grundlegenden Konzepte
der Figurenanalyse zu vermitteln, die Fähigkeit zur Unterscheidung und
Anwendung verschiedener Ansätze der Figurenanalyse zu entwickeln und die
Kompetenz zu fördern, literaturwissenschaftliche Forschungsfragen zu
formulieren. Im Rahmen des Grundkurses 2 Einführung in die
Literaturwissenschaft an der TU Darmstadt wurde diese Sitzung als Teil
der umfassenderen Lehrveranstaltung zur Einführung in die
Gattungstheorie - Fokus Prosa durchgeführt. Dieser Sitzungsteil deckt
ca. 30min einer 90min Lehrveranstaltungssitzung ab.

\subsection{Sitzung 2: Selbststudium: Einführung in die
literaturwissenschaftliche Textannotation (asynchrone
Selbststudieneinheit über zwei
Wochen)}\label{sitzung-2-selbststudium-einfuxfchrung-in-die-literaturwissenschaftliche-textannotation-asynchrone-selbststudieneinheit-uxfcber-zwei-wochen}

Die zweite „Sitzung`` besteht aus einer asynchronen
Selbststudieneinheit, welche sich über zwei Wochen erstreckt. Der
Abschnitt bietet sich daher beispielsweise als Feiertagsüberbrückung an.
In dieser Zeit haben die Studierenden die Gelegenheit, sich eigenständig
in die Grundlagen der literaturwissenschaftlichen Textannotation
einzuarbeiten. Sie setzen sich mit drei Einführungstexten auseinander,
die das manuelle und kollaborative Annotieren sowohl in analoger als
auch digitaler Form behandeln
(\citeproc{ref-schumacherToolbeitragCATMA2019}{Schumacher 2024};
\citeproc{ref-jackeMethodenbeitragManuelleAnnotation2018}{Jacke 2024a};
\citeproc{ref-jackeMethodenbeitragKollaborativesLiteraturwissenschaftliches2018}{Jacke
2024b}). Ergänzend dazu stehen Video-Tutorials zur Einführung in die
manuelle Annotation mit dem digitalen Annotationstool CATMA zur
Verfügung (\citeproc{ref-fortext_tutorial_2019}{forTEXT 2019a};
\citeproc{ref-fortext_tutorial_2020}{forTEXT 2020a};
\citeproc{ref-fortext_tutorial_2019-1}{forTEXT 2019b};
\citeproc{ref-fortext_tutorial_2019-2}{forTEXT 2019c};
\citeproc{ref-fortext_tutorial_2020-1}{forTEXT 2020b}). Diese
Materialien bieten den Studierenden eine umfassende Einführung in die
Annotationsmethoden und ermöglichen ihnen, das Gelernte praktisch
auszuprobieren.

Ziel dieser Einheit ist es, die Grundlagen des manuellen und digitalen
Annotierens zu vermitteln, die Fähigkeit zur Nutzung des Tools CATMA zu
entwickeln und ein Verständnis für die Vor- und Nachteile verschiedener
Annotationsmethoden zu schaffen. Die Studierenden lesen die
Einführungstexte, schauen sich die Video-Tutorials an und führen erste
praktische Übungen zur Annotation mit CATMA durch.

\subsection{Sitzung 3: Einführung in die literaturwissenschaftliche
Annotationspraxis (Synchrone Lehrveranstaltung, 90 bzw.
120min)}\label{sitzung-3-einfuxfchrung-in-die-literaturwissenschaftliche-annotationspraxis-synchrone-lehrveranstaltung-90-bzw.-120min}

In der dritten Sitzung steht die praktische Anwendung der
Annotationsmethoden im Fokus. Die Studierenden haben vorbereitend für
die Sitzung einen CATMA-Account eingerichtet, erste Anwendungsbeispiele
geübt und setzen die während der asynchronen Selbststudieneinheit
erworbenen Kenntnisse in einer synchronen Lehrveranstaltung in die
Praxis um. Begleitet wird die Sitzung vom
(\citeproc{ref-Foliensatz_Einfuxfchrung_Textannotation}{\textbf{Foliensatz\_Einführung\_Textannotation?}}).
Die Sitzung wurde für 90min konzipiert. Um den Studierenden mehr Zeit
für die Diskussionen und Anwendungsaufgaben zu geben, empfiehlt sich
eine Sitzung von ca. 120min. Ziel dieser Sitzung ist es, die Anwendung
der Annotationsmethoden in der Praxis zu vertiefen und die Fähigkeit zur
kritischen Reflexion von Textannotation zu fördern.

Die Sitzung beginnt mit der Begrüßung der Studierenden sowie einem
Brainstorming (5min) in Partnerarbeit zur Aktivierung des Gelernten aus
der zweiwöchigen Selbststudieneinheitsphase. Diese Aktivierung
unterstützt das Ankommen der Studierenden in der Sitzung, das warm
werden mit dem Thema sowie das ins Sprechen kommen durch die
Wiederholung des thematischen Vokabulars der Sitzung im geschützten Raum
der Partnerarbeit. Anschließend gibt es einen kurzen Austausch im Plenum
im Sinn einer reduzierten Form der „Think-Pair-Share`` Lernmethode von
Lyman (\citeproc{ref-lyman_responsive_1998}{1998}).

Es folgt ein Inputvortrag (8-10min) der Lehrperson zur Einführung in
Annotationen vom Generellen (Bildannotationen, Markup Languages,
Glossen) zum Spezifischen (Annotation als textwissenschaftliche Praktik
in der Literaturwissenschaft). Es werden drei verschiedene
Annotationsarten in der Sprach- und Literaturwissenschaft vorgestellt
und drei Textannotationsmethoden mit Beispielen unterlegt
gegenübergestellt.

Am Ende des Inputvortrags werden die Studierenden gebeten in
Einzelarbeit manuell loszuannotieren („Bitte annotieren Sie manuell
analog den ausgedruckten Textausschnitt aus Marie von Ebener-Eschenbachs
\emph{Krambambuli}``). Dazu wurde vor Sitzungsbeginn der Anfang des
Primärtexts ausgedruckt ausgeteilt. Die Studierenden bekommen dafür
5min. Das Ziel dieser Aufgabe ist es, verschiedene Annotationsideen
auszuprobieren, die während des Impulsvortrags aufgekommen sind. Der
Erfahrung nach variieren die Annotationen von Unterstreichungen von
Wortarten zu lexikalischen Wortfamilien, die Studierenden
unterstreichen, umkreisen, umkasten und markieren frei nach ihren
Vorstellungen. Nach 5min haben die Studierenden kurz die Möglichkeit,
dem Plenum vorzustellen, was sie wie annotiert haben. Dieses
Zusammenkommen im Plenum bildet den Übergang zur Diskussionsfrage:
„Braucht man Vorgaben zum Annotieren?``. Der Erfahrung nach sprechen die
Studierenden in der Diskussion ihre anfängliche Unsicherheit mit der
Annotationsaufgabe an, weil sie nicht wussten, was genau sie annotieren
sollten. Der nächste Punkt der Diskussion wird eingeleitet durch die
Reflexionsfragen: „Welche Vorgaben hätten Sie sich gewünscht?`` und
„Welche Vorgaben haben Sie sich selbst gegeben?``. Durch diese offenen
Fragen werden sich die Studierenden der Notwendigkeit von
Annotationsrichtlinien bewusst. Sie diskutieren ihre selbstdefinierten
Vorgaben, z.B. wann unterstrichen und wann umkreist wurde und welche
Farbe welche Annotationseinheit bedeutete. Die Lehrperson leitet sodann
über zum zweiten Impulsvortrag.

Im zweiten Inputvortrag (6-8min) stellt die Lehrperson den Nutzen und
Aufbau von Annotationsrichtlinien vor. Unterstützt durch die Abbildung
von Gius und Jacke (\citeproc{ref-gius_jacke_2016}{2016}), 7 wird die
Komplexität des Annotationsprozess verdeutlicht. Anschließend folgt eine
kurze Wiederholung zu Annotationskategorien und Tagsets, die
Verbildlichung von Textannotation als iterativer und zyklischer Prozess
(nach Rapp (\citeproc{ref-rapp_manuelle_2017}{2017}) und Zinsmeister und
Lemnitzer (\citeproc{ref-zinsmeister_korpuslinguistik_2015}{2015}))
sowie ein sehr kurzer Ausblick auf die Evaluation des
Annotationsprozesses sowie Goldannotationen. Nach einer kurzen
Unterbrechung für Fragen der Studierenden folgt der Anwendungsteil
„Annotation mit CATMA``.

Für den Anwendungsteil schlägt die Lehrperson eine Forschungsfrage,
Hypothese und Annotationsmethode vor und bringt ein CATMA-Gruppenprojekt
mit vorbereitetem Tagset mit, das die Studierenden anwenden können. Nach
einer kurzen Vorstellung der Materialien sowie dem Einwählen der
Teilnehmer:innen in das CATMA-Gruppenprojekt erhalten die Studierenden
den folgenden Arbeitsauftrag (ca. 20min):

\begin{quote}
Bitte annotieren Sie \textbf{manuell digital} den Anfang aus Marie von
Ebener-Eschenbachs \emph{Krambambuli}. 1. Erstellen Sie eine
\textbf{Annotation Collection} nach dem Schema:
„Name\_Annotation\_Collection``. 2. Öffnen Sie den Text im
\textbf{Annotationsmodus} („Annotate``). 3. Wählen Sie Ihre Annotation
Collection aus und \textbf{annotieren} Sie den Anfang von
\emph{Krambambuli} unter Anwendung des Tagsets zur Figurenannotation. 4.
\textbf{Synchronisieren} Sie Ihre Annotationen auf der Projektstartseite
(„SYNC`` → synchronize with the team).
\end{quote}

Während der Arbeitsphase steht die Lehrperson für Fragen und technische
Unterstützung zur Verfügung. Ggf. wird die Lehrperson dabei durch eine:n
Tutor:in vor Ort unterstützt. Die Arbeitsphase wird mit einer Reflexion
abgeschlossen, in der die Studierenden berichten, wie sie mit den
Annotationskategorien und den technischen Prozessen zurechtkamen,
aufführen, welche Tags ihnen gefehlt haben oder welche sie gar nicht
verwendet haben und schließlich auf der Grundlage ihrer
Annotationserfahrung argumentieren, wie die Forschungsfrage beantwortet
werden könnte.

Daran anschließend stellt die Lehrperson beispielhaft Abfragen,
Visualisierungen und halbautomatisierte Annotation mit CATMA vor, um
eine mögliche Annäherung an die Forschungsfrage mithilfe quantitativer
Annotationsauswertung (der vorbereiteten Annotationen) vorzustellen. Den
Abschluss der Sitzung bildet eine letzte Diskussion im Plenum mit der
Ausblickfrage, welche weiteren Forschungsfragen zum Primärtext und
darüber hinaus die Studierenden interessieren würden, der sie sich mit
einem CATMA-Annotationsprojekt nähern könnten.

\subsection{Sitzung 4: Überprüfung des Gelernten durch Klausuraufgabe
(ca. 15 bzw.
25min)}\label{sitzung-4-uxfcberpruxfcfung-des-gelernten-durch-klausuraufgabe-ca.-15-bzw.-25min}

In der vierten Sitzung wird das in den vorangegangenen Einheiten
Gelernte anhand einer Aufgabe in der Abschlussklausur angewendet. Im
Rahmen des Grundkurses 2 Einführung in die Literaturwissenschaft wurden
im Sommersemester 2022, 2023 und 2024 Variationen der folgenden zwei
Klausuraufgaben eingesetzt:

\subsubsection{Aufgabe 1:}\label{aufgabe-1}

Eine Klausuraufgabe mit Umfang von 5,5P. (ca. 15min Aufwand):
\textgreater{} a) Erklären Sie kurz die Methode der manuellen
Textannotation (1P.). \textgreater{} b) Entwickeln Sie ein Tagset
bestehend aus \textbf{min. drei verschiedenen Tags} für die Annotation
des Primärtextausschnitts. Nennen Sie dafür den Tagnamen und erläutern
Sie (kurz) Ihr Annotationsvorgehen (3,5P.: jeweils 0,5P. pro Nennung des
Namens (Tagset und 3 Tags) und jeweils 0,5P. pro Kurzerläuterung des
Annotationsvorgehens pro Tag, z.B. Unterstreichen, Tagfarbe, gewählte
Kategorie). \textgreater{} c) Annotieren Sie den Textausschnitt
systematisch mit Ihren neudefinierten Tags (1P.).

Die erste vorgeschlagene Klausuraufgabe sollte im Klausuraufbau einen
mittleren Platz einnehmen. Sie deckt die ersten drei Taxonomiestufen
kognitiver Lernziele nach Anderson und Krathwohl
(\citeproc{ref-anderson_taxonomy_2001}{2001}) ab: Einerseits wird
bereits das reine Nennen von Tagsetnamen und Tagnamen mit Punkten
vergütet (Taxonomiestufe 1), andererseits aber auch die Erklärung der
Methode gefordert sowie die Erläuterung des Annotationsvorgehens
(Taxonomiestufe 2) und die Anwendung der Annotationsmethode auf den
gegebenen Primärtextausschnitt (Taxonomiestufe 3).

\subsubsection{Aufgabe 2:}\label{aufgabe-2}

Eine Klausuraufgabe mit Umfang von 7P. (ca. 20-25min Aufwand):
\textgreater{} Lesen Sie den Primärtextausschnitt. \textgreater{} a)
Formulieren Sie eine Forschungsfrage hinsichtlich der im Kurs
behandelten Inhalte (z.B. aus dem Bereich zu Prosatheorie, Erzähltheorie
-- etwa Diskurs, Geschichte, Figuren -- oder Literaturgeschichte).
Beachten Sie bei der Formulierung die im Kurs besprochenen inhaltlichen
und formalen Anforderungen an eine Forschungsfrage (1 Satz = 1P.).
\textgreater{} b) Begründen Sie die (literaturwissenschaftliche)
Relevanz Ihrer Forschungsfrage kurz, indem Sie sie in den
literaturwissenschaftlichen Forschungskontext einordnen (1-2 Sätze,
2P.). \textgreater{} c) Entscheiden Sie, ob und wie
literaturwissenschaftliche Textannotation als Grundlage der Annäherung
an die Beantwortung Ihrer Forschungsfrage angewendet werden kann.
Begründen Sie Ihre Entscheidung und beschreiben Sie Ihre mögliche
Herangehensweise (3-5 Sätze, 4P.).

Die zweite vorgeschlagene Klausuraufgabe sollte im Klausuraufbau den
Platz einer der letzten oder der letzten Aufgabe einnehmen. Sie deckt
die höchsten Taxonomiestufen kognitiver Lernziele nach Anderson und
Krathwohl (\citeproc{ref-anderson_taxonomy_2001}{2001}) ab, indem eine
Forschungsfrage formuliert (Produktion, Taxonomiestufe 6), ihre Relevanz
begründet (Beurteilung, Taxonomiestufe 5) sowie für oder gegen die
Anwendung von Textannotation als Methode entschieden werden muss
(Entscheidung, Taxonomiestufe 5).

Anhand der erfolgreichen Bearbeitung der Klausuraufgabe beweisen die
Studierenden das Erreichen des Lernziels, theoretische Konzepte in
konkreten Analysekontexten anzuwenden, indem sie
literaturwissenschaftlich relevante Forschungsfragen formulieren und
Textannotation als Methode zur Annäherung an die selbstformulierte
Forschungsfrage verwenden sowie die Anwendung reflektieren.

\section{Reflexion des Lehrkonzepts: Gelungene Ansätze und
Herausforderungen}\label{reflexion-des-lehrkonzepts-gelungene-ansuxe4tze-und-herausforderungen}

\subsection{5.1. Rahmenbedingungen und Ausführung der
Veranstaltung}\label{rahmenbedingungen-und-ausfuxfchrung-der-veranstaltung}

Im Rahmen des Grundkurses Einführung in die Literaturwissenschaft 2
wurde die vorgestellte Selbststudieneinheit insgesamt drei Mal
durchgeführt (Sommersemester 2022, 2023 und 2024), wobei ihr Aufbau seit
der ersten Iteration nicht verändert wurde. Für eine erneute
Durchführung im Sommersemester 2025 wird jedoch eine Anpassung von CATMA
6 auf CATMA 7 notwendig sein.

Retrospektiv konnte ich feststellen, dass die Vierteilung der
Lehreinheit zur „Textannotation (mit CATMA)`` von den Studierenden gut
angenommen wurde. Vor allem das Angebot einer asynchronen
Selbststudieneinheit anstelle einer synchronen Sitzung wurde positiv
aufgenommen. Außerdem konnte so jeweils eine (z.B. gerade im
Sommersemester feiertagsbedingt) ausfallende Sitzung überbrückt werden.
In Bezug auf das Feedback der Studierenden wurde deutlich, dass sie die
Mischung aus synchronen und asynchronen Einheiten sowie die
praxisorientierte Anwendung der Theorie schätzten. Besonders die
Video-Tutorials fanden die Studierenden hilfreich, weil sie sich dadurch
die Inhalte in ihrem eigenen Tempo erarbeiten und wahlweise Teile des
Tutorials wiederholen konnten. Die Teilnehmenden meldeten außerdem
positiv zurück, dass sie das Gelernte aus der Selbststudieneinheit und
den Theorieinputvorträgen in der synchronen Sitzung (3) selbst anwenden
konnten und zum Weiterdenken angeregt wurden.

Die synchrone Sitzung (3) zur Einführung in die
literaturwissenschaftliche Annotationspraxis, das Herzstück des
vorgestellten Lehrkonzepts, verlief in allen drei Iterationen ohne
Probleme. Die technischen Voraussetzungen waren gegeben: Der Beamer und
das WLAN funktionierten, die Studierenden brachten internetverbundene
Laptops mit sowie eingerichtete CATMA-Accounts, sodass sie die
Anwendungsaufgaben durchführen konnten.

Den Erfolg der zweiwöchigen asynchronen Lernphase konnte ich überprüfen,
weil die Studierenden, die an der synchronen Sitzung teilnahmen, sie als
Vorbereitung durchgeführt haben mussten, um während der Sitzung gut
mitarbeiten zu können. Beim Umhergehen während der Bearbeitungszeit der
Anwendungsaufgaben konnte ich außerdem feststellen, ob die
grundsätzliche Verwendung von CATMA verstanden worden war. Ich schaute
auf die Bildschirme der Teilnehmenden und gab aktionales Feedback zu den
Arbeitsschritten, half bei Schwierigkeiten und beantwortete individuelle
Fragen.

Der Einsatz eine:r Tutor:in ab der dritten Iteration war eine wertvolle
Bereicherung der Unterstützung der Studierenden beim Umgang mit CATMA
und der individuellen Vorbereitung. Sollte bei Übernahme dieses
Lehrkonzepts in externe Veranstaltungskontexte keine Tutorunterstützung
möglich ist, könnte ein von den Studierenden selbstmoderiertes
Onlineforum (z.B. bei Moodle) eingesetzt werden, in dem Fragen während
der asynchronen Lernphase individuell gestellt und beantwortet werden
können (auch unterstützt durch die Lehrperson). Außerhalb des
Grundkurskontextes könnte zudem ein kürzerer Primärtext zur Annotation
zur Verfügung gestellt werden, damit die Teilnehmenden einen
vollständigen Text annotieren können und nicht nur einen kurzen
Ausschnitt. Dafür könnte zum Beispiel ein Märchen als Textgrundlage zum
Einsatz kommen. Ein kürzerer Text würde es den Teilnehmenden
ermöglichen, den gesamten Text zu erfassen und somit ein besseres
Verständnis für die Struktur und die narrativen Techniken zu entwickeln.

Die Klausuraufgabe erwies sich als sinnvolles Maß für das Erreichen der
Lernziele, wobei auffiel, dass auch Studierende, die nicht an der
synchronen Sitzung (3) teilgenommen hatten, sich durch das Nachbereiten
der bereitgestellten Materialien ausreichend für eine erfolgreiche
Bearbeitung der Klausuraufgabe vorbereiten konnten.

Insgesamt bin ich zufrieden mit dem Aufbau und allen drei Durchführungen
des Lehrkonzepts. Meine Bedenken, dass die geplanten Zeitabschnitte für
Arbeitsaufgaben und Plenumsdiskussionen zu eng getaktet sein könnten,
hatten sich nicht bestätigt. Dafür war es jedoch wichtig, die
Zeitplanung durchgängig im Blick zu haben und die Diskussionen so zu
moderieren, dass der vorgesehene Zeitrahmen eingehalten wurde.

Ich kann mir vorstellen, dass eine Verlängerung der synchronen Sitzung
(3) um 30min sinnvoll sein könnte, um den Studierenden mehr Zeit für
Annotation, Austausch und Diskussion zu geben oder eine Pause vor dem
zweiten Inputvortrag einzuplanen, damit die Studierenden für den zweiten
Teil der synchronen Veranstaltung mehr Energie und Aufmerksamkeit haben
sowie die gelernten Inhalte besser verarbeiten können. Je nach
Gruppengröße sowie bei erwartbar mehr technischen Hürden bei den
Teilnehmenden wäre auch eine Zweiteilung der synchronen Sitzung auf zwei
Sitzungen denkbar.

\subsection{5.2. Studierende}\label{studierende}

Mit Blick auf die tatsächlich teilnehmenden Studierenden habe ich in den
drei Iterationen der Lehreinheit (Sommersemester 2022, 2023 und 2024)
ganz unterschiedliche Erfahrungen gesammelt. Vor allem die Anzahl der
Teilnehmenden an der synchronen Sitzung (3) variierte: In den ersten
beiden Iterationen waren es jeweils 15 Teilnehmende, die sich aktiv in
den synchronen Sitzungen engagierten und in der Abschlussklausuraufgabe
zur Textannotation gute bis sehr gute Ergebnisse erzielten. Bei der
dritten Iteration waren nur vier Studierende in der zweiten synchronen
Sitzung anwesend, die sich dementsprechend aber auch sehr engagiert
beteiligten. Ich verpasste leider die Gelegenheit zu eruieren, ob es am
geringen Interesse für Textannotation oder der nicht ausreichend
vehementen Ankündigung von Textannotation als Klausurthema lag, dass so
wenige Studierende in der synchronen Sitzung anwesend waren (oder ganz
andere Gründe für das Fernbleiben vorlagen). Die Ergebnisse in der
Abschlussklausuraufgabe fielen jedoch überwiegend gut bis sehr gut aus,
was bedeutet, dass die zur Verfügung gestellten Materialien ausreichten,
um die Lernziele zu erreichen.

Alle Teilnehmenden waren wie geplant eingeschriebene Studierende im
Bachelor Germanistik oder Deutsch Lehramt an Gymnasien und brachten die
verlangten Vorkenntnisse und technischen Voraussetzungen mit, wobei bei
jeder Iteration immer min. ein:e Studierende:r dabei war, die/der
anstelle eines Laptops ein Tablet als internetfähiges Endgerät in die
Veranstaltung mitbrachte, auf dem zwar die Annotationen einsehbar aber
nicht selbst vorgenommen werden können. Diese Studierenden führten die
Anwendungsaufgabe in der synchronen Sitzung in Partnerarbeit durch.

Die Unterstützung durch eine:n Tutor:in erwies sich als sehr hilfreich.
Insbesondere bei technischen Fragen und individuellen Problemen konnte
der/die Tutor:in direkt vor Ort oder bereits vorher online im
Moodleforum Hilfestellung geben und die Studierenden bei der Anwendung
von CATMA unterstützen.

Zusammenfassend lässt sich sagen, dass die Selbststudieneinheit
„Textannotation (mit CATMA)`` in ihrer vorgestellten Form durch die
Kombination verschiedener Lehr- und Lernformate erfolgreich die
Studierenden unterstützt, die definierten Lernziele zu erreichen.

\phantomsection\label{refs}
\begin{CSLReferences}{1}{0}
\bibitem[\citeproctext]{ref-anderson_taxonomy_2001}
Anderson, Lorin W. und David R. Krathwohl, Hrsg. 2001. \emph{A taxonomy
for learning, teaching, and assessing: a revision of {Bloom}'s taxonomy
of educational objectives}. Complete ed. New York: Longman.

\bibitem[\citeproctext]{ref-von_ebner-eschenbach_marie_1896}
Ebner-Eschenbach, Marie von. 1896. \emph{Marie von {Ebner}-{Eschenbach}:
{Krambambuli}}.
\url{https://www.projekt-gutenberg.org/ebnresch/krambamb/krambamb.html}
(zugegriffen: 14. Oktober 2024).

\bibitem[\citeproctext]{ref-fortext_tutorial_2019}
forTEXT. 2019a. Tutorial: {CATMA} 6 zur manuellen {Annotation} nutzen.
Oktober. doi:
\href{https://doi.org/10.5281/zenodo.10353556}{10.5281/zenodo.10353556},
\url{https://zenodo.org/records/10353556}.

\bibitem[\citeproctext]{ref-fortext_tutorial_2019-1}
---------. 2019b. Tutorial: {Projektmanagement} in {CATMA} 6. November.
doi:
\href{https://doi.org/10.5281/zenodo.10353713}{10.5281/zenodo.10353713},
\url{https://zenodo.org/records/10353713}.

\bibitem[\citeproctext]{ref-fortext_tutorial_2019-2}
---------. 2019c. Tutorial: {Tagsets} in {CATMA} 6 anlegen. Dezember.
doi:
\href{https://doi.org/10.5281/zenodo.10377968}{10.5281/zenodo.10377968},
\url{https://zenodo.org/records/10377968}.

\bibitem[\citeproctext]{ref-fortext_tutorial_2020}
---------. 2020a. Tutorial: {In} {CATMA} 6 annotieren. Januar. doi:
\href{https://doi.org/10.5281/zenodo.10353910}{10.5281/zenodo.10353910},
\url{https://doi.org/10.5281/zenodo.10353910}.

\bibitem[\citeproctext]{ref-fortext_tutorial_2020-1}
---------. 2020b. Tutorial: {Analysieren} und visualisieren mit {CATMA}.
Februar. doi:
\href{https://doi.org/10.5281/zenodo.10276637}{10.5281/zenodo.10276637},
\url{https://doi.org/10.5281/zenodo.10276637}.

\bibitem[\citeproctext]{ref-gius_jacke_2016}
Gius, Evelyn und Janina Jacke. 2016. Zur {Annotation} narratologischer
{Kategorien} der {Zeit}. {Guidelines} zur {Nutzung} des
{CATMA}-{Tagsets}.
\url{http://heureclea.de/wp-content/uploads/2016/11/guidelinesV2.pdf}.

\bibitem[\citeproctext]{ref-graham_framework_2013}
Graham, Charles R., Wendy Woodfield und J. Buckley Harrison. 2013. A
framework for institutional adoption and implementation of blended
learning in higher education. \emph{The Internet and Higher Education}
18: 4--14. doi:
\href{https://doi.org/10.1016/j.iheduc.2012.09.003}{10.1016/j.iheduc.2012.09.003},
\url{https://linkinghub.elsevier.com/retrieve/pii/S1096751612000607}
(zugegriffen: 14. Oktober 2024).

\bibitem[\citeproctext]{ref-krogh_figuren_2016}
Hansen, Per Krogh. 2016. {IV}. 3.3 {Figuren}. In: \emph{Einführung in
die {Erzähltextanalyse}}, hg. von Silke Lahn und Jan Christoph Meister,
übers. von Marie Isabel Schlinzig, 234--249. 3. Aufl. Lehrbuch.
Stuttgart: J. B. Metzler Verlag.
\url{https://doi.org/10.1007/978-3-476-05415-9} (zugegriffen: 27. Januar
2021).

\bibitem[\citeproctext]{ref-modulhandbuch_2019}
Institut für Sprach- und Literaturwissenschaft. 2019. Modulhandbuch
{Germanistik} {J}.{B}.{A}. Technische Universität Darmstadt.
\url{https://www.tu-darmstadt.de/media/daa_responsives_design/02_studium_medien/01_studieninteressierte_medien/02_studienangebot_medien/joint_bachelor_of_arts_2/germanistik_1/modulhandbuch_7/MHB-JBA-Germanistik-2019.pdf}
(zugegriffen: 14. Oktober 2024).

\bibitem[\citeproctext]{ref-jackeMethodenbeitragKollaborativesLiteraturwissenschaftliches2018}
Jacke, Janina. 2024b. Methodenbeitrag: {Kollaboratives}
literaturwissenschaftliches {Annotieren}. \emph{forTEXT Heft} 1, Nr. 4
(August). doi:
\href{https://doi.org/10.48694/fortext.3749}{10.48694/fortext.3749},
\url{https://fortext.net/routinen/methoden/kollaboratives-literaturwissenschaftliches-annotieren}.

\bibitem[\citeproctext]{ref-jackeMethodenbeitragManuelleAnnotation2018}
---------. 2024a. Methodenbeitrag: {Manuelle} {Annotation}.
\emph{forTEXT Heft} 1, Nr. 4 (August). doi:
\href{https://doi.org/10.48694/fortext.3748}{10.48694/fortext.3748},
\url{https://fortext.net/routinen/methoden/manuelle-annotation}.

\bibitem[\citeproctext]{ref-lyman_responsive_1998}
Lyman, F. T. 1998. The responsive classroom discussion: {The} inclusion
of all students. In: \emph{Mainstreaming {Digest}}, hg. von A. S.
Anderson, 109--113. College Park: University of Maryland Press.

\bibitem[\citeproctext]{ref-rapp_manuelle_2017}
Rapp, Andrea. 2017. Manuelle und automatische {Annotation}. In:
\emph{Digital {Humanities}. {Eine} {Einführung}}, hg. von Fotis
Jannidis, Hubertus Kohle, und Malte Rehbein, 253--267.

\bibitem[\citeproctext]{ref-schumacherToolbeitragCATMA2019}
Schumacher, Mareike. 2024. Toolbeitrag: {CATMA}. \emph{forTEXT Heft} 1,
Nr. 4 (August). doi:
\href{https://doi.org/10.48694/fortext.3761}{10.48694/fortext.3761},
\url{https://fortext.net/tools/tools/catma}.

\bibitem[\citeproctext]{ref-zinsmeister_korpuslinguistik_2015}
Zinsmeister, Heike und Lothar Lemnitzer. 2015. \emph{Korpuslinguistik.
{Eine} {Einführung}}.

\end{CSLReferences}




\end{document}
