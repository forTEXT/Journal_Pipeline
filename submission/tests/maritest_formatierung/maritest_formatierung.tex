% Options for packages loaded elsewhere
\PassOptionsToPackage{unicode}{hyperref}
\PassOptionsToPackage{hyphens}{url}
\PassOptionsToPackage{dvipsnames,svgnames,x11names}{xcolor}
%
\documentclass[
          a4paper,
        ]{article}
\raggedbottom
\usepackage{amsmath,amssymb}
\usepackage{setspace}
\usepackage{iftex}
\RequireLuaTeX

\usepackage{lscape} % für Querformat-Seiten
\usepackage{hyperref}

\usepackage{unicode-math} % this also loads fontspec
\defaultfontfeatures{Scale=MatchLowercase}
\defaultfontfeatures[\rmfamily]{Ligatures=TeX,Scale=1}

  \usepackage[]{plex-otf}
  
  
  
  
  
  
  
\directlua{luaotfload.add_fallback
   ("fallbacks",
    {
      "STIX Two Math:;",
      "NotoColorEmoji:mode=harf;",
    }
   )}
\defaultfontfeatures{RawFeature={fallback=fallbacks}}

\usepackage{academicons}
\usepackage{fontawesome5}
\usepackage{ccicons}

  
\IfFileExists{upquote.sty}{\usepackage{upquote}}{}
\IfFileExists{microtype.sty}{% use microtype if available
  \usepackage[]{microtype}
  \UseMicrotypeSet[protrusion]{basicmath} % disable protrusion for tt fonts
}{}
\usepackage{lua-widow-control}
  \makeatletter
\@ifundefined{KOMAClassName}{% if non-KOMA class
  \IfFileExists{parskip.sty}{%
    \usepackage{parskip}
  }{% else
    \setlength{\parindent}{0pt}
    \setlength{\parskip}{6pt plus 2pt minus 1pt}}
}{% if KOMA class
  \KOMAoptions{parskip=half}}
\makeatother
  
  
\usepackage[dvipsnames,svgnames,x11names]{xcolor}

\definecolor{oa-orange}     {RGB} {246, 130, 18}
\definecolor{ft-red}        {RGB} {168, 0, 0}
\definecolor{orcid-green}   {RGB} {166, 206, 57}
\definecolor{fortext-green}   {RGB} {26, 73, 76}


\colorlet{oa-orange}        {black}

\colorlet{highlightcolor1}  {DarkOrange}
\colorlet{highlightcolor2}  {DarkBlue}
\colorlet{highlightcolor3}  {DarkGreen}
\colorlet{highlightcolor4}  {DarkMagenta}

\colorlet{filecolor}        {highlightcolor1}
\colorlet{linkcolor}        {fortext-green}
\colorlet{citecolor}        {fortext-green}
\colorlet{urlcolor}         {fortext-green}


\NewDocumentCommand \oalogo { }
{%
  \textcolor{oa-orange}{\sffamily%\bfseries%
  open \aiOpenAccess\ access}
}


\NewDocumentCommand \orcidlink { m }
{%
	\texorpdfstring
	{\href{https://orcid.org/#1}{\textcolor{orcid-green}{\raisebox{-.2ex}{\aiOrcid}}}}
	{https://orcid.org/#1}%
}

\NewDocumentCommand \emaillink { m }
{%
	\texorpdfstring
	{\href{mailto:#1}{\textcolor{black}{\raisebox{-.2ex}{\faEnvelopeOpen[regular]}}}}
	{mailto:#1}%
}

  \usepackage{geometry}
\geometry{
	paper=a4paper,
	top=25mm,
	bottom=20mm,
	right=30mm,
	left=30mm,
	footskip=10mm,
	% showframe,
}
  
  \usepackage{listingsutf8}
\lstset{
  language=python,
  basicstyle=\ttfamily\footnotesize,
  columns=fullflexible,
  xleftmargin=2em,
  xrightmargin=2em,
  % frame=single,
  breaklines=true,
  postbreak=\mbox{\textcolor{highlightcolor1}{\(\hookrightarrow\)}\space},
}
\newcommand{\passthrough}[1]{#1}
\lstset{defaultdialect=[5.3]Lua}
\lstset{defaultdialect=[x86masm]Assembler}
  
  
  
\usepackage{longtable,booktabs,array}
\usepackage{tabularray}
\usepackage{ifthen}

  
\usepackage{calc} % for calculating minipage widths
% Correct order of tables after \paragraph or \subparagraph
\usepackage{etoolbox}
\makeatletter
\patchcmd\longtable{\par}{\if@noskipsec\mbox{}\fi\par}{}{}
\makeatother
% Allow footnotes in longtable head/foot
\IfFileExists{footnotehyper.sty}{\usepackage{footnotehyper}}{\usepackage{footnote}}
\makesavenoteenv{longtable}
\usepackage{graphicx}
\usepackage[export]{adjustbox}
\makeatletter
\def\maxwidth{\ifdim\Gin@nat@width>\linewidth\linewidth\else\Gin@nat@width\fi}
\def\maxheight{\ifdim\Gin@nat@height>\textheight\textheight\else\Gin@nat@height\fi}
\makeatother
\setkeys{Gin}{width=.8\maxwidth,height=\maxheight,keepaspectratio}
\makeatletter
\def\fps@figure{hbp}
\makeatother

  
\usepackage{luacolor}
\usepackage[soul]{lua-ul}

\setlength{\emergencystretch}{3em} % prevent overfull lines
\providecommand{\tightlist}{%
  \setlength{\itemsep}{0pt}\setlength{\parskip}{0pt}}

  \setcounter{secnumdepth}{-\maxdimen} % remove section numbering
  
  
  
\usepackage{fancyhdr}

\fancypagestyle{plain}{% used for the first page
	\fancyhf{}% clear all header and footer fields
	\fancyhead[L]{}
	\fancyfoot[L]{}
	\fancyfoot[R]{\small\thepage}
	\renewcommand{\headrulewidth}{0pt}%
	\renewcommand{\footrulewidth}{0pt}%
}

\fancypagestyle{page}{%
	\fancyhf{}% clear all header and footer fields
	\fancyhead[L]{}
	\fancyhead[R]{\small forTEXT, \textit{Markdownformatierung}}
	\fancyfoot[L]{\small\textit{forTEXT} 2(13): Textannotation in der
Hochschullehre}
	\fancyfoot[R]{\thepage}
	\renewcommand{\headrulewidth}{0pt}%
	\renewcommand{\footrulewidth}{0pt}%
}

\pagestyle{page}


\usepackage[
	small,
	sf,bf,
	raggedright,
	clearempty,
]{titlesec}

\titleformat{\section}{\Large\sffamily\bfseries}{\thesection.}{.5em}{}
\titleformat{\subsection}{\large\sffamily\bfseries}{\thesubsection}{.5em}{}
\titleformat{\subsubsection}{\normalsize\sffamily\bfseries}{\thesubsubsection}{.5em}{}
\titleformat{\paragraph}{\normalsize\sffamily\bfseries}{\thesubsubsection}{.5em}{}
\titleformat{\subparagraph}{\normalsize\sffamily\bfseries}{\thesubsubsection}{.5em}{}

  \NewDocumentCommand\citeproctext{}{}
\NewDocumentCommand\citeproc{mm}{%
  \begingroup\def\citeproctext{#2}\cite{#1}\endgroup}
\makeatletter
 \let\@cite@ofmt\@firstofone
 \def\@biblabel#1{}
 \def\@cite#1#2{{#1\if@tempswa , #2\fi}}
\makeatother
\newlength{\cslhangindent}
\setlength{\cslhangindent}{1.5em}
\newlength{\csllabelwidth}
\setlength{\csllabelwidth}{3em}
\newenvironment{CSLReferences}[2] % #1 hanging-indent, #2 entry-spacing
 {\begin{list}{}{%
  \setlength{\itemindent}{0pt}
  \setlength{\leftmargin}{0pt}
  \setlength{\parsep}{0pt}
  % turn on hanging indent if param 1 is 1
  \ifodd #1
   \setlength{\leftmargin}{\cslhangindent}
   \setlength{\itemindent}{-1\cslhangindent}
  \fi
  % set entry spacing
  \setlength{\itemsep}{#2\baselineskip}}}
 {\end{list}}
\usepackage{calc}
\newcommand{\CSLBlock}[1]{\hfill\break\parbox[t]{\linewidth}{\strut\ignorespaces#1\strut}}
\newcommand{\CSLLeftMargin}[1]{\parbox[t]{\csllabelwidth}{\strut#1\strut}}
\newcommand{\CSLRightInline}[1]{\parbox[t]{\linewidth - \csllabelwidth}{\strut#1\strut}}
\newcommand{\CSLIndent}[1]{\hspace{\cslhangindent}#1}
  
  \usepackage[bidi=basic]{babel}
    \babelprovide[main,import]{ngerman}
                  \let\LanguageShortHands\languageshorthands
\def\languageshorthands#1{}
  
\usepackage[
	ragged,
	bottom,
	norule,
	multiple,
]{footmisc}

\makeatletter
\RenewDocumentCommand \footnotemargin { } {0em}
\RenewDocumentCommand \thefootnote { } {\arabic{footnote}}
\RenewDocumentCommand \@makefntext { m } {\noindent{\@thefnmark}. #1}
\interfootnotelinepenalty=10000
\makeatother

\usepackage{changepage}
% \newlength{\overhang}
% \setlength{\overhang}{\marginparwidth}
% \addtolength{\overhang}{\marginparsep}

  \usepackage{caption}
\captionsetup{labelformat=empty,font={small,it}}
  
\usepackage{selnolig}

  
  
  
  
\usepackage{csquotes}
\usepackage{bookmark}
\IfFileExists{xurl.sty}{\usepackage{xurl}}{} % add URL line breaks if available
\urlstyle{same}

  
  \usepackage[inkscapelatex=false]{svg}

\makeatletter
\def\@maketitle{%
	%
	% title
	%
	\newpage
	\null
	\vspace*{-\topskip}
	\begin{tblr}{
    vline{1,5} = {1pt},
    colspec={lllX[c,l]},
    width=\textwidth,
    columns={font=\small\sffamily,},
    column{1,3}={
      rightsep=.3em,
      font=\footnotesize\sffamily,
    },
}
\hline[1pt]
\SetCell[c=4]{t,l}{\normalsize\textbf{Markdownformatierung}} & & & \\
& & & \SetCell[r=3]{c,r}{\includegraphics[height=2\baselineskip]{}} \\
\SetCell[c=3]{t,l}{\small  forTEXT \emaillink{redaktion@fortext-hefte.de}
\textsuperscript{\scriptsize 1}
} & & &\\
\SetCell[c=2]{t,l}{% 
\footnotesize%
1. Technische Universität Darmstadt \\
} & & & \\
		\hline[1pt]
    Thema:                & Textannotation in der Hochschullehre & & \\
    DOI:                  & \href{https://doi.org/}{} & & \\
    Jahrgang:             & 2 & & \\
    Ausgabe:              & 13 & & \\
		Erscheinungsdatum:    &  & & \\
    Begutachtende:        & \href{mailto:mari@test.de}{Mari}, \href{mailto:janis@test.de}{Janis} & & \\
		Lizenz:               & \faCreativeCommons\ \faCreativeCommonsBy\ \faCreativeCommonsSa\ & & \oalogo  \\
		\hline[1pt]
	\end{tblr}
}
\makeatother

\hypersetup{
      pdftitle={Markdownformatierung},
      pdfauthor={ forTEXT},
      pdflang={de},
      pdfsubject={forTEXT 2(13): Textannotation in der Hochschullehre},
      pdfkeywords={Selbststudieneinheit, Blended
Learning, CATMA, Manuelle Annotation, Literaturwissenschaft},
      colorlinks=true,
  linkcolor={linkcolor},
  filecolor={filecolor},
  citecolor={citecolor},
  urlcolor={urlcolor},
  pdfcreator={LuaLaTeX via pandoc}
}


  
  \usepackage{ragged2e}
\usepackage[section]{placeins}
% Manage float placement
\usepackage{float}
\floatplacement{figure}{H}

\usepackage{marginnote}
\RenewDocumentCommand \marginfont { }
{ \sffamily\footnotesize }
\RenewDocumentCommand \raggedleftmarginnote { } { }
% \author{true}

% \date{}

\usepackage{everypage}
\usepackage[printwatermark]{xwatermark}
\usepackage{lipsum}
\AddEverypageHook{%
  \begin{picture}(0,0)
    \put(250,-500){\makebox(0,0){\scalebox{8}{\rotatebox{45}{\textcolor[gray]{0.90}{MANUSCRIPT}}}}}
  \end{picture}
}

\begin{document}



\pagestyle{plain}


\maketitle



% \marginnote{\RaggedRight}[30\baselineskip]%









\setstretch{1.2}


\pagestyle{page}

\renewcommand{\arraystretch}{3}  % Zeilenabstand in Tabellen erhöhen

\section{Markdownformatierungen}\label{markdownformatierungen}

Ich mlöchte dass
\textbackslash{}(\citeproc{ref-rendition}{\textbf{rendition?}})
angezeigt wird. wie geht das
„\textbackslash{}(\citeproc{ref-renditioN}{\textbf{renditioN?}})`` oder
„@rendition``

\begin{itemize}
\item
  Anführungszeichen und Bindestriche müssen nicht gesondert kodiert
  werden

  \begin{itemize}
  \tightlist
  \item
    Anführungszeichen in Markown bitte so hinterlegen: „Einführung in
    die Gattungstheorie: Prosa``
  \item
    Striche bei Seitenangaben „-`` oder „--``
  \end{itemize}
\item
  Codeblöcke werden durch ```python eingeführt und müssen entsprechend
  geschlossen werden:

\begin{lstlisting}[language=Python]
Textbeispiel im Codeblock: 
- @rendition ist mein Lieblingstag

Codebeispiel:
tagslist = ["@rendition", "@format", "@highlight"]
for x in tagslist:
    print(x)
\end{lstlisting}

  \begin{itemize}
  \tightlist
  \item
    Sonderzeichen, die in solchen Codeblöcken vorkommen müssen nicht
    weiter kodiert oder gekennzeichnet werden, um korrekt dargestellt zu
    werden
  \item
    Sonderzeichen, die nicht innerhalb eines Codeblocks vorkommen, bitte
    durch einen vorangestellten Backslash kennzeichnen, bspw.
    @rendition, @format, Lehrer*innen
  \end{itemize}
\end{itemize}

Noch ein weiterer Hinweis zu gültigen Intextreferenzen:

\begin{itemize}
\tightlist
\item
  Zimmermann (\citeproc{ref-Zimmermann-2000}{2000, 21--22}) behaupten,
  dass\ldots{} wird zu „Zimmermann (2000, 21--22) behaupten,
  dass\ldots{}``
\item
  Boekaerts und Niemivirta (\citeproc{ref-Boekaerts-2000}{2000,
  418--419, 432}) wird zu „Boekaerts und Niemivirta (2000, 418--419,
  432)``
\item
  (vgl. \citeproc{ref-Zimmermann-2000}{Zimmermann 2000, 21--22};
  \citeproc{ref-Boekaerts-2000}{Boekaerts und Niemivirta 2000, 418--419,
  432}) wird zu „(vgl. Zimmermann 2000, 21--22; Boekaerts und Niemivirta
  2000, 418--419, 432)``
\end{itemize}

\hfill\break
\hfill\break
\hfill\break
\hfill\break

\section{Gesamtablauf}\label{gesamtablauf}

\section{Elemente der Lehreinheit}\label{elemente-der-lehreinheit}

\begin{landscape}

\begin{longtable}[]{@{}
  >{\raggedright\arraybackslash}p{(\columnwidth - 16\tabcolsep) * \real{0.1083}}
  >{\raggedright\arraybackslash}p{(\columnwidth - 16\tabcolsep) * \real{0.0917}}
  >{\raggedright\arraybackslash}p{(\columnwidth - 16\tabcolsep) * \real{0.0917}}
  >{\raggedright\arraybackslash}p{(\columnwidth - 16\tabcolsep) * \real{0.1000}}
  >{\raggedright\arraybackslash}p{(\columnwidth - 16\tabcolsep) * \real{0.1167}}
  >{\raggedright\arraybackslash}p{(\columnwidth - 16\tabcolsep) * \real{0.1500}}
  >{\raggedright\arraybackslash}p{(\columnwidth - 16\tabcolsep) * \real{0.1500}}
  >{\raggedright\arraybackslash}p{(\columnwidth - 16\tabcolsep) * \real{0.1750}}
  >{\raggedright\arraybackslash}p{(\columnwidth - 16\tabcolsep) * \real{0.0167}}@{}}
\toprule\noalign{}
\begin{minipage}[b]{\linewidth}\raggedright
\textbf{Einheit}
\end{minipage} & \begin{minipage}[b]{\linewidth}\raggedright
\textbf{Modus}
\end{minipage} & \begin{minipage}[b]{\linewidth}\raggedright
\textbf{Thema}
\end{minipage} & \begin{minipage}[b]{\linewidth}\raggedright
\textbf{Inhalt}
\end{minipage} & \begin{minipage}[b]{\linewidth}\raggedright
\textbf{Lernziel}
\end{minipage} & \begin{minipage}[b]{\linewidth}\raggedright
\textbf{Vorbereitung}
\end{minipage} & \begin{minipage}[b]{\linewidth}\raggedright
\textbf{Für Lehrende}
\end{minipage} & \begin{minipage}[b]{\linewidth}\raggedright
\textbf{Abgabe/ Aufgabe}
\end{minipage} & \begin{minipage}[b]{\linewidth}\raggedright
\end{minipage} \\
\midrule\noalign{}
\endhead
\bottomrule\noalign{}
\endlastfoot
1 & Synchron, Präsenz & Theoretische Einführung in die Analyse von
Figuren & Einführung in die Figurenanalyse und das Formulieren von
Forschungsfragen & Grundlagen der Figurenanalyse verstehen und anwenden;
Forschungsfragen formulieren & Lektüre
(\citeproc{ref-Hansen_figuren_2016}{Hansen 2016}) & Laptop, Beamer,
vorbereitende Texte, Foliensatz „Figuren`` (siehe Guhr
(\citeproc{ref-Foliensatz_Figuren}{2024})) & Formulieren einer
Forschungsfrage zur Figurenanalyse & \\
2 & Asynchron, online & Selbststudium: Einführung in die
literaturwissenschaftliche Textannotation & Erarbeitung von
Einführungstexten und Video-Tutorials zur manuellen und digitalen
Annotation & Grundlagen des manuellen und digitalen Annotierens
verstehen; Anwendung des Tools CATMA & Lektüre
(\citeproc{ref-schumacherToolbeitragCATMA2019}{Schumacher 2024};
\citeproc{ref-jackeMethodenbeitragManuelleAnnotation2018}{Jacke 2024a};
\citeproc{ref-jackeMethodenbeitragKollaborativesLiteraturwissenschaftliches2018}{Jacke
2024b}), Anschauen der Tutorials
(\citeproc{ref-fortext_tutorial_2019}{forTEXT 2019a};
\citeproc{ref-fortext_tutorial_2020}{forTEXT 2020a};
\citeproc{ref-fortext_tutorial_2019-1}{forTEXT 2019b};
\citeproc{ref-fortext_tutorial_2019-2}{forTEXT 2019c};
\citeproc{ref-fortext_tutorial_2020-1}{forTEXT 2020b}) & Sicherstellen,
dass Links funktionieren und Materialien über Moodle verfügbar sind &
Durcharbeiten der Materialien, Vorbereitung auf die synchrone Sitzung
(inklusive der Einrichtung eines persönlichen CATMA-Accounts) & \\
3 & Synchron, Präsenz & Einführung in die literaturwissenschaftliche
Annotationspraxis & Zusammenfassende Einführung in die
literaturwissenschaftliche Annotationspraxis und praktische Anwendung
der Annotationsmethoden mit CATMA & Anwendung der Annotationsmethoden;
Reflexion der Nützlichkeit der Annotation & Vorbereitung des
CATMA-Projekts mit Tagset zur Figurenanalyse (orientiert an Jacke
(\citeproc{ref-jackeRessourcenbeitragTagsetNarratologie2020}{2024c}))
und Annotationsbeispielen in vorbereiteter Annotationscollection;
Formulieren von Übungsaufgaben zur Klausurvorbereitung & Laptop, Beamer,
Internetzugang, Zugriff auf CATMA & Eigenständige Annotation einer
Textpassage; Diskussion und Übung von Klausuraufgaben zu
literaturwissenschaftlicher Textannotation & \\
4 & Synchron, Präsenz & Überprüfung des Gelernten durch Klausuraufgabe &
Durchführung der Anwendungsaufgabe in der Klausur & Sicherstellung der
Lernzielerreichung & Diskussion und Übung von Prüfungsfragen zur
Textannotation als Teil der vorhergehenden Sitzung zur
Prüfungsvorbereitung & Bereitstellung der Prüfungsaufgabe und eines
Beispielprimärtextausschnitts & Formulieren einer Forschungsfrage,
Begründung von Nutzen und Relevanz von Textannotation, Erstellung eines
Annotationstagsets, Anwendung von Textannotation zur Näherung an die
Beantwortung der Forschungsfrage & \\
\end{longtable}

\end{landscape}

\section{Reflexion}\label{reflexion}

\section*{Bibliographie}\label{bibliography}
\addcontentsline{toc}{section}{Bibliographie}

\phantomsection\label{refs}
\begin{CSLReferences}{1}{0}
\bibitem[\citeproctext]{ref-Boekaerts-2000}
Boekaerts, Monique und Markku Niemivirta. 2000. Self-Regulated Learning:
Finding A Balance Between Learning Goals and Ego-Protective Goals. In:
\emph{Handbook of Self-Regulation}, hg. von Monique Boekaerts, Paul R.
Pintrich, und Moche Zeidner, 417--450. San Diego: Academic Press.

\bibitem[\citeproctext]{ref-fortext_tutorial_2019}
forTEXT. 2019a. Tutorial: {CATMA} 6 zur manuellen {Annotation} nutzen.
Oktober. doi:
\href{https://doi.org/10.5281/zenodo.10353556}{10.5281/zenodo.10353556},
\url{https://zenodo.org/records/10353556}.

\bibitem[\citeproctext]{ref-fortext_tutorial_2019-1}
---------. 2019b. Tutorial: {Projektmanagement} in {CATMA} 6. November.
doi:
\href{https://doi.org/10.5281/zenodo.10353713}{10.5281/zenodo.10353713},
\url{https://zenodo.org/records/10353713}.

\bibitem[\citeproctext]{ref-fortext_tutorial_2019-2}
---------. 2019c. Tutorial: {Tagsets} in {CATMA} 6 anlegen. Dezember.
doi:
\href{https://doi.org/10.5281/zenodo.10377968}{10.5281/zenodo.10377968},
\url{https://zenodo.org/records/10377968}.

\bibitem[\citeproctext]{ref-fortext_tutorial_2020}
---------. 2020a. Tutorial: {In} {CATMA} 6 annotieren. Januar. doi:
\href{https://doi.org/10.5281/zenodo.10353910}{10.5281/zenodo.10353910},
\url{https://doi.org/10.5281/zenodo.10353910}.

\bibitem[\citeproctext]{ref-fortext_tutorial_2020-1}
---------. 2020b. Tutorial: {Analysieren} und visualisieren mit {CATMA}.
Februar. doi:
\href{https://doi.org/10.5281/zenodo.10276637}{10.5281/zenodo.10276637},
\url{https://doi.org/10.5281/zenodo.10276637}.

\bibitem[\citeproctext]{ref-Foliensatz_Figuren}
Guhr, Svenja. 2024. Foliensatz: Figurenanalyse.

\bibitem[\citeproctext]{ref-Hansen_figuren_2016}
Hansen, Per Krogh. 2016. {IV}. 3.3 {Figuren}. In: \emph{Einführung in
die {Erzähltextanalyse}}, hg. von Silke Lahn und Jan Christoph Meister,
übers. von Marie Isabel Schlinzig, 234--249. 3. Aufl. Lehrbuch.
Stuttgart: J. B. Metzler Verlag.
\url{https://doi.org/10.1007/978-3-476-05415-9} (zugegriffen: 27. Januar
2021).

\bibitem[\citeproctext]{ref-jackeRessourcenbeitragTagsetNarratologie2020}
Jacke, Janina. 2024c. Ressourcenbeitrag: {Tagset} {Narratologie}
(histoire). \emph{forTEXT Heft} 1, Nr. 4 (August). doi:
\href{https://doi.org/10.48694/fortext.3757}{10.48694/fortext.3757},
\url{https://fortext.net/ressourcen/tagsets/tagset-narratologie-histoire}.

\bibitem[\citeproctext]{ref-jackeMethodenbeitragKollaborativesLiteraturwissenschaftliches2018}
---------. 2024b. Methodenbeitrag: {Kollaboratives}
literaturwissenschaftliches {Annotieren}. \emph{forTEXT Heft} 1, Nr. 4
(August). doi:
\href{https://doi.org/10.48694/fortext.3749}{10.48694/fortext.3749},
\url{https://fortext.net/routinen/methoden/kollaboratives-literaturwissenschaftliches-annotieren}.

\bibitem[\citeproctext]{ref-jackeMethodenbeitragManuelleAnnotation2018}
---------. 2024a. Methodenbeitrag: {Manuelle} {Annotation}.
\emph{forTEXT Heft} 1, Nr. 4 (August). doi:
\href{https://doi.org/10.48694/fortext.3748}{10.48694/fortext.3748},
\url{https://fortext.net/routinen/methoden/manuelle-annotation}.

\bibitem[\citeproctext]{ref-schumacherToolbeitragCATMA2019}
Schumacher, Mareike. 2024. Toolbeitrag: {CATMA}. \emph{forTEXT Heft} 1,
Nr. 4 (August). doi:
\href{https://doi.org/10.48694/fortext.3761}{10.48694/fortext.3761},
\url{https://fortext.net/tools/tools/catma}.

\bibitem[\citeproctext]{ref-Zimmermann-2000}
Zimmermann, Barry J. 2000. Attaining Self-Regulation: A Social Cognitive
Perspective. In: \emph{Handbook of Self-Regulation}, hg. von Monique
Boekaerts, Paul R. Pintrich, und Moche Zeidner, 13--39. San Diego:
Academic Press.

\end{CSLReferences}




\end{document}
